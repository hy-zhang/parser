\section{Conclusion}\label{sec:conclusion}

This paper presents a solution for type-safe modular parsing. Our solution
not only enables parsers to evolve together with the abstract syntax,
but also allows parsing code to be modularly type-checked and separately compiled.

We identify the algorithmic challenges of building modular parsers,
and use standard OO techniques including inheritance and overriding
for our goal. However, the extensibility issue of
traditional OO ASTs motivates us to adopt Object Algebras for full
extensibility and more useful features. Then language feature abstraction further enhances code reuse and modularity.
The TAPL case study demonstrates that a lot of boilerplate can be reduced by
modular parsing.

There are certainly some aspects that can be improved. We observed that the
glue code of composition appears to be boilerplate, for which family polymorphism~\cite{ernst01FP}
is a potential solution. Moreover, we can possibly
adopt the Shy framework~\cite{Zhang2015} and algebra composition
patterns~\cite{oliveira2013feature}, to improve the usage of Object
Algebras. For future work, it will be interesting to see how modular parsing
appears in functional programming languages, as they usually do not support subtyping
or inheritance. Potentially open recursion~\cite{CookThesis} can contribute.
%It is also interesting to see that parsing, to some
%extent, can be viewed as a special example of \textit{unfolds}. So it
%is worthwhile considering to generalize our approach under certain
%circumstances.
