\section{Full Extensibility with Object Algebras}\label{sec:algebrasandparsing}

The inheritance-based approach allows building extensible parsers, based on an OO class hierarchy.  Nevertheless, the addition of new operations over ASTs is problematic using traditional OO ASTs. In this section, we
show how to support both forms of extensibility on ASTs (easy
addition of language constructs, and easy addition of operations)
using Object Algebras~\cite{Oliveira:2012}.

\subsection{Problem with Traditional OO ASTs}\label{subsec:problemwithoutoa}

The Expression Problem~\cite{wadler1998expression} illustrates the
difficulty of extending data structures or ASTs in two dimensions.
In brief, it is hard to add new operations with traditional OO ASTs.
In the last section we have seen a language
that supports pretty-printing (in \inlinecode{Expr}).
To modularly add an operation like collecting free variables, one attempt
would be extending
\inlinecode{Expr} with the new operation to obtain a new abstract type
for ASTs:

\lstinputlisting[linerange=23-23]{code/src/papercode/Sec4OA/Code1.scala}% APPLY:linerange=FVARSEXPR

\noindent Then all classes representing language constructs could be
extended to implement the operation. A first well-known problem is that such
approach is problematic in terms of type-safety (but see recent work
by Wang and Oliveira~\cite{wang2016expression}, which shows a technique that is type-safe
in many cases). More importantly, a second problem is that
even if that approach would work, the parsing code in \lstinline{VarParser} is no longer reusable!
The types \inlinecode{Expr, Lit, Add}, and so on, are all old types without the free variables operation.
To match the new ASTs, we have to substitute \inlinecode{NewExpr} for \inlinecode{Expr} (the
same for \inlinecode{Lit}, \inlinecode{Add}, ...). This requires either code modification or type casts.
The goal of semantic modularity motivates us to find a different approach for building ASTs.

%Object Algebras separate data variants and
%operations, and offer high flexibility in the choice of operations to
%be performed over ASTs. The next section provides a brief introduction
%to Object Algebras.

\subsection{Object Algebras}\label{subsec:objectalgebras}

Fortunately, Object Algebras~\cite{Oliveira:2012} enable us to solve
this problem.
They capture a design pattern that addresses the Expression
Problem, achieving two dimensions of extensibility (language
constructs and operations) in a modular and type-safe way. The
definition of data structures is separated from their behaviours, and
future extensions on both dimensions no longer require existing code
to be modified, supporting separate compilation.

Using Object Algebras in Scala, ASTs as recursive data structures are defined by traits, where each constructor corresponds to an abstract method inside.
Essentially, Object Algebras generalize the {\sc Abstract Factory} pattern~\cite{gamma1995design}, and promote the use of factory methods, instead of constructors, for instantiating objects. The example from Section~\ref{subsec:packratparsing} is used here again for illustration.
At first the language only supports literals and additions:

\lstinputlisting[linerange=5-8]{code/src/papercode/Sec4OA/Code2.scala}% APPLY:linerange=OVERVIEW_OA_ALG
Here \inlinecode{Alg} is called an \textit{Object Algebra interface}, parameterized by the type
\inlinecode{E}, which abstracts over the concrete type of the AST.

\paragraph{Adding New Operations}
To realize an operation on expressions, we simply instantiate the type parameter by a concrete type and
provides implementations for all cases. Below is an example of pretty-printing:

\lstinputlisting[linerange=12-16]{code/src/papercode/Sec4OA/Code2.scala}% APPLY:linerange=OVERVIEW_OA_PRINT
Here \inlinecode{Print} is called an \textit{Object Algebra}. It traverses an expression bottom-up, and returns a string as the result.
One can also define an evaluation operation as a new trait that extends \inlinecode{Alg[Int]}. Hence adding new operations is modular. We omit that code
due to space reasons.

%\lstinputlisting[linerange=20-23]{code/src/papercode/Sec4OA/Code2.scala}% APPLY:linerange=OVERVIEW_OA_EVAL

\paragraph{Adding New AST Constructs}
Furthermore, new language constructs can be added by extending \inlinecode{Alg} and adding new cases only. Now we extend the language
with variables. A new Object Algebra interface \inlinecode{VarAlg} is defined as follows:

\lstinputlisting[linerange=27-29]{code/src/papercode/Sec4OA/Code2.scala}% APPLY:linerange=OVERVIEW_OA_ALGEXT
Now pretty-printing on the new language can be realized without modifying existing code:

\lstinputlisting[linerange=33-35]{code/src/papercode/Sec4OA/Code2.scala}% APPLY:linerange=OVERVIEW_OA_EXTPRINT
An observation is that only the new case is implemented for pretty-printing, and the others have been inherited.
Thus existing code was reused and was not modified!

To create an expression representing \inlinecode{1 + x}, a generic method is defined as follows:

\lstinputlisting[linerange=42-43]{code/src/papercode/Sec4OA/Code2.scala}% APPLY:linerange=OVERVIEW_OA_MAKEEXP

Note how the construction of the abstract syntax happens through
the use of factory methods, instead of constructors.
To pretty-print the expression, the code \inlinecode{"makeExp(new VarPrint \{\})"}
results in \inlinecode{"(1 + x)"} as expected.

\subsection{Parsing with Object Algebras}\label{subsec:parsingwithoa}

Parsing produces ASTs as the result. When Object Algebras are used
to build ASTs, an Object Algebra containing the constructor/factory
methods has to be used by the parsing function. Thus, a first attempt
at defining the parser for the small arithmetic language is:

\lstinputlisting[linerange=8-17]{code/src/papercode/Sec4OA/Code3.scala}% APPLY:linerange=BASE_OA_PARSER_BAD
Such a parser looks fine, but it is not extensible. For example, we have demonstrated in Section~\ref{sec:inheritance} that method overriding is essential to update \inlinecode{pExpr} for an extended syntax. However, trying to do a similar method overriding for \inlinecode{pExpr} would require a type \inlinecode{VarAlg[E] =>} \inlinecode{Parser[E]}, which is a \emph{supertype} of the old type \inlinecode{Alg[E] =>} \inlinecode{Parser[E]}, since
the \emph{extended} Object Algebra interface appears in \emph{contravariant} position. This violates
overriding in Scala.


\paragraph{A Solution}
A solution to this problem is to declare a field of Object Algebra interface
in the parser. Figure~\ref{fig:oa-parser} shows the code of true modular parser,
whose methods can be overridden for future extension.

\begin{figure}
  \lstinputlisting[linerange=21-30]{code/src/papercode/Sec4OA/Code3.scala}% APPLY:linerange=BASE_OA_PARSER
  \caption{Pattern of modular parsing using Object Algebras.}
  \label{fig:oa-parser}
  \vspace{-0.2cm}
\end{figure}

That is precisely the pattern that we advocate for modular parsing.
One important remark is we introduce \inlinecode{pE} for recursive calls.
The reason why we use it as an extra and seemingly redundant field, is due to a subtle issue caused by Scala language and its parser combinator library. There is a restriction of \inlinecode{super} keyword in Scala that \inlinecode{super} can only use methods defined by keyword \inlinecode{def}, but cannot access fields defined by \inlinecode{val}, while the parser combinator library suggests using \inlinecode{val} to define parsers, especially for left-recursive ones. Our workaround is that we use different synonyms for \inlinecode{pE} in different traits, so that we can directly distinguish them by names without using \inlinecode{super}.

\paragraph{Extensions}
Now let's try on the variables extension:

\lstinputlisting[linerange=34-39]{code/src/papercode/Sec4OA/Code3.scala}% APPLY:linerange=EXT_OA_PARSER

\noindent The type of the Object Algebra field \lstinline{alg} is first refined
to \lstinline{VarAlg[E]}, to allow calling the additional factory method
for variables. Unlike the previous attempt, such a type-refinement is allowed.
Now, the code for parsing variables (\lstinline{pVar}) can
call \lstinline{alg.varE}. The following code illustrates how to use
the parser from a client's perspective:

\lstinputlisting[linerange=46-49]{code/src/papercode/Sec4OA/Code3.scala}% APPLY:linerange=EXT_OA_PARSER_CLIENT

In the client code above,
we pick the pretty-printing algebra \lstinline{VarPrint} to initialize the \lstinline{alg} field, but any other Object
Algebra that implements \lstinline{VarAlg} would work.
With an instance of \lstinline{VarOAParser} in hand, we can call
\lstinline{pE} to obtain the parser to feed to the \lstinline{parse} method.
Such a pattern provides modular parsing as expected.

Note that, similar to the approach in Section~\ref{sec:inheritance}, independent extensibility is also supported via multiple trait inheritance.
Since it is achieved using essentially the same technique as in Section~\ref{sec:inheritance}, we omit the code here.
