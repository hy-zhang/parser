\vspace{-2pt}
\section{Packrat Parsing for Modularity}\label{sec:packrat}

This section discusses the algorithmic challenges introduced by
modular parsing and argues that Packrat parser combinators~\cite{Ford2002}
are suitable to address them. The algorithmic challenges
are important because they rule out various common techniques
used by non-modular code using parser combinators.
To avoid pitfalls related to those algorithmic challenges,
we propose the following methodology:

\vspace{-2pt}
\begin{itemize}[leftmargin=*]

\item \textbf{Modular parsers should support left-recursion.}

\item \textbf{Modular parsers should use a longest match composition operator.}

\end{itemize}
\vspace{-2pt}

Moreover, the underlying parsing formalism should make \emph{backtracking
cheap}, due to its pervasiveness in modular parsing.
Although we chose Packrat parsing, any other parsing formalism that provides similar
features should be ok.

\subsection{Algorithmic Challenges of Modularity}\label{subsec:challenges}
For the goal of modular parsing, parser combinators seem suitable because they are
naturally modular for parser composition, but also they ensure type safety.
Unfortunately many parser combinators have important limitations.
In particular, several parser combinators including the famous Parsec~\cite{Leijen2001} library, require
programmers to manually do \textit{left-recursion elimination}, \textit{longest match composition}, and
require significant amounts of \textit{backtracking}. All of those are
problematic in a modular setting.

\paragraph{Left-Recursion Elimination} The top-down, recursive descent parsing strategy adopted by those parser combinator libraries cannot support left-recursive grammars directly. For instance, we start with a simple arithmetic language containing only integers and subtractions. The
grammar with concrete syntax and part of the parsing code in Parsec are presented below:

\begin{minipage}{12em}
\setlength{\grammarindent}{4em}
\begin{grammar}
<expr> ::= <int> \alt <expr> `-' <int>
\end{grammar}
\end{minipage}
\begin{minipage}{15em}
\begin{lstlisting}[language=PlainCode]
parseExpr =
  parseSub <|> parseInt
parseSub = do
  e <- parseExpr ...
\end{lstlisting}
\end{minipage}
Such a left-recursive implementation will cause an infinite loop, since \inlinecode{parseExpr} and \inlinecode{parseSub} call each other and never stop.
A common solution is to rewrite the grammar into an equivalent but non-left-recursive one, called left-recursion elimination:

\vspace{0.4\baselineskip}
\begin{minipage}{15em}
\setlength{\grammarindent}{5em}
\begin{grammar}
<expr> ::= <int> <expr'>

<expr'> ::= <empty> \alt `-' <int> <expr'>
\end{grammar}
\end{minipage}
\vspace{0.4\baselineskip}

\noindent After left-recursion elimination, the structure of grammar is changed,
as well as its corresponding parser. In a modular setting, it is
possible but unnecessarily complicated to analyse the grammar and
rewrite it when doing extensions. Anticipating that every non-terminal
has left-recursive rules is helpful for extensibility but overkill,
since it is inconvenient and introduces extra complexity for
representation of grammars and implementation of parsers.

Another issue of left-recursion elimination is that it requires extra
bookkeeping work to retain the original semantics. For example, the
expression $1-2-3$ is parsed as $(1-2)-3$ in the left-recursive
grammar, but after rewrite the information of left-associativity is lost. The parse tree
must be transformed to recover the correct syntactic structure.

\paragraph{Longest Match Composition} Another problematic issue
in parser combinator libraries is the need for manually prioritizing/ordering
alternatives in a grammar.
Consider the grammar:
\vspace{-2pt}
\setlength{\grammarindent}{5em}
\begin{grammar}
<expr> ::= <int> \alt <int> `+' <expr>
\end{grammar}
\vspace{-2pt}

In Parsec, for instance, the parser \inlinecode{"parseInt <|> parseAdd"} will only parse the input \inlinecode{"1 + 2"} to \inlinecode{"1"}, as \inlinecode{parseInt} successfully parses \inlinecode{"1"}
and terminates parsing.

Traditional alternative composition will only find the first parser that succeeds on a prefix of the input,
even if subsequent parsers may parse the whole input. In contrast to the previous parser, \inlinecode{"parseAdd <|> parseInt"} works as expected with because the two cases are swapped.
In this case, reordering the alternatives ensures that
the \emph{longest match} is picked among the possible results. However, manual reordering for the longest match is inconvenient, and worst still, it is essentially non-modular. When the grammar is extended with new rules, programmers should \emph{manually} adjust
the order of parsers, by rewriting previously written code.

\paragraph{Backtracking} The need for backtracking can also be problematic
in a modular setting. Consider a grammar with \inlinecode{"import..from"}, 
and is extended with an \inlinecode{"import..as"} case:

\vspace{-2pt}
\setlength{\grammarindent}{5em}
\begin{grammar}
<stmt> ::= `import' <ident> `from' <ident>
    \alt ...
    \alt `import' <ident> `as' <ident>
\end{grammar}
\vspace{-2pt}
Since the two cases share a common prefix, when the former fails, we must backtrack to the beginning.
For example, the choice combinator in Parsec only tries the second alternative if the first fails
without any token consumption. We have to use \inlinecode{try} for explicit backtracking.

\vspace{-4pt}
\begin{lstlisting}[language=PlainCode]
oldParser = parseImpFrom <|> ...
newParser = try parseImpFrom <|> ... <|> parseImpAs
\end{lstlisting}
\vspace{-4pt}

Similarly, this violates a modular setting because it also requires a global view of the full grammar.
Hence the worst case where all alternatives may share common prefixes with future cases
should always be anticipated. Therefore we need to backtrack for all the branches. To avoid failures in the future, we have to add \inlinecode{try} everywhere. However this results in the worst-case exponential time
complexity.

\subsection{Packrat Parsing}\label{subsec:packratparsing}
Fortunately, some more advanced parsing techniques such as Packrat
parsing~\cite{Ford2002} have been developed to address limitations of
simple parser combinators. Packrat parsing uses memoization to record
the result of applying each parser at each position of the input, so
that repeated computation is eliminated. Moreover, it supports both direct left-recursion
and (in theory) indirect left-recursion~\cite{warth2008}.
All of these properties are very suitable for modularity, thus we decided to use Packrat parsers as the underlying parsing technique for modular parsing.
\begin{comment}
It is worth mentioning that the choice of parser combinators will not
affect the other parts of our library. One can choose other parser
combinators like Parsec, in cases that the performance and supporting
of left-recursion are not major concerns. A different library can even build a new
\name with fancy features or higher efficiency.
\end{comment}
Scala has a standard parser combinator library\footnote{https://github.com/scala/scala-parser-combinators}
\cite{moors2008parser} for Packrat parsers.
The library provides a number of parser combinators, including the longest match alternative combinator.
%Below we present an example to illustrate Scala parsers.

\paragraph{Code Demonstration}
For more concise demonstration, we assume that all the Scala code in the rest of this paper are in the object \inlinecode{Code}, as shown in Figure~\ref{fig:papercode}. It extends traits \inlinecode{StandardTokenParsers} and \inlinecode{PackratParsers} from the Scala parser combinator library. Furthermore, we will use \lstinline{Parser} as a type synonym for \lstinline{PackratParser} and a generic \inlinecode{parse} function for testing.

\begin{figure}[t]
\centering
\lstinputlisting[linerange=16-27]{code/src/papercode/Sec2Packrat/Code2.scala}% APPLY:linerange=PACKRAT_PAPERCODE
\caption{Helper object for code demonstration in this paper.}\label{fig:papercode}
\end{figure}

\paragraph{Parsing a Simple Arithmetic Language}
Suppose we want to parse a simple language with literals and
additions. The concrete syntax is:

\setlength{\grammarindent}{5em}
\begin{grammar}
<expr> ::= <int>
    \alt <expr> `+' <expr>
\end{grammar}

It is straightforward to model the abstract syntax by classes. The ASTs support pretty-printing via the \lstinline{print} method.

\vspace{-4pt}
\lstinputlisting[linerange=9-15]{code/src/papercode/Sec2Packrat/Code1.scala}% APPLY:linerange=PACKRAT_OO_AST
\vspace{-4pt}

Then we write corresponding parsers for all cases.
Note that a parser has type \lstinline{Parser[E]} for some
\lstinline{E}, which indicates the type of results it produces.

\vspace{-4pt}
\lstinputlisting[linerange=19-26]{code/src/papercode/Sec2Packrat/Code1.scala}% APPLY:linerange=PACKRAT_SIMPLE_EXPR
\vspace{-4pt}

In the trait \lstinline{AParser}, \lstinline{lexical} is used for lexing. \lstinline{pLit} parses an integer for the literal case.
\lstinline{pAdd} handles the addition case and creates an object of \lstinline{Add}. It parses two sub-expressions by calling \lstinline{pExpr}
recursively. Finally \lstinline{pExpr} composes \lstinline{pLit} and \lstinline{pAdd} using the longest match alternative combinator \inlinecode{|||}.
Table~\ref{tab:packrat} shows common parser combinators from the library.

\begin{table}
\begin{tabularx}{\linewidth}{X}
\hline
\lstinline{def ~[U](q: => Parser[U]): Parser[~[T, U]]} \\
- A parser combinator for sequential composition. \\ \hline
\lstinline{def \^\^[U](f: (T) => U): Parser[U]} \\
- A parser combinator for function application. \\ \hline
%\lstinline{def \^\^\^[U](v: => U): Parser[U]} \\
%- A parser combinator that changes a successful result into the specified value. \\ \hline
\lstinline{def <~[U](q: => Parser[U]): Parser[T]} \\
- A parser combinator for sequential composition which keeps only the left result. \\ \hline
\lstinline{def ~>[U](q: => Parser[U]): Parser[U]} \\
- A parser combinator for sequential composition which keeps only the right result. \\ \hline
\lstinline{def ident: Parser[String]} \\
- A parser which matches an identifier. \\ \hline
\lstinline{def |[U >: T](q: => Parser[U]): Parser[U]} \\
- A parser combinator for alternative composition. \\ \hline
\lstinline{def |||[U >: T](q0: => Parser[U]): Parser[U]} \\
- A parser combinator for alternative with longest match composition. \\ \hline
\end{tabularx}
\caption{Common combinators from the Scala standard parser combinator library.}\label{tab:packrat}
\end{table}

It is worth mentioning that the left-recursive grammar above is well supported without extra code.
 The longest match composition is also employed by using the combinator \lstinline{|||}. Furthermore, the parser does not suffer from the backtracking problem, as the memoization technique of Packrat parsing guarantees reasonable efficiency.

The code below demonstrates how to parse a valid expression \inlinecode{1 + 2} using our parser.

\vspace{-4pt}
\lstinputlisting[linerange=6-7]{code/src/papercode/Sec2Packrat/Code2.scala}% APPLY:linerange=PACKRAT_RUNPARSER
\vspace{-10pt}
