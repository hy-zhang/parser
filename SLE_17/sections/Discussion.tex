\section{More Features}

The use of inheritance-based approach and Object Algebras enables us
to build modular parsers, which are able to evolve with
syntax together. This section explores more interesting features, including
parsing multi-sorted syntax, overriding existing parsing rules,
language components for abstracting language features, and alternative
techniques under the whole framework.

\subsection{Parsing Multi-Sorted Syntax}\label{subsec:differentsyntax}

\begin{comment}
As illustrated above, using Object Algebras separates data structures from behaviors, thus enabling more modularity and reuse. New language constructs correspond to the new cases in the algebra. Different operations
 on structures, with both code reuse and separate compilation supported.
\end{comment}

Using Object Algebras, it is easy to
model multi-sorted languages. If the syntax has multiple sorts, we can distinguish them by different type parameters.
For instance, we extend the expression language from the end of Section~\ref{sec:algebrasandparsing}, with a
primitive type \inlinecode{int} type and typed lambda abstractions:

\vspace{0.8mm}

\setlength{\grammarindent}{5em}
\begin{grammar}
<type> ::= `int'

<expr> ::=  ... \alt `\\' <ident> `:' <type> `.' <expr>
\end{grammar}

\vspace{0.8mm}

The code below illustrates the corresponding Scala code
that extends the Object Algebra interface, pretty-printing operation and parser.

\lstinputlisting[linerange=7-23]{code/src/papercode/Sec4OA/Code4.scala}% APPLY:linerange=OVERVIEW_OA_MULTI_SYNTAX

We use two type parameters \inlinecode{E} and \inlinecode{T} for expressions and types. The type system guarantees that invalid terms such as \inlinecode{int + int} will be rejected.
Besides lexing, the trait \inlinecode{LamOAParser} also introduces parsers for types, and the new case for expressions.
We use \inlinecode{pTypedLamT} and \inlinecode{pTypedLamE} as copies of current \inlinecode{pT} and \inlinecode{pE}, due to the issue
with \inlinecode{super} in Scala (see discussion in Section~\ref{subsec:parsingwithoa}). \inlinecode{pT} and \inlinecode{pE} are used for recursion.

\subsection{Overriding Existing Rules}\label{subsec:overriding-rules}

As many syntactically extensible parsers, our approach also supports
modifying part of existing parsers, including updating or eliminating existing rules,
but in a type-safe way. This can be useful in many
situations, for instance when conflicts or ambiguities arise upon composing languages.
As an illustration, suppose we have an untyped lambda abstraction case in a base parser, defined as a value:
\begin{lstlisting}
val pLam: Parser[E] =
  ("\\" ~> ident) ~ ("." ~> pE) ^^ ...
\end{lstlisting}
Here \inlinecode{pLam} parses a lambda symbol, an identifier, a dot and an expression in sequence.
Then we want to replace the untyped lambda abstractions by typed
lambdas. With inheritance and method overriding, it is easy to only
change the implementation of \lstinline{pLam} in the extended parser.
Due to dynamic dispatch, our new
implementation of lambdas will be different without affecting the other parts of the parser.
\begin{lstlisting}
override val pLam: Parser[E] =
  ("\\" ~> ident) ~ (":" ~> pT) ~ ("." ~> pE) ^^ ...
\end{lstlisting}


One can even ``eliminate'' a production rule in the extension, by overriding it with a failure parser. The lexer can also be updated, since keywords and delimiters are represented by sets of strings.


\subsection{Language Components}\label{subsec:language-component}

Modular parsing not only enables us to build a corresponding parser
which evolves with the language together, but also allows us to
abstract language features as reusable, independent components.
Generally, a language feature includes related abstract syntax,
methods to \textit{build} the syntax (parsing), and methods to
\textit{process} the syntax (evaluation, pretty-printing, etc.). From
this perspective, not only one language, but many languages can be
developed in a modular way, with common language features reused.

Instead of designing and building a language from scratch, we can
easily add a new feature by reusing the corresponding language
component. For example, if a language is composed from a component of
boolean expressions, including if-then-else, it immediately knows how
to parse, traverse, and pretty-print the if-then-else structure.
Grouping language features in this way can
be very useful for rapid development of DSLs.

For implementation, a language component is represented by a Scala object, and it consists of three parts: Object Algebra interface, parser, and Object Algebras.

\begin{itemize}[leftmargin=*]
    \item \textbf{Object Algebra interface:} defined as a trait for the abstract syntax. The type parameters represent multiple sorts of syntax, and  methods are constructs.
    \item \textbf{Parser:} corresponding parser of the abstract syntax, written in a modular way as we demonstrated before.
    \item \textbf{Object Algebras (optional):} concrete operations on ASTs, such as pretty-printing.
\end{itemize}

We take the example in Section~\ref{subsec:parsingwithoa} again. It can be defined as a language component \inlinecode{VarExpr}.
For space reasons we omit some detailed code.

\lstinputlisting[linerange=115-124]{code/src/papercode/Sec5/Code1.scala}% APPLY:linerange=LANGUAGE_COMPONENTS_VAREXPR

For the extension of types and lambda abstractions in Section~\ref{subsec:differentsyntax}, instead of inheriting from the previous language directly, we can define it as another independent language component \inlinecode{TypedLam}.

\lstinputlisting[linerange=128-137]{code/src/papercode/Sec5/Code1.scala}% APPLY:linerange=LANGUAGE_COMPONENTS_TYPEDLAM

The code below shows how we merge those two components together to obtain the language we want. Furthermore, the new language is still a modular
component ready for future composition. In that case modularity is realized over higher-order hierarchies.

\lstinputlisting[linerange=141-152]{code/src/papercode/Sec5/Code1.scala}% APPLY:linerange=LANGUAGE_COMPONENTS_VARLAMEXPR

The only drawback is that the glue code of composition appears to be
boilerplate. As shown above, we are combining ASTs, parsers and
pretty-printers of \lstinline{VarExpr} and \lstinline{TypedLam}
respectively. Such a pattern refers to \textit{family
  polymorphism}~\cite{ernst01FP} which is unfortunately not fully supported
in Scala, since nested classes/traits have to be manually composed.
%Nonetheless, one can avoid such boilerplate by using metaprogramming techniques.

\subsection{Alternative Techniques}

Our prototype uses Packrat parsing as the underlying parsing technique, OO inheritance for composing and extending parsers, and Object Algebras for parsing extensible ASTs. Yet such a framework is itself flexible and modular, because those techniques can have alternatives.
For example, as we mentioned before, any parsing library
that resolves the algorithmic challenges in modular parsing can work well. Regarding OO inheritance for the extensibility, an alternative approach, called \textit{open recursion}~\cite{CookThesis} can be used in other languages, by introducing explicit ``self-reference'' parameters for the recursion. Furthermore, besides Object Algebras, v\textit{Data types à la carte} (DTC)~\cite{swierstra2008data} and the Cake pattern~\cite{odersky2005independently} also support extensible data structures. For the goal of modular parsing a custom combination of those alternatives can be adopted.
