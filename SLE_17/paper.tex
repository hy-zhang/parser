%% For double-blind review submission
\documentclass[sigplan,10pt,review,anonymous]{acmart}\settopmatter{printfolios=false}
%% For single-blind review submission
%\documentclass[sigplan,10pt,review]{acmart}\settopmatter{printfolios=true}
%% For final camera-ready submission
%\documentclass[sigplan,10pt]{acmart}\settopmatter{}

%% Note: Authors migrating a paper from traditional SIGPLAN
%% proceedings format to PACMPL format should change 'sigplan,10pt' to
%% 'acmlarge'.


%% Some recommended packages.
\usepackage{booktabs}   %% For formal tables:
                        %% http://ctan.org/pkg/booktabs
\usepackage{subcaption} %% For complex figures with subfigures/subcaptions
                        %% http://ctan.org/pkg/subcaption

\usepackage{listings}
\usepackage{syntax}
\usepackage[T1]{fontenc}
\usepackage[scaled=0.85]{beramono}
\usepackage{multirow}
\usepackage{enumitem}
\usepackage{tabularx}

\lstdefinelanguage{JavaScala}{
  morekeywords={public,int,interface,implements,default,
    abstract,case,catch,class,def,static,%
    do,else,extends,false,final,finally,%
    for,if,implicit,import,match,mixin,%
    new,null,object,override,package,%
    private,protected,requires,return,sealed,%
    super,this,throw,trait,true,try,%
    type,var,while,yield,with,val},
  otherkeywords={=>,<\%,<:,>:,\#,@},
  sensitive=true,
  morecomment=[l]{//},
  morecomment=[n]{/*}{*/},
  morestring=[b]",
  morestring=[b]',
  morestring=[b]"""
}

\lstset{
  language=JavaScala,                % choose the language of the code
  columns=flexible,
  lineskip=-1pt,
  basicstyle=\ttfamily\small,       % the size of the fonts that are used for the code
  numbers=none,                   % where to put the line-numbers
  numberstyle=\ttfamily\tiny,      % the size of the fonts that are used for the line-numbers
  stepnumber=1,                   % the step between two line-numbers. If it's 1 each line will be numbered
  numbersep=5pt,                  % how far the line-numbers are from the code
  backgroundcolor=\color{white},  % choose the background color. You must add \usepackage{color}
  showspaces=false,               % show spaces adding particular underscores
  showstringspaces=false,         % underline spaces within strings
  showtabs=false,                 % show tabs within strings adding particular underscores
  morekeywords={var, lazy},
  %  frame=single,                   % adds a frame around the code
  tabsize=2,                  % sets default tabsize to 2 spaces
  captionpos=none,                   % sets the caption-position to bottom
  breaklines=true,                % sets automatic line breaking
  breakatwhitespace=false,        % sets if automatic breaks should only happen at whitespace
  title=\lstname,                 % show the filename of files included with \lstinputlisting; also try caption instead of title
  escapeinside={(**}{**)},          % if you want to add a comment within your code
  keywordstyle=\ttfamily\bfseries,
}

\makeatletter\if@ACM@journal\makeatother
%% Journal information (used by PACMPL format)
%% Supplied to authors by publisher for camera-ready submission
\acmJournal{PACMPL}
\acmVolume{1}
\acmNumber{1}
\acmArticle{1}
\acmYear{2017}
\acmMonth{1}
\acmDOI{10.1145/nnnnnnn.nnnnnnn}
\startPage{1}
\else\makeatother
%% Conference information (used by SIGPLAN proceedings format)
%% Supplied to authors by publisher for camera-ready submission
\acmConference[PL'17]{ACM SIGPLAN Conference on Programming Languages}{January 01--03, 2017}{New York, NY, USA}
\acmYear{2017}
\acmISBN{978-x-xxxx-xxxx-x/YY/MM}
\acmDOI{10.1145/nnnnnnn.nnnnnnn}
\startPage{1}
\fi


%% Copyright information
%% Supplied to authors (based on authors' rights management selection;
%% see authors.acm.org) by publisher for camera-ready submission
\setcopyright{none}             %% For review submission
%\setcopyright{acmcopyright}
%\setcopyright{acmlicensed}
%\setcopyright{rightsretained}
%\copyrightyear{2017}           %% If different from \acmYear


%% Bibliography style
\bibliographystyle{ACM-Reference-Format}
%% Citation style
%% Note: author/year citations are required for papers published as an
%% issue of PACMPL.
%\citestyle{acmauthoryear}  %% For author/year citations
%\citestyle{acmnumeric}     %% For numeric citations
%\setcitestyle{nosort}      %% With 'acmnumeric', to disable automatic
                            %% sorting of references within a single citation;
                            %% e.g., \cite{Smith99,Carpenter05,Baker12}
                            %% rendered as [14,5,2] rather than [2,5,14].
%\setcitesyle{nocompress}   %% With 'acmnumeric', to disable automatic
                            %% compression of sequential references within a
                            %% single citation;
                            %% e.g., \cite{Baker12,Baker14,Baker16}
                            %% rendered as [2,3,4] rather than [2-4].

\lstdefinelanguage{PlainCode}{}
\newcommand\inlinecode[1]{\lstinline[language=PlainCode]{#1}}

\newcommand{\mynote}[3]{{\color{#2} {#1}: #3}}
\newcommand\bruno[1]{\mynote{bruno}{red}{#1}}
\newcommand\huang[1]{\mynote{huang}{blue}{#1}}
\newcommand\haoyuan[1]{\mynote{haoyuan}{cyan}{#1}}

\begin{document}

%% Title information
\title{Type-Safe Modular Parsing}         %% [Short Title] is optional;
                                        %% when present, will be used in
                                        %% header instead of Full Title.
%%\titlenote{with title note}             %% \titlenote is optional;
                                        %% can be repeated if necessary;
                                        %% contents suppressed with 'anonymous'
%%\subtitle{Subtitle}                     %% \subtitle is optional
%%\subtitlenote{with subtitle note}       %% \subtitlenote is optional;
                                        %% can be repeated if necessary;
                                        %% contents suppressed with 'anonymous'


%% Author information
%% Contents and number of authors suppressed with 'anonymous'.
%% Each author should be introduced by \author, followed by
%% \authornote (optional), \orcid (optional), \affiliation, and
%% \email.
%% An author may have multiple affiliations and/or emails; repeat the
%% appropriate command.
%% Many elements are not rendered, but should be provided for metadata
%% extraction tools.

%% Author with single affiliation.
\author{First1 Last1}
\authornote{with author1 note}          %% \authornote is optional;
                                        %% can be repeated if necessary
\orcid{nnnn-nnnn-nnnn-nnnn}             %% \orcid is optional
\affiliation{
  \position{Position1}
  \department{Department1}              %% \department is recommended
  \institution{Institution1}            %% \institution is required
  \streetaddress{Street1 Address1}
  \city{City1}
  \state{State1}
  \postcode{Post-Code1}
  \country{Country1}
}
\email{first1.last1@inst1.edu}          %% \email is recommended

%% Author with two affiliations and emails.
\author{First2 Last2}
\authornote{with author2 note}          %% \authornote is optional;
                                        %% can be repeated if necessary
\orcid{nnnn-nnnn-nnnn-nnnn}             %% \orcid is optional
\affiliation{
  \position{Position2a}
  \department{Department2a}             %% \department is recommended
  \institution{Institution2a}           %% \institution is required
  \streetaddress{Street2a Address2a}
  \city{City2a}
  \state{State2a}
  \postcode{Post-Code2a}
  \country{Country2a}
}
\email{first2.last2@inst2a.com}         %% \email is recommended
\affiliation{
  \position{Position2b}
  \department{Department2b}             %% \department is recommended
  \institution{Institution2b}           %% \institution is required
  \streetaddress{Street3b Address2b}
  \city{City2b}
  \state{State2b}
  \postcode{Post-Code2b}
  \country{Country2b}
}
\email{first2.last2@inst2b.org}         %% \email is recommended


%% Paper note
%% The \thanks command may be used to create a "paper note" ---
%% similar to a title note or an author note, but not explicitly
%% associated with a particular element.  It will appear immediately
%% above the permission/copyright statement.
\thanks{with paper note}                %% \thanks is optional
                                        %% can be repeated if necesary
                                        %% contents suppressed with 'anonymous'


%% Abstract
%% Note: \begin{abstract}...\end{abstract} environment must come
%% before \maketitle command
\begin{abstract}
  Over the years a lot of effort has been put on solving
  extensibility problems, while retaining important software
  engineering properties such as modular type-safety and
  separate compilation. Most previous work focused on
  operations that traverse and process extensible Abstract Syntax Tree
  (AST) structures. However, there is almost no work on operations that
  \emph{build} such extensible ASTs, including \emph{parsing}.

  This paper investigates solutions for the problem of \emph{modular
    parsing}. We focus on \emph{semantic} modularity and not just
  \emph{syntactic} modularity. That is, the solutions should not only
  allow complete parsers to be built out of modular parsing
  components, but also enable the parsing components to be
  \emph{modularly type-checked} and \emph{separately compiled}. We
  present a technique based on parser combinators that enables modular
  parsing. Interestingly, the modularity requirements for modular
  parsing rule out several existing parser combinator approaches,
  which rely on some non-modular techniques. We show that Packrat
  parsing techniques, provide solutions for such modularity problems,
  and enable reasonable performance in a modular setting.
  Extensibility is achieved using \emph{multiple inheritance} and
  \emph{Object Algebras}.  To evaluate
  the approach we conduct a case study based on the ``Types and
  Programming Languages'' interpreters. The case study shows the
  effectiveness at reusing parsing code from existing interpreters,
  and the total parsing code is 69\% shorter than an existing code
  base using a non-modular parsing approach.
\end{abstract}


%% 2012 ACM Computing Classification System (CSS) concepts
%% Generate at 'http://dl.acm.org/ccs/ccs.cfm'.
\begin{CCSXML}
<ccs2012>
<concept>
<concept_id>10011007.10011006.10011008</concept_id>
<concept_desc>Software and its engineering~General programming languages</concept_desc>
<concept_significance>500</concept_significance>
</concept>
<concept>
<concept_id>10003456.10003457.10003521.10003525</concept_id>
<concept_desc>Social and professional topics~History of programming languages</concept_desc>
<concept_significance>300</concept_significance>
</concept>
</ccs2012>
\end{CCSXML}

\ccsdesc[500]{Software and its engineering~General programming languages}
\ccsdesc[300]{Social and professional topics~History of programming languages}
%% End of generated code


%% Keywords
%% comma separated list
\keywords{keyword1, keyword2, keyword3}  %% \keywords is optional


%% \maketitle
%% Note: \maketitle command must come after title commands, author
%% commands, abstract environment, Computing Classification System
%% environment and commands, and keywords command.
\maketitle


\section{Introduction}\label{sec:introduction}
 
The quest for improved modularity, variability and extensibility of
programs has been going on since the early days of Software
Engineering~\cite{}. Modern Programming Languages enable a certain
degree of modularity, but they have limitations as illustrated by
well-known problems such as the Expression Problem~\cite{}. The
Expression Problem refers to the difficulty of writing data
abstractions that can be easily extended with both new operations and
new data variants. Traditionally the kinds of data abstraction found
in functional languages can be extended with new operations, but
adding new data variants is difficult. The traditional object-oriented
approach to data abstraction facilitates adding new data variants
(classes), while adding new operations is more difficult.

A reason why a solution to the Expression Problem is important in
practice is that it is necessary for the development of
\emph{Software-Product Lines} (SPLs)~\cite{}. A software-product line
is a reusable set of components, which can be combined in multiple ways
to obtain different programs. Programming languages offer a concrete
example for SPLs. A SPL for programming languages would allow us to
model various typical operations of programming languages (such as
evaluation, compilation, or parsing) for various different language
constructs (such as binding, arithmetic, conditional or loops)
independently and separately. For example, evaluation components could be defined
independently for binding and arithmetic constructs. If the language
to be implemented is the pure lambda calculus, only evaluation of
binding constructs is necessary. However, more realistic programming
languages will include arithmetic constructs, and will require 
evaluation for such constructs as well. In this case 
both the component for evaluation of binders and arithmetic 
expressions can be combined to implement the desired functionality.

\begin{comment}
Most programming languages share alot of features in
common. 

For example, most languages have language constructs for:
binding (such as variables, functions, and function applications);
basic arithmetic operations; basic logic and conditional operations;
loops; as well as various other features. For each language construct,
various operations (such as evaluation, compilation, or parsing) need
to be implemented. It is reasonable to wonder whether we can simply
implement those features independently of a particular implementation
of a programming language. Evaluation could be defined independently 
for binding and arithmetic constructs. If the language to be
implemented is the pure lambda calculus, only evaluation of binding 
constructs is necessary. Thus only the component that implements 
evaluation for binding needs to be used in such an implementation.
However, more realistic programming languages 
will include arithmetic constructs, and will require an evaluation
function for those. 


Then it would be possible to \emph{reuse}
some of those features in \emph{multiple} different implementations of
programming languages. Essentially, this would enable a SPL for
programming languages, where all

A solution to the Expression Problem could ena



A concrete 
example that illustrates this issue is 
\end{comment}

To address the modularity limitations of Programming Languages several
different approaches have been proposed in the past. Existing
approaches can be broadly divided into two categories:
\emph{syntactic} or \emph{semantic} modularization
techniques. Syntactic modularization techniques are quite popular in
practice. Examples include many tools for developing Software-Product
Lines~\cite{}, some Language Workbenches~\cite{}, or extensible parser
generators~\cite{}.  Most syntactic approaches employ textual
composition techniques such as \emph{superimposition}~\cite{} to
enable the development modular program features. Such textual
composition techniques collect the code for multiple features and
merge it together when a concrete combination of features is needed
for a particular program. As Kastner~\cite{} notes, the typical
drawback of such techniques is that
``\emph{most feature-oriented implementation mechanisms lack proper
  interfaces and support neither modular type checking nor separate
  compilation}''\bruno{reference to ``The road to
  Feature Modularity''}. Syntactic modularization techniques have also
been applied to the problem of \emph{extensible parsing}. There are
several approaches~\cite{} that enable the development of
\emph{syntactically} modular parsers or grammars. However these
approaches do not support separate compilation or modular
type-checking either.\bruno{mention that a reason why syntactic 
modularization techniques are popular is simplicity (in implementation
and also use).}

Semantic modularization techniques move away
from syntactic composition techniques that rely on the textual source
code. This allows us to go one step further in terms of modularity,
and also enable components or features to be modularly type-checked
and separately compiled. Modular type-checking and separate
compilation are desirable properties to have from a Software
Engineering point-of-view, and enable the composition of compiled
binaries as well as ensuring the type-safety of the code composed of
multiple components. Examples of semantic modularization techniques 
include various approaches to \emph{family polymorphism}~\cite{}, as 
well as several techniques for solving the Expression Problem. 
\bruno{strenghten? mention dynamic languages (separate compilation 
but no modular type-checking), and techniques that weaken type-safety 
requirements.} Most semantic
modularization techniques have focused on operations that traverse or 
or process extensible datastructures, such as ASTs.
\bruno{Object Algebras here?} However, as far as
we know there is little work on operations that build/produce ASTs. 
In particular the problem of modular parsing has not been studied in
semantic modularization approaches. This is a shame because, to
realise the vision of software product-lines for programming
languages, modular parsing is a necessity. 

\paragraph{Modular and Extensible Parsing}
  This paper investigates presents \name: a parsing
  combinator library that enables modular parsing.
  \name provides a solution for the problem of \emph{semantic modular
    parsing}. That is, the solutions should not only
  allow complete parsers to be built out of modular parsing
  components, but also enable the parsing components to be \emph{modularly
  type-checked} and \emph{separately compiled}. \name is
  built on top of a Packrat parsing library, but adds new parsing
  combinators to enable modular parsing. The new parsing combinators 
  employ \emph{delegation-based} techniques and \emph{Object Algebras} 
  to support extensibility. The choice of Packrat parsing over other
  parsing techniques turns out to be important for achieving
  performance in a modular setting. \bruno{left recursion, and
    backtracking removal.} 

  By analising the \emph{full} grammar it is possible to remove
  backtracking, which would otherwise increase parsing times. Many
  parsing combinator libraries routinely use backtracting elimination
  to achieve performance. However, in a modular setting this technique
  cannot be used, because the full grammar is not known. Thus we have
  to be very conservative at eliminating backtracting. Unfortunatelly,
  this has a severe impact on performance.

  To evaluate \name we conduct a case study based on the ``Types and
  Programming Languages'' (TAPL) interpreters.  The case study shows
  that \name is effective at reusing parsing code from existing
  interpreters, and the total parsing code is 60\% shorter than the
  non-modular parsing code.\bruno{comment on the efficiency}

In summary our contributions are:

\begin{itemize}

\item {\name:} A parser combinator library that allows the development 
of modular parser. The library uses \emph{delegation} and
\emph{Object-Algebras} to achieve modularity and extensibility.

\item {{\bf A Parsing Technique for OO ASTs:}} A simplified version of
  our technique also enables parsing OO-style ASTs, where new language
  constructs can be easily added.

\item {{\bf Limitations of existing Parser Combinator Techniques:}}
\bruno{Improve and write something here.}

\item {{\bf TAPL case study:}} We conduct a case study with 18 interpreters
  from the TAPL book. The case study show the effectiveness of modular 
  parsing in terms of reuse.

\end{itemize}


\haoyuan{I suggest this part to be moved to Introduction, and only discuss why Packrat is selected among
different parser combinators in Overview.}

\begin{comment}
Although there are many parsing techniques, not all of them are
suitable for type-safe modular parsing. In particular there are many
techniques which fail to provide modular type-checking and separate
compilation. Moreover, even if modular type-checking and separate
compilation are supported, efficiency is another
concern. A parsing technique should have low overhead when applied
in a modular setting. In the remaider of this section, we eliminate
various techniques that fail to satisfy our requirements, and argue
that that Packrat parsing~\cite{Ford2002} is a suitable candidate for
type-safe modular parsing.

\paragraph{Parser Generators} The most widely use tools for parsing
are parser generators. Parser generators help users generate parsers automatically or
semi-automatically from a given grammar. There is no restriction on
the algorithm, while most of them adopts table-based LL~\cite{lewis1968syntax} and LR~\cite{knuth1965translation} parsing
algorithms.
Although efficient, the main drawback of parser generators is that they do not support
modular type-checking and separate compilation.

Modular parsing based on parser generators is supported by many libraries~\cite{antlr1995,Grimm2006,Gouseti2014,Warth2016}. Users can separate the syntax definition and related parsing code into reusable components. Then the corresponding parsers are built by their library utilities. For example, NOA~\cite{Gouseti2014} uses Java annotation processing to collect grammar information, and then generates ANTLR4 parsers. However, such generation procedure requires a whole compilation after the collection of all grammar pieces. Once the grammar changed, even slightly, grammar information must be re-collected and the parser must be re-generated. Hence those libraries only have syntactic modularity.

Generating parsers often requires full information of grammars, thus semantic modularity is difficult to achieve in this way.

\paragraph{Parser Combinators}
Comparing with the parser generators, a \textit{parser combinator}~\cite{burge1975,Wadler1985}
takes several parsers and produce a new parser as its output. Parser combinators are
popular in functional programming, where the parsers are represented
by functions and parser combinators are higher-order functions accepts
them.

At a first look, parser combinators are very suitable for our purpose, because of two
reasons. Firstly, they are naturally modular. The manner of using them
is to write small parsers and use combinators to composed them
together. The construction procedure is explicit and fully controlled
by the programmer. Secondly, each parser combinator is represented by
a piece of code, and also are the parsers it takes. Thus in a
statically typed programming language they can be statically
type-checked.
\end{comment}

\section{Packrat Parsing for Modularity}\label{sec:packrat}

This section discusses the algorithmic challenges introduced by modular parsing and argues that parser combinators from Packrat parsing
are suitable to address those challenges.

\subsection{Algorithmic Challenges of Modularity}\label{subsec:challenges}
At a first look, parser combinators are very suitable for modular parsing, because of two reasons. Firstly, they are naturally modular. The manner of using them is to write small parsers and use combinators to composed them together. The construction procedure is explicit and fully controlled by the programmer. Secondly, each parser combinator is represented by a piece of code, and also are the parsers it takes. Thus in a statically typed programming language they can be statically type-checked.
Unfortunately many parser combinators have important limitations.
In particular several parser combinators,
including the famous Parsec~\cite{Leijen2001} library, require
programmers to manually do \textit{left-recursion elimination}, \textit{longest match composition}, and
require significant amounts of \textit{backtracking}. All of them are
problematic in a modular setting.

\paragraph{Left-Recursion Elimination} The top-down, recursive descent parsing strategy adopted by those parser combinator libraries cannot support left-recursive grammars directly. The common solution is to rewrite the grammar into an equivalent but not left-recursive one, so called left-recursion elimination.

The grammar below represents a simple arithmetic language containing only integers. Its corresponding parser is shown on the right side, written in Parsec.

\begin{tabular}{m{0.4\linewidth}m{0.5\linewidth}}
\setlength{\grammarindent}{5em}
\begin{grammar}
<expr> ::= <int>
\end{grammar}
&
\begin{lstlisting}[language=PlainCode]
parseExpr = parseInt
\end{lstlisting}
\end{tabular}

Consider we extended the language by adding subtractions to the grammar. If we still write its parser by directly following the grammar structure, the new left-recursive case will get the parser into an infinite loop. Namely, \inlinecode{parseExpr} and \inlinecode{parseSub} call each other and never stop.

\begin{tabular}{m{0.4\linewidth}m{0.5\linewidth}}
\setlength{\grammarindent}{5em}
\begin{grammar}
<expr> ::= <int> \alt <expr> `-' <int>
\end{grammar}
&
\begin{lstlisting}[language=PlainCode]
parseExpr = parseSub <|> parseInt
parseSub = do
  e <- parseExpr
  ...
\end{lstlisting}
\end{tabular}

To solve this issue, we have to rewrite the grammar as below to eliminate left-recursion and build the parser based on this new grammar.\\

\begin{tabular}{m{0.4\linewidth}m{0.5\linewidth}}
\setlength{\grammarindent}{5em}
\begin{grammar}
<expr> ::= <int> <expr'>

<expr'> ::= <empty> \alt `-' <int> <expr'>
\end{grammar}
&
\end{tabular}

After left-recursion elimination, the structure of grammar is changed,
as well as its corresponding parser. In a modular setting, it is
possible but unnecessarily compilcated to analyse the grammar and
rewrite it when doing extension. Anticipating that every non-terminal
has left-recursive rules is helpful for extensibility but overkill,
since it is inconvenient and introduces extra complexity for
representation of grammar and implementation of parser.

Another issue of left-recursion elimination is that it requires extra
bookkeeping work to retain the original semantics. For example, the
expression $1-2-3$ is parsed as $(1-2)-3$ in the left-recursive
grammar, but after rewrite the result is $1((-2)-3))$. The parse tree
must be transformed to recover its structure.

\paragraph{Longest Match Composition} Another problematic issue
in parser combinator libraries is the need for manually ordering
alternatives in a grammar.
Consider the grammar:
\setlength{\grammarindent}{5em}
\begin{grammar}
<expr> ::= <int> \alt <int> `+' <expr>
\end{grammar}

In Parsec, for instance, the parser below will only parse the input \inlinecode{"1 + 2"} to \inlinecode{"1"}, as \inlinecode{parseInt} successfully parses \inlinecode{"1"}
and terminates parsing.

\begin{lstlisting}[language=PlainCode]
parser = parseInt <|> parseAdd
\end{lstlisting}

Using traditional alternative composition, when a preceding parser successfully parses a prefix of the input, it will finish parsing and return the result, in spite that subsequent parsers may be able to parse the whole input.
%\huang{I've rewritten this paragraph}\bruno{The example is good, I think but the explanation is not.
%You want to say that when you know
%  the full grammar, you can figure out where ``try'' is
%  needed. Without the full grammar you'd need assume the worst
%  case. Make an effort to make your explanation cristal clear!
%I think you want to miss the abstract explanation that you give first,
%with the explanation about the concrete example.
%Start with ``The need for backtracking is also problematic
%in a modular setting. For example, suppose ...'' and synchronize
%the abstract explanation and the explanation for the example.
%}
In contrast with the parser above, \inlinecode{parser = parseAdd <|> parseInt} works as expected with the two cases swapped.

In this case, reordering the alternatives ensures that
the \emph{longest match} is picked among the possible results. However, manual reording for the longest match is inconvenient, and worst still, it is essentially non-modular. When the grammar is extended with new rules, programmers are supposed to \emph{manually} adjust
the order of parsers, which requires rewritting previously written code.

\paragraph{Backtracking} The need for backtracking is also problematic
in a modular setting. For example, consider a grammar that includes
``import'' statements including \inlinecode{import..from}.
Now we want to extend the grammar with an \inlinecode{import..as} case, as shown in the third line of grammar below.

\setlength{\grammarindent}{5em}
\begin{grammar}
<stmt> ::= `import' <ident> `from' <ident>
    \alt ...
    \alt `import' <ident> `as' <ident>
\end{grammar}

Since the \inlinecode{import..from} case shares a prefix with the new case \inlinecode{import..as}, when the former case fails, we must backtrack to the beginning. Take Parsec as an example, its choice
combinator \inlinecode{<|>} only tries the second alternative if the first fails
without any token consumption. An auxiliary function \inlinecode{try} is used for explicit backtracking.

\begin{lstlisting}[language=PlainCode]
oldParser = parseImpFrom <|> parseA <|> parseB <|> ...
newParser = try parseImpFrom <|> parseA <|> parseB <|> ... <|> parseImpAs
\end{lstlisting}

Given the full grammar, we can decide where to put \inlinecode{try} for backtracking. However, with modular parsing we are unable to have a global view of the full grammar. Hence the worst case should be considered that all alternatives may share common prefixes with future cases. In that case we need to backtrack for all the branches. To avoid failures in the future, we have to add \inlinecode{try} everywhere:

\begin{lstlisting}[language=PlainCode]
parser = try parseImpFrom <|> try parseA <|> try parseB <|> ... <|> try parseImpAs
\end{lstlisting}

\noindent However, this results in the worst case exponential time complexity in Parsec, because it does not have related optimization.

\subsection{Packrat Parsing}\label{subsec:packratparsing}
Fortunately some more advanced parsing techniques such as Packrat
parsing~\cite{Ford2002} have been developed to address limitations of
simple parser combinators. Packrat parsers use memoization to record
the result of applying each parser at each position of the input, so
that repeated computation is eliminated. Moreover, Packrat parsing, in
theory, supports both direct and indirect left-recursion~\cite{warth2008}. All of these properties are very suitable for modularity, thus we decided to use Packrat parsers as the underlying parsing technique in for modular parsing.
\begin{comment}
It is worth mentioning that the choice of parser combinators will not
affect the other parts of our library. One can choose other parser
combinators like Parsec, in cases that the performance and supporting
of left-recursion are not major concerns. A different library can even build a new
\name with fancy features or higher efficiency.
\end{comment}
Scala has a standard parser combinator library\footnote{https://github.com/scala/scala-parser-combinators}
\cite{moors2008parser} which implements Packrat parsers.
The library provides a number of parser combinators, including the longeset match alternative combinator.
Below we present an example to illustrate Scala parsers.
%
%\bruno{So, we make a big fuss identifying the problems with other
%parser combinators in the previous subsection and now we present
%an example that does not suffer from any of the previous problems!
%What makes sense here is an example that uses: a \emph{left-recursive grammar}; the \emph{longest match combinator}; and where we can argue that
%there's no need for the user to do manual backtracking. Just use
%the example with the left-recursive grammar for parsing integers and additions and argue that none of the limitations
%discussed previously applies!
%}

\paragraph{Parsing a Simple Arithmetic Language}
Suppose we want to parse a simple language with literals and
additions. The concrete syntax is as follows:
%\bruno{Here's a good opportunity to both improve the paper and make it
%more compact! This is the example that should be illustrated and
%discussed at the end of Section 2. Then we can just create a figure to
%refer to the code in Section 2 and refer again to the figure here. No
%need for two different examples.}

\setlength{\grammarindent}{5em}
\begin{grammar}
<expr> ::= <int>
    \alt <expr> `+' <expr>
\end{grammar}
%\alt `(' <expr> `)'

It is straightforward to model ASTs by inheritance and write corresponding parsers for all cases.
Note that a parser has type \lstinline{PackratParser[E]} for some
\lstinline{E}, which indicates the type of results it produces. Figure~\ref{fig:packrat-arith} shows the abstract syntax and
the parser code that generates expressions as ASTs. The AST supports
pretty-printing via the \lstinline{print} method.

\begin{figure}[t]
\centering
\lstinputlisting[linerange=6-29]{../Scala/Parser/src/PaperCode/Sec2Packrat/Code1.scala}% APPLY:linerange=PACKRAT_SIMPLE_EXPR
\caption{Packrat Parsing for an arithmetic language.}\label{fig:packrat-arith}
\end{figure}


In the trait \lstinline{ExprParser}, \lstinline{lexical} is used for lexing. \lstinline{pLit} parses an integer for the literal case.
\lstinline{pAdd} handles the addition case and creates an object of \lstinline{Add}. It parses two sub-expressions by calling \lstinline{pExpr}
recursively. Finally \lstinline{pExpr} composes \lstinline{pLit} and \lstinline{pAdd} using the longest match alternative combinator \inlinecode{|||}.
Table~\ref{tab:packrat} gives a closer look at part of the API that we have used in our work.

\begin{table}[t]
\begin{tabular}{l}
\hline
\begin{lstlisting}
def ~[U](q: => Parser[U]): Parser[~[T, U]]
\end{lstlisting} \\
\hspace{.2in}- A parser combinator for sequential composition. \\
\hline
\begin{lstlisting}
def ^^[U](f: (T) => U): Parser[U]
\end{lstlisting} \\
\hspace{.2in}- A parser combinator for function application. \\
\hline
\begin{lstlisting}
def ^^^[U](v: => U): Parser[U]
\end{lstlisting} \\
\hspace{.2in}- A parser combinator that changes a successful result into the specified value. \\
\hline
\begin{lstlisting}
def <~[U](q: => Parser[U]): Parser[T]
\end{lstlisting} \\
\hspace{.2in}- A parser combinator for sequential composition which keeps only the left result. \\
\hline
\begin{lstlisting}
def ~>[U](q: => Parser[U]): Parser[U]
\end{lstlisting} \\
\hspace{.2in}- A parser combinator for sequential composition which keeps only the right result. \\
\hline
\begin{lstlisting}
def repsep[T](p: => Parser[T], q: => Parser[Any]): Parser[List[T]]
\end{lstlisting} \\
\hspace{.2in}- A parser generator for interleaved repetitions. \\
\hline
\begin{lstlisting}
def ident: Parser[String]
\end{lstlisting} \\
\hspace{.2in}- A parser which matches an identifier. \\
\hline
\begin{lstlisting}
def numericLit: Parser[String]
\end{lstlisting} \\
\hspace{.2in}- A parser which matches a numeric literal. \\
\hline
\begin{lstlisting}
def |[U >: T](q: => Parser[U]): Parser[U]
\end{lstlisting} \\
\hspace{.2in}- A parser combinator for alternative composition. \\
\hline
\begin{lstlisting}
def |||[U >: T](q0: => Parser[U]): Parser[U]
\end{lstlisting} \\
\hspace{.2in}- A parser combinator for alternative with longest match composition. \\
\hline \\
\end{tabular}
\caption{Part of Scala Parser API.\bruno{We should double-check that this table lists the
  combinators used in the paper.}}\label{tab:packrat}
\end{table}

It is worth mentioning that the grammar is left-recursive in the addition case. Since left-recursion is well supported by Packrat parsers, we do not need extra code for it. Our parser also employs longest match composition by using the combinator \lstinline{|||} in the library. Furthermore, it does not suffer from the backtracking problem, as the memoization technique of Packrat parsing guarantees the efficiency.

For more concise demonstration, in the rest of paper we assume that all the code are in an enclosing environment of traits \inlinecode{StandardTokenParsers} and \inlinecode{PackratParsers}. Futhermore, we will use \lstinline{Parser} as a type synonym for \lstinline{PackratParser} and a generic \inlinecode{parse} function for testing as below. The last line demonstrates how we parse a valid expression \inlinecode{1 + 2} using our parser.

\lstinputlisting[linerange=34-41]{../Scala/Parser/src/PaperCode/Sec2Packrat/Code2.scala}% APPLY:linerange=PACKRAT_RUNPARSER


\section{Modularizing Parsers with Inheritance}\label{sec:inheritance}

This section introduces the problem that motivates our work. In Figure~\ref{fig:packrat-arith} we have observed
that Packrat parsers can be defined in a convenient way, with recursive calls. The example is actually inspired
by Walder's Expression Problem. Similarly, our goal is to build parsers that are extensible in some dimensions.

\subsection{Dimensions of Extensibility}
We continue with the expression language of literals and additions. At this point,
one would like to have two kinds of extensions:

%\section{An Overview of \name}\label{sec:overview}
%
%This section gives an overview of our library \name, motivates
%additional problems that occur when parsing modular ASTs.  The first
%problem is that traditional parsers using parser combinators use
%hard-coded recursive calls. This problem is solved using
%delegation-based techniques, which enable parsing OO ASTs.  The second
%problem is how to achieve dual extensibility of ASTs (that is: how to
%allow for both new language constructs and new operations over the
%ASTs). This problem is solved using Object Algebras.

\begin{comment}
Basically, \name consists of four parts: underlying parsing technique, delegation mechanism encoded by open recursion, Object Algebras, and glue code of new combinators and utility functions. We start from Section \ref{subsec:overview-parsing}, which discusses the choice of parsing technique and how it affects modularity of parsers. Section \ref{subsec:overview-problem} demonstrates the goal of extending parsers together with ASTs in a semantic modular way, with both separate compilation and type-safe code reuse. Then we will see traditional parser combinators fail to achieve it because of hard-coded recursive calls. In Section \ref{subsec:overview-delegation}, we show how delegation can solve this problem and allow us to build extensible parsers. Finally, Section \ref{subsec:overview-oa} gives examples of using Object Algebras for more extensibility, including extension of operations and parsing multiple sorts of syntax.\haoyuan{TODO}
\end{comment}

\begin{comment}
\subsection{Choosing the Parsing Technique}\label{subsec:overview-parsing}

%A technique for type-safe modular parsing should the following 3
%features: \emph{modular type-checking}; \emph{separate compilation};
%low performance overhead

In the last section, we have argued that parser combinators are a suitable parsing technique for
our purpose, as they naturally build modular parsers for type-checking.
Unfortunately many parser combinators have important limitations.
In particular several parser combinators,
including the famous Parsec~\cite{Leijen2001} library, require
programmers to manually do \textit{left-recursion elimination} and \textit{longest match composition}, and
require significant amounts of \textit{backtracking}. All of them are
problematic in a modular setting.




\paragraph{Packrat Parsing}
Fortunately some more advanced parsing techniques such as Packrat
parsing~\cite{Ford2002}, address the limitations of simple parser combinators
such as Parsec. Packrat parsers use
memoization to record the result of applying each parser at each
position of the input, so that repeated computation is eliminated.
Moreover, theoretically the algorithm behind Packrat parsers
supports both direct and indirect left-recursion~\cite{warth2008}.
The current version of the library is still
buggy with indirect left-recursion, but we believe that
they will fix it in the future, and direct left-recursion is
already practical to use for a large number of applications. All of these properties are very
suitable for modularity, thus we decided to use Packrat parsers as the underlying
parsing technique in \name.


It is worth mentioning that the choice of parser combinators will not
affect the other parts of our library. One can choose other parser
combinators like Parsec, in cases that the performance and supporting
of left-recursion are not major concerns. A different library can even build a new
\name with fancy features or higher efficiency.
\end{comment}

%\huang{done, added the attempt/failure/reason of extending conventional parsers}\bruno{You are not motivating the problem! You are going straight to
%  the solution without pointing out what the problem is first. What
%  you need to do is: First show what happens with conventional
%  parsers: at some point, if you add extensions the recursive calls
%  will be wrong. Then you show (in the next section) the solution:
% use delegation/open recursion.}

%\huang{partly done, only extend a new language construct in the example, how about leaving extension of new operations in the OA subsection?}\bruno{I think we need to set up a challenge here, similar to the challenge of
%  the expression problem. The challenge should be like. Build a parser
%for a simple expression language, then extend the language with both
%a new language construct and a new operation. This section will show
%how we can do that using traditional parsers, but without modular
%type-safety and separate compilation. The remaining sections will show
%that the techniques introduced by us, enable us to solve those two challenges.}

\begin{itemize}
\item \textbf{Extension 1}: adding a new case (or rather, a new construct) to \lstinline{Expr};
\item \textbf{Extension 2}: adding a new operation to \lstinline{Expr}.
\end{itemize}

\paragraph{Attempt to Extend the Parser} Now we first consider
extension 1, namely we are extending the syntax as well as the parser. Hence we introduce variables as a new case.

\setlength{\grammarindent}{5em}
\begin{grammar}
<expr> ::= ...
   \alt <ident>
\end{grammar}

As in an OO solution to the Expression Problem, it is easy to extend
the corresponding AST:

\lstinputlisting[linerange=95-97]{../Scala/Parser/src/PaperCode/Overview/Overview.scala}% APPLY:linerange=OVERVIEW_SIMPLE_LAM

And since we already have the parser for literals and additions, we would
like to build the new parser by reusing the old one. Here one may quickly come up
with the following attempt, where a new parser is defined for \lstinline{Var}, then composed
with \lstinline{pExpr}:

\lstinputlisting[linerange=62-63]{../Scala/Parser/src/PaperCode/Overview/Overview.scala}% APPLY:linerange=BAD_ATTEMPT

Unfortunately, it fails to parse some expressions like \lstinline{"1 + x"}, which is obviously valid in the new grammar.
The reason is that \lstinline{pAdd} makes two recursive calls to parse sub-expressions, by using \lstinline{pExpr}, which
covers both cases in the old grammar. But the newly added case \lstinline{pVar} is not observed by the recursive \lstinline{pExpr},
hence the parser does not work as expected. Still, one may continue to change the recursive call in \lstinline{pAdd} with \lstinline{pExtExpr},
but modification on existing code sacrifices separate compilation, as mentioned by the Expression Problem.

\subsection{Inheritance-Based Approach}\label{subsec:inheritance-approach}

Fortunately in Scala we can make use of inheritance for modular parsing. Specifically,
the new parser can be defined in an enclosing trait that extends
the old one:

\lstinputlisting[linerange=68-73]{../Scala/Parser/src/PaperCode/Overview/Overview.scala}% APPLY:linerange=INHERITANCE_APPROACH
Now \lstinline{ExtParser} successfully represents the parser for the extended language, because Scala uses dynamic dispatch for
method overriding in inheritance. When the input \lstinline{"1 + x"} is fed to the parser \lstinline{this.pExpr}, it firstly delegates
the work to \lstinline{super.pExpr}, which parses literals and additions. However, the recursive call \lstinline{pExpr} in \lstinline{pAdd}
actually refers to \lstinline{this.pExpr} again, due to dynamic dispatch. It implies that the recursive call has included the extension. This
elegant inheritance-based approach opens the recursion of parsers in a modular way.

Notice that Scala has a linearized-style multiple inheritance for traits. It can be very helpful when composing several languages. Suppose now
we want to compose the parsers for expressions from pre-defined languages \lstinline{LanguageA} and \lstinline{LanguageB} using alternative.
The use of keyword \lstinline{super} is able to
specify the implementation from inheritance, like:

\lstinputlisting[linerange=78-84]{../Scala/Parser/src/PaperCode/Overview/Overview.scala}% APPLY:linerange=MULTIPLE_INHERITANCE

As we demonstrated, inheritance is the key technique to obtain semantic modularity.
It enables type-safe code reuse and separate compilation for parsing OO style ASTs with extension 1.

\subsection{Object Algebras for full Extensibility}\label{subsec:overview-oa}

Although delegation enables OO extensibility, the use of an OO class
hierarchy makes the addition of new operations over the AST
problematic. Modification on existing code breaks the modularity. This
refers to the famous \textit{Expression Problem} []. For challenge 2, suppose now we want to
support evaluation on expressions, in Scala one attempt would be extending \lstinline{Expr} with
evaluation:

\lstinputlisting[linerange=178-178]{../Scala/Parser/src/PaperCode/Overview/Overview.scala}% APPLY:linerange=OVERVIEW_ATTEMPT_EXPRWITHEVAL
But then the programmer has to define new classes for literals and additions that extend old ones and \lstinline{EvalExpr}. Specifically,
the critical point is that existing parsing code only generates \lstinline{Expr} objects with pretty printing, so modification on parsing is again required.


In contrast, Object Algebras~\cite{Oliveira2012} enable us to solve this problem, as it separates data variants and
operations, to
offer high flexibility in the choice of operations to be performed
over the AST. It also makes parsing with multiple sorts of syntax easier.

\paragraph{Parsing with Object Algebras} Using Object Algebras, the abstract syntax of the expression language is defined as below.

\lstinputlisting[linerange=8-11]{../Scala/Parser/src/PaperCode/Overview/OA.scala}% APPLY:linerange=OVERVIEW_OA_ALG

Then we are able to define operations over the syntax in a modular way. For instance, pretty printing operation can be realized as:

\lstinputlisting[linerange=15-18]{../Scala/Parser/src/PaperCode/Overview/OA.scala}% APPLY:linerange=OVERVIEW_OA_PRINT

And also parser for it as below. Notice the parsing function \inlinecode{pExpr} now takes an argument of type \inlinecode{ExprAlg[E]}, which means it accepts any instance of \inlinecode{ExprAlg}. Such an instance can for example be \lstinline{Print}, which takes charge of operations on expressions, and parsing code only does delegation.

\lstinputlisting[linerange=22-27]{../Scala/Parser/src/PaperCode/Overview/OA.scala}% APPLY:linerange=OVERVIEW_OA_PARSER

\haoyuan{I'm trying to avoid using terminology like algebra and algebra interface.}

\paragraph{Extensibility of Syntax} Following previous examples, the code below shows how to extend the language with variables.

\lstinputlisting[linerange=31-44]{../Scala/Parser/src/PaperCode/Overview/OA.scala}% APPLY:linerange=OVERVIEW_OA_EXT
Note that we use the new combinator \lstinline{<|>} defined in our library, to compose two parsing functions into one. The function \lstinline{pExtExpr} has the compound type \lstinline{ExprAlg[E] with VarAlg[E]} in its argument, where two language ASTs are combined to represent the new language.

To use the parser, we must provide an algebra instance as the operation to construct the parsing results. In the code below, we use the combination of \lstinline{Print} and \lstinline{PrintVar}, so that the parsing result is a pretty printing of the AST.

\lstinputlisting[linerange=53-58]{../Scala/Parser/src/PaperCode/Overview/OA.scala}% APPLY:linerange=OVERVIEW_OA_USE

\paragraph{Extensibility of Operations} With Object Algebras, operations over ASTs can also be extended in a modular way. Here is an example of collecting free variables from an expression. We can feed this operation to the parser and obtain a set of free variables.

\lstinputlisting[linerange=72-78]{../Scala/Parser/src/PaperCode/Overview/OA.scala}% APPLY:linerange=OVERVIEW_OA_EXT_OP

\haoyuan{I think in our library language ASTs are composed using intersection, not inheritance.}

\paragraph{Multiple Sorts of Syntax} Another advantage of using Object Algebras is that it supports multiple sorts of syntax easily. In several cases, we want to divide the syntactic elements into some groups. For example, the grammar below has two sorts, which are expressions and types.

\begin{tabular}{m{0.45\linewidth}m{0.45\linewidth}}
\setlength{\grammarindent}{5em}
\begin{grammar}
<type> ::= `int' \alt <type> `->' <type>
\end{grammar}
&
\setlength{\grammarindent}{5em}
\begin{grammar}
<expr> ::=  `\\' <ident> `:' <type> `.' <expr>
\end{grammar}
\end{tabular}

Using Object Algebras, we can easily distinguish them just by adding an extra type parameter. The abstract syntax is
presented below:

\lstinputlisting[linerange=63-67]{../Scala/Parser/src/PaperCode/Overview/OA.scala}% APPLY:linerange=OVERVIEW_OA_MULTI_SYNTAX
In the next sections, we will discuss them with more details.


\section{Object Algebras and Parsing}\label{sec:algebrasandparsing}

Object Algebras, first introduced by ... et al as a solution to the famous \textit{Expression Problem}, provide extensibility on
both data variants and operations for structures like abstract syntax trees. We integrate Object Algebras directly to enhance the
extensibility of open parsers.

\subsection{Object Algebras}\label{subsec:objectalgebras}
Object Algebras captures a design pattern to address the Expression Problem nicely,
achieving two dimensions of extensibility (data variants and operations) in a modular and type-safe way.
Because of this, the definition of data structures is separated from behaviors on them, and future extensions
to both sides no longer require existing code to be modified, supporting separate compilation.

In Object Algebras, ASTs as recursive data structures are defined using traits, where each constructor corresponds
to an abstract method inside. The example from Section~\ref{subsec:parsingwithopen} is again used here for illustration.
At first the language only supports variables and applications:
\begin{lstlisting}
trait ExprAlg[E] {
    def varE(x : String) : E
    def appE(e1 : E, e2 : E) : E
}
\end{lstlisting}
The language has two constructors: literals and additions. The trait \lstinline{ExprAlg} is called an \textit{Object Algebra interface},
whereas as a factory, it cannot produces objects for the expressions directly like traditional approaches, but abstract the results in its
type parameter \lstinline{E} instead. To realize an operation on expressions, we simply instantiate the type parameter by a concrete type and
provides implementations for all cases. Below is an example of collecting all free variables in an expression:
\begin{lstlisting}
trait FreeVars extends ExprAlg[List[String]] {
    def varE(x : String) = List(x)
    def appE(e1 : List[String], e2 : List[String]) = e1 ++ e2
}
\end{lstlisting}
Here \lstinline{FreeVars} is called an \textit{Object Algebra}. It traverses an expression bottom-up, and returns a list of strings as the result.
Furthermore, one can define a new trait to implement pretty-printing on expressions:
\begin{lstlisting}
trait ExprPrint extends ExprAlg[String] {
    def varE(x : String) = x
    def appE(e1 : String, e2 : String) = "(" + e1 + " " + e2 + ")"
}
\end{lstlisting}
On the other hand, the data variants can be extended by inheriting \lstinline{ExprAlg} and adding new cases only. Suppose we want to
have literals in expressions, a new Object Algebra interface \lstinline{LitAlg} is defined as follows:
\begin{lstlisting}
trait LitAlg[E] extends ExprAlg[E] {
    def litE(n : Int) : E
}
\end{lstlisting}
Now pretty-printing on the new language can be realized by code reuse, without modifying existing code:
\begin{lstlisting}
trait LitPrint extends ExprPrint {
    def litE(n : Int) = n.toString
}
\end{lstlisting}
On this, to create an expression of \lstinline{LitAlg}, a generic method is defined as follows:
\begin{lstlisting}
def build[E](alg : LitAlg[E]) : E =
    alg.appE(alg.varE("x"), alg.litE(3))
\end{lstlisting}
Such a method implicitly represents the expression \lstinline{"x 3"}. The code
\begin{lstlisting}
build(new LitPrint(){})
\end{lstlisting}
results in \lstinline{"(x 3)"}, as the result of pretty-printing. Now we observe that an ``object'' of expression actually
has the function type \lstinline{LitAlg[E] =>} \lstinline{E} for generic \lstinline{E}. At this point, one may argue
that it is illusory compared to traditional objects, and since we are struggling against the parsing problem, if a parser
produces structures as functions, it could be hardly be further processed like objects, especially, in the design of a compiler
there will be a lot of desugarings (or transformations). However, we argue that this is not an issue, because functions can be used
in a very clever way. For example, if one wants to transform an \lstinline{ExprAlg} expression to a \lstinline{LitAlg} expression by
replacing some variables with values based on a variable environment, a transformation algebra can be implemented as follows:
\begin{lstlisting}
trait Refactor[E] extends ExprAlg[E] {
    def alg : SubAlg[E]
    def env : Map[String, Int]
    def varE(x : String) = if (env.contains(x)) alg.litE(env(x)) else alg.varE(x)
}
\end{lstlisting}\haoyuan{Is this example OK?}
When \lstinline{env = Map("y"} \lstinline{-> 3)}, the \lstinline{ExprAlg} expression \lstinline{"x y"} is changed to \lstinline{LitAlg} expression \lstinline{"x 3"}. The abstract method
\lstinline{alg} in \lstinline{Refactor} can be any algebra of \lstinline{SubAlg}, simply for delegation. By passing a concrete implementation
to \lstinline{alg}, \lstinline{Refactor} will be able to apply transformation to an \lstinline{ExprAlg} expression before \lstinline{alg} returns the result.

\subsection{Parsing with Object Algebras}\label{subsec:parsingwithoa}

Just as shown above, \lstinline{ExprAlg} is defined for a small language in an Object-Algebra style. Furthermore,
A parser can be defined for this whole language at once. Since Object Algebras represent ``objects'' implicitly as functions like \lstinline{ExprAlg[E] =>} \lstinline{E} for generic \lstinline{E}, such a parser should consume an algebra from its parameter, then return a value of \lstinline{Fix[Parser[E]]} as in open recursion, where \lstinline{E} is abstract.
\begin{lstlisting}
trait ExprParser[E] {
    val pE : ExprAlg[E] => Fix[Parser[E]] = alg => p =>
        ident ^^ alg.varE |||
        p ~ p ^^ { case e1 ~ e2 => alg.appE(e1, e2) }
}
\end{lstlisting}
It is observed that \lstinline{pE} is in fact a parser generator, where \lstinline{alg} appears as a parameter, which can be any algebra, and after \lstinline{alg} is fed, it returns a \lstinline{Fix[Parser[E]]}. Again \lstinline{p} is the explicit self-reference of the open parser. Inside the body, the algebra is invoked correspondingly for all cases right after parsing. Note that the two cases are combined using the original \lstinline{|||} operator. To make \lstinline{alg} as a generic algebra, the enclosing trait of the parser, namely \lstinline{ExprParser}, is abstracted over \lstinline{E}. Behaviors, or algebras that can be fed to such a parser, are expected to be defined independently from the Object Algebra interface, just like the algebra \lstinline{ExprPrint} in Section~\ref{subsec:objectalgebras}.

To explain the modularity, we again add literals to expressions, but instead of using inheritance as Section~\ref{subsec:objectalgebras}, \lstinline{LitAlg} is defined independently together with its own parser and algebra:
\begin{lstlisting}
trait LitAlg[E] {
    def litE(n : Int) : E
}

trait LitPrint {
    def litE(n : Int) = n.toString
}

trait LitParser[E] {
    val pE : LitAlg[E] => Fix[Parser[E]] = alg => p =>
        numericLit ^^ alg.litE
}
\end{lstlisting}
In this case, \lstinline{LitAlg} is yet another small language, which is self-contained. To merge the two languages together, we require their grammars
to be merged, and we obtain a combined language, whose parser integrates the two small parsers using alternative. In Scala we use compound types, namely use \lstinline{"with"} for two Object Algebra interface types, to avoid polluting the namespace.

\begin{lstlisting}
trait ExprLitParser[E] {
    val pE : (ExprAlg[E] with LitAlg[E]) => Fix[Parser[E]] = alg => p => {
        val pExprE = new ExprParser[E](){}.pE(alg)(p)
        val pLitE = new LitParser[E](){}.pE(alg)(p)
        pExprE ||| pLitE
    }
}
\end{lstlisting}

%But we can only abstract over the whole interface type, instead of working on its type parameters, as the number of type parameters is not fixed, where we will discuss more on multiple syntax later. In summary, \lstinline{"|||"} produces a combined parser, which takes a compound algebra as its parameter.

In client code, a user can either define a new function with type \lstinline{(ExprAlg[E] with LitAlg[E]) =>} \lstinline{Parser[E]} based on \lstinline{pE}, which implies an algebra is applied after \lstinline{fix}, or simply feed an algebra to \lstinline{pE} at first, then obtain the result from \lstinline{fix}:
\begin{lstlisting}
trait ExprLitPrint extends ExprPrint with LitPrint
val parsePrint = fix(new ExprLitParser[String](){}.pE(new ExprLitPrint(){}))
val result = runParser(parsePrint)("x 3") // prints "(x 3)"
\end{lstlisting}
\haoyuan{Is there a way to avoid defining ExprLitPrint?}


\section{More Features}

The use of inheritance-based approach and Object Algebras enables us
to build modular parsers, which are able to evolve with together with
syntax. This section explores more interesting features, including
parsing multi-sorted syntax, overriding existing parsing rules,
language components for abstracting language features, and alternative
techniques under the whole framework.

\subsection{Parsing Multi-Sorted Syntax}\label{subsec:differentsyntax}

\begin{comment}
As illustrated above, using Object Algebras separates data structures from behaviors, thus enabling more modularity and reuse. New language constructs correspond to the new cases in the algebra. Different operations
 on structures, with both code reuse and separate compilation supported.
\end{comment}

Using Object Algebras is that it is easy to
model multi-sorted languages. If the syntax contains multiple sorts, we can distinguish them by different type parameters. As an example, we extend the language of literals, additions and variables by introducing types as a new kind of type argument, and we add lambda abstractions to expressions. Now the language has two sorts in the syntax, types and expressions:\\

\begin{tabular}{m{0.45\linewidth}m{0.45\linewidth}}
\setlength{\grammarindent}{5em}
\begin{grammar}
<type> ::= `int' \alt <type> `->' <type>
\end{grammar}
&
\setlength{\grammarindent}{5em}
\begin{grammar}
<expr> ::=  ... \alt `\\' <ident> `:' <type> `.' <expr>
\end{grammar}
\end{tabular}

Figure~\ref{fig:multi} illustrates the corresponding Scala code
that extends the Object Algebra interface, pretty-printing operation and parser. We use two type parameters \inlinecode{E} and \inlinecode{T} for expressions and types. They guarantee that invalid terms such as \inlinecode{int + int} will be rejected by the parser.
Besides lexing, the trait \inlinecode{TypedLamOAParser} also introduces parsers for types, and the new case for expressions.
We use \inlinecode{pTypedLamT} and \inlinecode{pTypedLamE} as copies of current \inlinecode{pT} and \inlinecode{pE}, due to the issue
with \inlinecode{super} in Scala (see discussion in Section~\ref{subsec:parsingwithoa}). \inlinecode{pT} and \inlinecode{pE} are the parsers used for performing recursion.

\begin{figure}[ht]
\lstinputlisting[linerange=31-59]{../Scala/Parser/src/PaperCode/Sec4OA/Code4.scala}% APPLY:linerange=OVERVIEW_OA_MULTI_SYNTAX
\caption{Code for parsing a language with multi-sorted syntax for types and expressions.}
\label{fig:multi}
\end{figure}

From the code, we observe that the multi-sorts of Object Algebra interface
express multiple syntax. The isolation for different syntax
is guaranteed by the type system. The following illustrates some client code
for the new parser:

\lstinputlisting[linerange=63-66]{../Scala/Parser/src/PaperCode/Sec4OA/Code4.scala}% APPLY:linerange=OVERVIEW_OA_MULTI_SYNTAX_CLIENT
The code works as expected to parse and pretty-print the result.

\subsection{Overriding Existing Rules}

As many syntactically extensible parsers, our approach also supports
changing part of existing parsers, but in a type-safe way. We can
easily update existing parsing rules with new implementations, or
eliminate them in extended parsers. This can be useful in many
situations, including when conflicts or ambiguities arize when
composing languages.
As an illustration, suppose we have an untyped lambda abstraction case in a base parser as below. \inlinecode{pLam} parses a lambda symbol, an identifier, a dot and an expression in sequence. Some irrelevant details are omitted.

\lstinputlisting[linerange=76-81]{../Scala/Parser/src/PaperCode/Sec5/Code1.scala}% APPLY:linerange=BASEPARSER_UNTYPEDLAM

Then we want to replace the untyped lambda abstractions by typed
lambdas. With inheritance and method overriding, it is easy to only
change the implementation of \lstinline{pLam} in the extended parser.
The new \inlinecode{pLam} parses a colon and a type in addition,
between the identifier and the dot. Due to dynamic dispatch, our new
implementation of lambdas will be different without affecting the other parts of the parser.

\lstinputlisting[linerange=85-90]{../Scala/Parser/src/PaperCode/Sec5/Code1.scala}% APPLY:linerange=EXTPARSER_TYPEDLAM


One can even ``eliminate'' a production rule in the extension, by overriding it with a failure parser. The lexer can also be updated, since keywords and delimiters are represented by sets of strings.

\begin{comment}
It is natural to keep overriding existing parsers, whereas another potential use of this pattern is to reuse old versions of a parser. For instance,
we have a \lstinline{NewParser} that extends \lstinline{ExtParser}, with some more cases added, but for the lambda case we want to go back to the untyped one.
It can be achieved by instantiating an instance of \lstinline{BaseParser} to obtain the old \lstinline{pLam}. However, we must have \inlinecode{lazy} modifier for all the \inlinecode{pLam} starting from \inlinecode{BaseParser}.

\lstinputlisting[linerange=94-102]{../Scala/Parser/src/PaperCode/Sec5/Code1.scala}% APPLY:linerange=NEWPARSER_UNTYPEDLAM
\end{comment}

\subsection{Language Components}\label{subsec:language-component}

Modular parsing not only enables us to build a corresponding parser
which evolves with the language together, but also allows us to
abstract language features as reusable, independent components.
Generally, a language feature includes related abstract syntax,
methods to \textit{build} the syntax (parsing), and methods to
\textit{process} the syntax (evaluation, pretty-printing, etc.). From
this perspective, not only one language, but many languages can be
developed in a modular way, with common language features reused.

Instead of designing and building a language from scratch, we can
easily add a new feature by reusing the corresponding language
component. For example, if a language is composed from a component of
boolean expressions, including if-then-else, it immediately knows how
to parse, traverse, and pretty-print the if-then-else structure.
Grouping language features in this way can
be very useful for rapid development of DSLs.

For implementation, a language component is represented by a Scala object, and it consists of three parts: Object Algebra interface, parser, and Object Algebras.

\begin{itemize}
    \item \textbf{Object Algebra interface:} defined as a trait for the abstract syntax. The type parameters represent multiple sorts of syntax, and the methods are constructs.
    \item \textbf{Parser:} corresponding parser of the abstract syntax, written in a modular way as we demonstrated in previous sections.
    \item \textbf{Object Algebras (optional):} concrete operations on ASTs, such as pretty-printing.
\end{itemize}

We take the example in Section~\ref{subsec:parsingwithoa} again, which is a language of literals, additions and variables. It is defined as a
language component on the left side of Figure~\ref{fig:lng-components}.
Then for the extension of types and lambda abstractions in Section~\ref{subsec:differentsyntax}, instead of inheriting from the previous language directly, we define it as another independent language component. The code is presented on the right side of Figure~\ref{fig:lng-components}.

\begin{figure}[t]
\begin{tabular}{m{0.42\linewidth}m{0.52\linewidth}}
\lstinputlisting[linerange=113-132]{../Scala/Parser/src/PaperCode/Sec5/Code1.scala}% APPLY:linerange=LANGUAGE_COMPONENTS_VAREXPR
&
\lstinputlisting[linerange=136-156]{../Scala/Parser/src/PaperCode/Sec5/Code1.scala}% APPLY:linerange=LANGUAGE_COMPONENTS_TYPEDLAM
\end{tabular}
\caption{Language components \lstinline{VarExpr} (left) and \lstinline{TypedLam} (right).}\label{fig:lng-components}
\end{figure}

As shown below, those two language components can be merged into one to obtain the language we want. Furthermore, the new language is still a modular
component ready for future composition. In that case modularity is realized over higher-order hierarchies.

\begin{figure}[t]
\lstinputlisting[linerange=160-170]{../Scala/Parser/src/PaperCode/Sec5/Code1.scala}% APPLY:linerange=LANGUAGE_COMPONENTS_VARLAMEXPR
\caption{Reuse two language components \lstinline{VarExpr} and \lstinline{TypedLam}.}\label{fig:compose-components}
\end{figure}

The only drawback is that the glue code of composition appears to be
boilerplate. As shown above, we are combining ASTs, parsers and
pretty-printers of \lstinline{VarExpr} and \lstinline{TypedLam}
respectively. Such a pattern refers to \textit{family
  polymorphism}~\cite{ernst01FP} which is unfortunately not fully supported
in Scala, since nested classes/traits have to be manually composed. 
%Nonetheless, one can avoid such boilerplate by using metaprogramming techniques.

\subsection{Alternative Techniques}

Under our modular parsing framework, we use Packrat parsing as the underlying parsing technique, OO inheritance with method overriding for composing and extending parsers, and Object Algebras for parsing extensible ASTs. However, our framework is more general and thus more powerful, because those aspects are orthogonal to each other, and hence can have alternatives.

\begin{itemize}

\item {\bf Parsing Technique}

Our approach does not depend on a particular parser combinator library
or parsing algorithm. We demonstrated that Packrat parsing has some
advantages and is suitable in a modular setting. However, under
certain circumstances or in other programming languages, it may not be
the best choice. One can use other parser combinator libraries as long
as th guidelines proposed in Section~\ref{sec:packrat} are met.

\item {\bf Composing and Extending Parsers}

  We use traits to model parsers, OO inheritance to compose them, and
  method overriding for their extensibility. These language features
  may not be supported in other programming languages, especially in
  functional languages such as Haskell. Nevertheless, \textit{open
    recursion}~\cite{CookThesis} could be used as an alternative. An explicit
  ``self-reference'' parameter is able to explain the recursive calls
  dynamically, by the real argument passed at runtime.

\item {\bf Extensible ASTs}

Besides Object Algebras, many other techniques including \textit{Data
  types à la carte} (DTC)~\cite{swierstra2008data} ot the Cake pattern~\cite{odersky2005independently} also support extensible data structures. These alternatives could also be adopted to build ASTs and corresponding parsers.

\end{itemize}

As we demonstrated, the whole modular parsing framework itself is
modular, which means it can be customized easily. 
%We conducted some previous experiments of modular parsing in Haskell, based on open recursion, MRM and Parsec. However, we suffered from issues of Parsec, as illustrated in Section~\ref{subsec:challenges}.


\section{Case Studies}\label{sec:casestudy}


To demonstrate the utility of our modular parsing approach, we
implemented parsers of the first 18 calculi from book \textit{Types and Programming Languages} (TAPL) \cite{pierce2002types}. We compared our implementation with a non-modular implementation
we found online, which is also written in Scala and uses the same Packrat parsing library.
We counted source lines of code (SLOC) and measured execution time for both implementations.
The result suggests that our implementation saves 69\% code comparing with that non-modular one.

\subsection{Implementation}\label{subsec:implementation}

TAPL introduces several calculi from simple to complex, by gradually adding new features to syntax. These calculi are suitable for our case study for mainly two reasons. Firstly, they capture many of the language features
required in realistic programming languages, such as lambdas, records and variants. Secondly, the evolution of
calculi in the book reveals the advantages of modular representation
of abstract syntax and modular parsing, which is the key functionality
of our approach. By extracting common components from those calculi
and reusing them, we obtain considerably code reuse as shown later.

\paragraph{Extracting Language Components}
Using the pattern demonstrated in Section~\ref{subsec:language-component}, we extract
reusable components from all the calculi. Each
component, which may contain several syntactical structures,
represents a certain feature of the language. For
example, the \lstinline{VarApp} component below represents variables and
function applications.

\lstinputlisting[linerange=13-29]{code/src/papercode/Sec6CaseStudy/Code1.scala}% APPLY:linerange=CASESTUDY_VARAPP

Similar as before, each component is represented by a Scala object which includes \lstinline{Alg} for the abstract syntax, \lstinline{Print} for pretty-printing, and \lstinline{Parse} for parsing.

We have some naming conventions in our code, \lstinline{E} represents
expressions, \lstinline{T} represents types and \lstinline{K}
represents kinds. They are the three sorts of syntax in our case study.
In the component \lstinline{VarApp} we only have expressions. We use some helper traits for eliminating duplicate definitions, such as \inlinecode{EParser} containing \inlinecode{pE} for parsing expressions.

\lstinputlisting[linerange=9-9]{code/src/papercode/Sec6CaseStudy/Code1.scala}% APPLY:linerange=CASESTUDY_EPARSER

\paragraph{Composing Language Components}
Each calculus could be composed directly from components and other
calculi as needed. For example, the calculus \lstinline{Untyped} in our case study,
representing the famous untyped lambda calculus, is constructed from components \lstinline{VarApp}
and untyped lambda abstraction \lstinline{UntypedAbs}.

The code of building \lstinline{Untyped} is presented in Figure~\ref{fig:casestudy-untyped}. Note that in the parser \inlinecode{Parse}, we need to override the object algebra interface and
parsing functions accordingly.

\begin{figure}[t]
\lstinputlisting[linerange=33-60]{code/src/papercode/Sec6CaseStudy/Code1.scala}% APPLY:linerange=CASESTUDY_UNTYPED
\caption{Build the \inlinecode{Untyped} calculus by composing to language components.}\label{fig:casestudy-untyped}
\end{figure}


\paragraph{Dependency Overview}
Figure \ref{fig:dependency} shows the
dependency of all the components and calculi in our case study. Grey
boxes are calculi and white boxes are components. An arrow starting
from box A to box B denotes that B includes and thus reuses A.

\begin{figure*}
    \centering
    \includegraphics[width=\textwidth]{resources/depGraph.pdf}
    \caption{Dependency graph of all calculi and components in case study.}
    \label{fig:dependency}
\end{figure*}

As shown in the graph, some components such as \lstinline{VarApp} are
created from scratch, while others such as \lstinline{Typed} are
extended from existing components. Since calculi and components have similar signatures, each calculus
can also be extended and reused directly. For example, calculus \lstinline{FullRef} extends from
calculus \lstinline{FullSimple}.

From the dependency graph, we know
that common components such as \lstinline{VarApp} are reused in
lots of calculi. Such reuse could shorten the code considerably. We
will show this advantage and inspect the possible performance penalty in
a quantitative way in the next subsection.

\subsection{Comparison}\label{subsec:cs-comparison}

\newcommand\ourimpl{$\texttt{Mod}_{\texttt{OA}}$}
\newcommand\ilyaimpl{\texttt{NonMod}}
\newcommand\ourclass{$\texttt{Mod}_{\texttt{CLASS}}$}
\newcommand\ilyalongest{$\texttt{NonMod}_{\texttt{|||}}$}

We compared our implementation (named \ourimpl{}) with an implementation
by Ilya Klyuchnikov (named \ilyaimpl{}), available online\footnote{https://github.com/ilya-klyuchnikov/tapl-scala/}.
\ilyaimpl{} is suitable for comparison, because it is also
written in Scala using the same parser combinator library.
Furthermore, it includes parsers of all the 18 calculi we have, but
written in a non-modular way. Thus it is not able to reuse existing
code when those calculi share common features.

The comparison is made from two aspects. First, we want to discover
the amount of code reuse using our modular parsing approach.
For this purpose, we measured source lines of code (SLOC) of two implementations.
Second, we are interested to assess the performance penalty caused by modularity.
Thus we compared the execution time of parsing random expressions between two implementations.

\paragraph{Standard of Comparison}
In the SLOC comparison, all blank lines and comment lines are excluded,
and we formatted the code of both implementations to guarantee that
the length of each line does not exceed 120 characters. Furthermore,
because \ilyaimpl{} has extra code such as semantics,
we removed all irrelevant code and only keep abstract
syntax definition, parser and pretty-printer for each calculus, to
ensure a fair comparison.

For the comparison of execution time, we built a generator to randomly
generate valid expressions of each calculus, according to the syntax. These expressions are
written to test files, one file per calculus. Each test file consists of 500
expressions randomly generated, and the size of test files varies from 20KB to 100KB.
Then we run the corresponding parser to parse the file and the pretty-printer to print the result.
The average execution time of 5 runs excluding reading input file was calculated, in milliseconds.

\begin{table*}
    \centering
    \begin{tabular}{|l|r|r|r|r|r|r|}
      \hline
        \multirow{2}{*}{\bfseries Calculus Name} & \multicolumn{3}{ c| }{\bfseries SLOC} & \multicolumn{3}{ c| }{\bfseries Time (ms)} \\ \cline{2-7}
        \multicolumn{1}{|c|}{} & \ilyaimpl{} & \ourimpl{} & \bfseries (+/-)\% & \ilyaimpl{} & \ourimpl{} & \bfseries (+/-)\% \\
      \hline
      Arith & 77 & 79 & +2.6 & 112 & 117 & +4.5 \\
Untyped & 48 & 59 & +22.9 & 85 & 134 & +57.6 \\
FullUntyped & 131 & 95 & -27.5 & 199 & 277 & +39.2 \\
TyArith & 89 & 55 & -38.2 & 89 & 105 & +18.0 \\
SimpleBool & 90 & 59 & -34.4 & 114 & 149 & +30.7 \\
FullSimple & 240 & 144 & -40.0 & 350 & 491 & +40.3 \\
Bot & 87 & 61 & -29.9 & 101 & 229 & +126.7 \\
FullRef & 273 & 79 & -71.1 & 344 & 537 & +56.1 \\
FullError & 111 & 52 & -53.2 & 141 & 191 & +35.5 \\
RcdSubBot & 125 & 32 & -74.4 & 147 & 171 & +16.3 \\
FullSub & 221 & 50 & -77.4 & 297 & 343 & +15.5 \\
FullEquiRec & 246 & 52 & -78.9 & 415 & 514 & +23.9 \\
FullIsoRec & 255 & 61 & -76.1 & 323 & 383 & +18.6 \\
EquiRec & 81 & 31 & -61.7 & 100 & 84 & -16.0 \\
Recon & 138 & 37 & -73.2 & 163 & 172 & +5.5 \\
FullRecon & 142 & 37 & -73.9 & 175 & 197 & +12.6 \\
FullPoly & 244 & 89 & -63.5 & 358 & 625 & +74.6 \\
FullOmega & 311 & 89 & -71.4 & 342 & 394 & +15.2 \\
\hline
Total & 2909 & 1161 & -60.1 & 3855 & 5113 & +32.6 \\

      \hline
      \multicolumn{7}{c}{}
    \end{tabular}
    \caption{Comparison of SLOC and execution time.}
    \label{tab:comparison}
\end{table*}

\paragraph{Comparison Results}
Table \ref{tab:comparison} shows results of the comparison.
The overall result is that 69.2\% of code is reduced using our
approach, and our implementation is 42.7\% slower.

The good SLOC result is because of that the code of common language features
such as variables, lambda abstractions, etc., are reused lots of times in
the whole case study. We can see that in the first two calculi
\lstinline{Arith} and \lstinline{Untyped} we are not better than \ilyaimpl{}, because in such two cases we do not reuse any existing
components. However in the following 16 calculi, we can reuse
language components, resulting considerably code reduction. In particular,
the calculi \inlinecode{EquiRec}, \inlinecode{Recon} and some others are only 22 lines
in our implementation, because we only compose existing codes in such cases.

To discover the reasons of slower execution time, we made some more experiments
on two possible factors which could affect the performance.
They are object algebra and the longest match alternative combinator.
We use object algebra for ASTs and the longest match alternative combinator \inlinecode{|||} for parsing,
while \ilyaimpl{} uses case class and the ordinary alternative combinator.

We implemented two more versions. One is a modified version of our implementation, named \ourclass{}, with object algebra replaced by case class for the ASTs.
The other is a modified version of \ilyaimpl{}, named \ilyalongest{}, using the longest match alternative combinator instead of the ordinary one.

\begin{table*}
    \centering
    \begin{tabular}{|l|r|r|r|r|r|r|r|}
      \hline
        \multirow{2}{*}{\bfseries Calculus Name} & \ilyaimpl{} & \multicolumn{2}{ c| }{\ourimpl{}} & \multicolumn{2}{ c| }{\ilyalongest{}} & \multicolumn{2}{ c| }{\ourclass{}} \\ \cline{2-8}
        \multicolumn{1}{|c|}{} & \multicolumn{1}{c|}{\bfseries Time} & \bfseries Time & \bfseries (+/-)\% & \bfseries Time & \bfseries (+/-)\% & \bfseries Time & \bfseries (+/-)\% \\
      \hline
        Arith & 741 & 913 & +23.2 & 793 & +7.0 & 932 & +25.8 \\
        Untyped & 770 & 1018 & +32.2 & 821 & +6.6 & 1007 & +30.8 \\
        FullUntyped & 1297 & 1854 & +42.9 & 1343 & +3.5 & 1767 & +36.2 \\
        TyArith & 746 & 888 & +19.0 & 772 & +3.5 & 918 & +23.1 \\
        SimpleBool & 1376 & 1782 & +29.5 & 1494 & +8.6 & 1824 & +32.6 \\
        FullSimple & 1441 & 2270 & +57.5 & 1574 & +9.2 & 2226 & +54.5 \\
        Bot & 1080 & 1287 & +19.2 & 1078 & -0.2 & 1306 & +20.9 \\
      %\hline
        %\multicolumn{1}{|c|}{\dots} & \multicolumn{7}{c|}{\dots} \\
      %\hline
        FullRef & 1438 & 2291 & +59.3 & 1544 & +7.4 & 2142 & +49.0 \\
        FullError & 1410 & 1946 & +38.0 & 1524 & +8.1 & 1981 & +40.5 \\
        RcdSubBot & 1247 & 1524 & +22.2 & 1285 & +3.0 & 1612 & +29.3 \\
        FullSub & 1320 & 1979 & +49.9 & 1393 & +5.5 & 1899 & +43.9 \\
        FullEquiRec & 1407 & 2200 & +56.4 & 1561 & +10.9 & 2156 & +53.2 \\
        FullIsoRec & 1492 & 2253 & +51.0 & 1648 & +10.5 & 2236 & +49.9 \\
        EquiRec & 994 & 1254 & +26.2 & 1048 & +5.4 & 1304 & +31.2 \\
        Recon & 1044 & 1482 & +42.0 & 1128 & +8.0 & 1506 & +44.3 \\
        FullRecon & 1094 & 1645 & +50.4 & 1161 & +6.1 & 1652 & +51.0 \\
        FullPoly & 1398 & 2086 & +49.2 & 1511 & +8.1 & 2019 & +44.4 \\
        FullOmega & 1451 & 2352 & +62.1 & 1582 & +9.0 & 2308 & +59.1 \\
      \hline
        Total & 21746 & 31024 & +42.7 & 23260 & +7.0 & 30795 & +41.6 \\
      \hline
        \multicolumn{8}{c}{}
    \end{tabular}
    \caption{Execution time of four implementations.}
    \label{tab:ext-comparison}
\end{table*}

Table \ref{tab:ext-comparison} shows comparison results of these four versions. It suggests that the difference of running time between
using object algebra and class is little, roughly 1\%.
The usage of longest match combinator slows the performance by 7\%. The main reason of slower
execution time may be the overall structure of the modular parsing approach, because we indeed have
more intermediate function calls and method overriding. However, it is worth mentioning that
because of the memoization technique of Packrat parsers, we are only constant times
slower, the algorithmic complexity is still the same.


\section{Related Work}\label{sec:relatedwork}

- extensible parsing, language workbenches: rats, noa (this one already uses OA), modular semantic actions, (syntactic modularity, no separate compilation, modular type-checking)
(read more papers, see if they talk about this issue, some potential solutions)
(attribute grammars?)

- parser combinators for type-checking, previous work has not shown how to support modularity (ASTs); left-recursion and back-tracking in related techniques

- modularity: object algebras, dtc and mrm (problem with parsing, is there any related work? (PB: a paper on unfolds: build the AST))

(parsing in Javascript: using delegation, does it support modular AST)

noa, shy: shy: only override some interesting cases (transformation is tedious)
bruijn indices: parsing + transformation

Our work integrates several components including extensible parsing, parser combinators and modular datatypes. There has been a great amount of related papers
on those hot topics, of which some inspired us of this paper and encourages us for more exploration. This section will try to lead a discussion on what difference we have made.

\paragraph*{Extensible Parsing} Extensible parsing is achieved in many different areas, of which parser generators are a mainstream area specially designed for modular syntax and parsing. Many parser generators [OMeta, ANTLR, Rats!, noa, Ohm] \haoyuan{correct for Ohm?} support modular grammars, more specifically, they allow users to create new modules where new non-terminals and production rules can be introduced, some can even override existing rules in the old grammar modules. For instance, \textit{Rats!} [Rats!]
constructs its own module system for the collection of grammars, while NOA [NOA] uses Java annotation processing to gather information together. Those parser combinators focus on the \textit{syntactic extensibility} of grammars, and rely on a whole compilation to generate a global parser, even with a slight modification. Some of them may statically check the correctness and unambiguity of grammars, but separate compilation and modular type-checking remain unsolved, together with the issue of performance.

Besides, macro systems like C preprocessor, C++ templates [..] and Racket [..], and other meta-programming techniques [..] are a similar area aiming at syntactic extensibility as well, which sacrifices type safety. SugarJ [] is another well-known tool that conveniently introduces syntactic sugars in Java programming by library imports. Composition of syntactic sugars is easy for users, whereas it requires many rounds of parsing and adaption, which highly affects efficiency of compilation, moreover, the implementation was based on SDF [] and Stratego [], which focused little on separate compilation. Extensible compilers like JastAdd [] and Polyglot [], however, are somehow more ambitious, but they involve extensible parsing mostly by parser generators as well, and they focus more on the extensions to a host language. Our library is designed for modular language parsers in a type-safe way, with flexible language composition. Although overriding existing production rules is tricky with our approach, we could perhaps make use of embedded domain-specific languages on top of parser combinators and transformations for overriding, but it is anyway an orthogonal issue. \haoyuan{???} \haoyuan{BTW what about Racket?} \haoyuan{what about metafront? it is a macro system but does it have type safety?} \haoyuan{Extensible syntax with lexical scoping?} \haoyuan{"Extensible syntax" proposes a system for extensible syntax, where users write EDSLs in their language with concrete syntax. Users can write rules for type-based disambiguation. But separate compilation is again not mentioned. Shall we mention that thesis?} \haoyuan{attribute grammars?}

On the other hand, extensible parsing algorithms are another area that indeed relates to separate compilation. Specifically, [MB] introduces \textit{parse table composition}, in which paper grammars are compiled to the generation of parse tables, which are DFAs or NFAs, later they can be composed by an algorithm, so as to provide separate compilation for parsing. Nevertheless, the generation of parse tables is rather expensive, and furthermore, our approach supports separate compilation as well as modular type-checking, and the idea is not restricted to Scala but applicable to many functional languages, on whose type system the safety of modular parsing can rely. The extensibility of parsing in our approach is further available at language composition and lexical level.
\haoyuan{I mentioned one shortcoming of our approach here, which is we cannot override existing production rules; another one is that we do not have explicit correspondence/relationship between abstract syntax and the parser.}

\paragraph*{Parser Combinators} Since [Recursive Programming Techniques 1975] firstly introduced parser generators, and after Wadler discussed more details on backtracking in [1985], parser combinators have been more and more popularly developed and used in the research area of parsing. Among them many parsing libraries produces recursive descent parsers by introducing functional, or more specifically, monadic parser combinators with a foundation on [monadic parser combinators]. Related work includes Parsec [], which is frequently used in Haskell for context-sensitive grammars with infinite lookahead. Nevertheless, left-recursion is known as a big issue in recursive descent parsers, in which case programmers using libraries like Parsec have to manually eliminate left-recursion in the grammar, which is cumbersome, but also it distorts the shape of its old grammar, restricting the modularity of parsing to extreme extent.
There are certainly some approaches to afford the limitation of parser combinators, for example, we have conducted our previous experiment in Haskell, by using open recursion, MRM [] and Parsec. To support direct left-recursive grammars, we made use of the state monad in Parsec to ensure that a left-recursive production rule will not be applied successively in parsing. It turned out to be successful, yet introduced complexity in programming, hence we believe that the parsing library is obligated for left-recursion.

Some more recent papers [Packrat 2006, parsing with derivatives 2011, parser combinator for ambiguous left-recursive grammars] proposed a series of novel parsing techniques, moreover, they did put an eye on the issue of left-recursion. As we explained before, we have selected Packrat Parsing for our prototype implementation, as the combinators are convenient to use in Scala, which is a platform where functional laziness and Object Algebras can perfectly coexist. Also, its support for direct left-recursion is already helpful for designing real-world parsers practically, though [packrat parsing can support left recursion] further demonstrated that general left-recursion is supported from the theoretical point of view. We have also mentioned that the components used in our library can have alternatives, for instance, a different set of parser combinators may support ambiguous and left-recursive grammars, or be applicable to a different subset of context-free grammars or more, or even lead us to another level of performance.

Another important issue in previous papers, to the best of our knowledge, is that none of them deal with modular datatypes or abstract syntax trees. \haoyuan{Well, I need to go to check JS and Patrick Bahr's paper.} The modularity of data structures is yet necessary for modular parsing that supports separate compilation and modular type-checking. In our library, Object Algebras have been adopted for extensible ASTs, together with the behaviors on them.

\paragraph*{Modular Datatypes} Text.


\section{Conclusion}\label{sec:conclusion}

\haoyuan{Huang: to write a couple of sentences for a summary.}

For a better user experience, there are certainly some aspects in our pattern for modular parsing that needs
improvements. Specifically, in Section~\ref{subsec:implementation} we have observed that the glue code in \lstinline{Untyped}
appears to be boilerplate, as we need to compose the data structures, parsers and operations respectively from the same set of super types. Such an issue refers to \textit{family polymorphism} [], and we may rely it on a language feature or meta-programming techniques. Moreover, to make better use of Object Algebras, we can possibly adopt some patterns in Shy, together with the composition of algebras, to reduce boilerplate by using code generation.

For future work, we will experiment more on how open recursion contributes to extensible parsing in functional languages, by making use of laziness. It is also interesting to see that parsing, to some extent, can be viewed as a special example of unfolds. So it is worthwhile considering to generalize the composition of unfolds under certain circumstances.


%% Acknowledgments
\begin{acks}                            %% acks environment is optional
                                        %% contents suppressed with 'anonymous'
  %% Commands \grantsponsor{<sponsorID>}{<name>}{<url>} and
  %% \grantnum[<url>]{<sponsorID>}{<number>} should be used to
  %% acknowledge financial support and will be used by metadata
  %% extraction tools.
  This material is based upon work supported by the
  \grantsponsor{GS100000001}{National Science
    Foundation}{http://dx.doi.org/10.13039/100000001} under Grant
  No.~\grantnum{GS100000001}{nnnnnnn} and Grant
  No.~\grantnum{GS100000001}{mmmmmmm}.  Any opinions, findings, and
  conclusions or recommendations expressed in this material are those
  of the author and do not necessarily reflect the views of the
  National Science Foundation.
\end{acks}


%% Bibliography
\bibliography{paper}


%% Appendix
\appendix
\section{Appendix}

Text of appendix \ldots

\end{document}
