\section{Extensible ASTs with Object Algebras}\label{sec:algebrasandparsing}

The inheritance-based approach benefits us from building extensible parsers, based on an OO class hierarchy.
Nevertheless, the addition of new operations over the AST becomes problematic. Often it requires modification on
the existing code, which breaks modularity. In this section, we will have a closer look at how it motivates us to
use Object Algebras for full extensibility.

\subsection{Problem with Traditional OO ASTs}\label{subsec:problemwithoutoa}

Now it comes to extension 2 in Section~\ref{subsec:dimensions}. In the previous code, we have presented pretty-printing as
an example of operations on the AST. Suppose now we also want to collect free variables on the expression language, which models
literals, additions and variables. Based on the traditional OO class hierarchy, one attempt would be extending \lstinline{Expr} with the new
operation directly:

\lstinputlisting[linerange=89-89]{../Scala/Parser/src/PaperCode/Overview/Overview.scala}% APPLY:linerange=ATTEMPT_EXPRWITHFREEVARS
It refers to one Scala solution to the Expression Problem []. But then programmers have to define new classes for literals, additions and variables
by extending the old ones together with \lstinline{FreeVarsExpr}. The critical issue is that, the existing parsing code only generates objects of the old
types, again it is a modularity issue, since it is a bad idea to modify existing code again and again with extensions. Other solutions including modelling constructors as functions introduce unnecessary complexity.

In contrast, Object Algebras~\cite{Oliveira2012} enables us to solve this problem. As a nice solution to the Expression Problem, it separates data variants
and operations, and offers high flexibility in the choice of operations to be performed over the AST. Before adopting it for parsing, the next section introduces the background of Object Algebras.

\subsection{Object Algebras}\label{subsec:objectalgebras}
Object Algebras captures a design pattern to address the Expression Problem nicely,
achieving two dimensions of extensibility (data variants and operations) in a modular and type-safe way.
Because of this, the definition of data structures is separated from behaviors on them, and future extensions
to both sides no longer require existing code to be modified, supporting separate compilation.

In Object Algebras, ASTs as recursive data structures are defined using traits, where each constructor corresponds
to an abstract method inside. The example from Section~\ref{subsec:packratparsing} is again used here for illustration.
At first the language only supports literals and additions:

\lstinputlisting[linerange=8-11]{../Scala/Parser/src/PaperCode/Overview/OA.scala}% APPLY:linerange=OVERVIEW_OA_ALG
The language has two constructors: literals and additions. The trait \lstinline{ExprAlg} is called an \textit{Object Algebra interface},
whereas as a factory, it cannot produces objects for the expressions directly like traditional approaches, but abstract the results in its
type parameter \lstinline{E} instead. To realize an operation on expressions, we simply instantiate the type parameter by a concrete type and
provides implementations for all cases. Below is an example of pretty-printing an expression:

\lstinputlisting[linerange=15-18]{../Scala/Parser/src/PaperCode/Overview/OA.scala}% APPLY:linerange=OVERVIEW_OA_PRINT
Here \lstinline{Print} is called an \textit{Object Algebra}. It traverses an expression bottom-up, and returns a string as the result.
Furthermore, one can define a new trait to implement evaluation on arithmetic expressions:

\lstinputlisting[linerange=22-25]{../Scala/Parser/src/PaperCode/Overview/OA.scala}% APPLY:linerange=OVERVIEW_OA_EVAL

On the other hand, the data variants can be extended by inheriting \lstinline{ExprAlg} and adding new cases only. Now we extend the language
with variables. A new Object Algebra interface \lstinline{VarAlg} is defined as follows:

\lstinputlisting[linerange=29-31]{../Scala/Parser/src/PaperCode/Overview/OA.scala}% APPLY:linerange=OVERVIEW_OA_ALGEXT
Now pretty-printing on the new language can be realized without modifying existing code:

\lstinputlisting[linerange=35-37]{../Scala/Parser/src/PaperCode/Overview/OA.scala}% APPLY:linerange=OVERVIEW_OA_EXTPRINT
We observe that only the new case is implemented for pretty-printing, and the others have been inherited.
It offers programmers with great code reuse, in the sense that existing cases can also be
overridden in \lstinline{VarExprPrint} using \lstinline{"override def"}.

On this, to create an expression of \lstinline{VarExprAlg}, a generic method is defined as follows:

\huang{todo: better name for build}\haoyuan{makeExp}
\lstinputlisting[linerange=41-41]{../Scala/Parser/src/PaperCode/Overview/OA.scala}% APPLY:linerange=OVERVIEW_OA_MAKEEXP
Such a method implicitly represents the expression \lstinline{"1 + x"}. The code

\begin{lstlisting}
makeExp(new VarExprPrint {})
\end{lstlisting}
results in \lstinline{"(1 + x)"}, as the result of pretty-printing. Now we observe that an ``object'' of expression actually
has the function type \lstinline{VarExprAlg[E] =>} \lstinline{E} for generic \lstinline{E}. At this point, one may argue
that it is illusory compared to traditional objects, and since we are struggling against the parsing problem, if a parser
produces structures as functions, it could be hardly be further processed like objects, especially, in the design of a compiler
there will be a lot of desugarings (or transformations). However, we argue that this is not an issue, because functions can be used
in a very clever way. For example, if one wants to transform a \lstinline{VarExprAlg} expression to an \lstinline{ExprAlg} expression by
replacing some variables with values based on a variable environment, a transformation algebra can be implemented as follows:\haoyuan{mention in discussion that shy deals with boilerplate.}

\lstinputlisting[linerange=45-56]{../Scala/Parser/src/PaperCode/Overview/OA.scala}% APPLY:linerange=OVERVIEW_OA_REFACTOR
Here the \lstinline{VarExprAlg} expression \lstinline{"1 + x"} is changed to \lstinline{ExprAlg} expression \lstinline{"1 + 2"}, before the pretty-printing algebra is applied. The abstract method
\lstinline{alg} in \lstinline{Refactor} can actually be any algebra simply for delegation. By passing a concrete implementation
to \lstinline{alg}, \lstinline{Refactor} will be able to transform a \lstinline{VarExprAlg} expression before we obtain the result.

\subsection{Parsing with Object Algebras}\label{subsec:parsingwithoa}

Just as shown above, \lstinline{ExprAlg} is defined for a small language in an Object-Algebra style. Furthermore,
A parser can be defined for this whole language at once. Since Object Algebras represent ``objects'' implicitly as functions like \lstinline{ExprAlg[E] =>} \lstinline{E} for generic \lstinline{E}, such a parser should have type \lstinline{ExprAlg[E] =>} \lstinline{Parser[E]}, where \lstinline{E} can be abstracted by the enclosing type.

\lstinputlisting[linerange=60-68]{../Scala/Parser/src/PaperCode/Overview/OA.scala}% APPLY:linerange=BASE_OA_PARSER_BAD
Such a parser works fine, yet it does not appear to be extensible. For example, we have demonstrated in Section~\ref{subsec:inheritance-approach} that method overriding can update \lstinline{pExpr} for an extended syntax, but now \lstinline{pExpr} will have a different type, like \lstinline{VarExprAlg[E] =>} \lstinline{Parser[E]} where the difference appears in the argument. The difference of types prohibits method overriding, however, this problem refers to \textit{binary methods}. \haoyuan{Shall we mention this?}

One solution to this problem is taking out the argument as a single field, hence it supports overriding as well. Therefore we refactor the parsing code as follows:
\lstinputlisting[linerange=72-81]{../Scala/Parser/src/PaperCode/Overview/OA.scala}% APPLY:linerange=BASE_OA_PARSER
This is precisely the pattern that we advocate for modular parsing. Note that \lstinline{pE} is the one that is used recursively, exported by the trait and considered to be overridden from inheritance. The reason why we use an extra and seemingly redundant \lstinline{pExprE}, is that \lstinline{pE} can no longer be referred to by the \lstinline{super} keyword, as a limitation of Scala for generics, hence we can directly invoke \lstinline{pExprE} as the specific \lstinline{pE} in \lstinline{ExprOAParser}. And we still want code reuse for the composition of \lstinline{pLit} and \lstinline{pAdd}. Now it is worth trying on this pattern with an extension, like \lstinline{VarExprAlg} which we have been using:
\lstinputlisting[linerange=85-96]{../Scala/Parser/src/PaperCode/Overview/OA.scala}% APPLY:linerange=EXT_OA_PARSER
In the client code shown above, an instance of algebra is fed to \lstinline{VarExprOAParser} for pretty-printing expressions. Then the trait exports \lstinline{pE}, parses the input and applies the algebra. The result shows that such a pattern is a nice solution to modular parsing. At this point, one may argue that with a different operation like evaluation, the redundant code is required to obtain the parser, and it results in parsing the same input twice. This is generally solved in the original Object Algebras paper [] by composing two algebras into one in advance.

\subsection{Parsing with Multiple Syntax}\label{subsec:differentsyntax}

One intuition of using Object Algebras, as illustrated above, is to separate data structures from behaviors, so as to modularize the design, and both parts are even more modular and reusable. We can simply add new variants corresponding to the new cases in the grammar, or realize different operations on structures, without modifying existing code, meanwhile separate compilation is supported. Another one is that Object Algebras benefit us from using generics, where we can abstract multiple syntax in the type parameters, and define less classes, simplifying code comparatively. Below we further extend the language with types, by introducing an extra type parameter in the algebra interface. The extended grammar is:\\

\begin{tabular}{m{0.45\linewidth}m{0.45\linewidth}}
\setlength{\grammarindent}{5em}
\begin{grammar}
<type> ::= ... \alt `int' \alt <type> `->' <type>
\end{grammar}
&
\setlength{\grammarindent}{5em}
\begin{grammar}
<expr> ::=  ... \alt `\\' <ident> `:' <type> `.' <expr>
\end{grammar}
\end{tabular}

Hence we write code for the extended AST, pretty-printing operation and the extended parser:

\lstinputlisting[linerange=100-128]{../Scala/Parser/src/PaperCode/Overview/OA.scala}% APPLY:linerange=OVERVIEW_OA_MULTI_SYNTAX
Besides lexing, the trait \lstinline{TypedLamOAParser} also introduces parsers for types, and the new case for expressions. For the same reason,
we use \lstinline{pTypedLamT} and \lstinline{pTypedLamE} as copies of current \lstinline{pT} and \lstinline{pE}, recursively. \lstinline{pT} and \lstinline{pE} are actually the parsers for recursion.

From the code we observe that the multi-sorts of Object Algebra interface
exactly express multiple syntax. The isolation for different syntax
is guaranteed by the type system. Now it is straightforward to show some client code
for the new parser:

\lstinputlisting[linerange=132-134]{../Scala/Parser/src/PaperCode/Overview/OA.scala}% APPLY:linerange=OVERVIEW_OA_MULTI_SYNTAX_CLIENT
Pretty-printing the parsed results just works as expected.
