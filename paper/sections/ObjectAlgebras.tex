\section{Full Extensibility with Object Algebras}\label{sec:algebrasandparsing}

The inheritance-based approach allows building extensible parsers, based on an OO class hierarchy.  Nevertheless, the addition of new operations over ASTs is problematic using traditional OO ASTs. In this section, we
show how to support both the forms of extensibility for ASTs (easy
addition of language constructs, and easy addition of operations)
using Object Algebras~\cite{Oliveira:2012}.

\subsection{Problem with Traditional OO ASTs}\label{subsec:problemwithoutoa}

The Expression Problem~\cite{wadler1998expression} illustrates the difficulty of extending data structures or ASTs in two dimensions. For traditional OO ASTs, adding new language constructs is easy, but adding new operations is hard. In the last section, we have a simple language containing literals, additions and variables. The corresponding ASTs are built using classes with a pretty-printing operation. Now suppose we want to collect free variables over the ASTs. By traditional OO approach, one attempt would be extending \inlinecode{Expr} with the new operation to obtain a new abstract type for ASTs:

\lstinputlisting[linerange=23-23]{../Scala/Parser/src/PaperCode/Sec4OA/Code1.scala}% APPLY:linerange=FVARSEXPR

But then programmers have to define a new class for each construct by extending the old class with \lstinline{FVarsExpr}, and implementing the corresponding \lstinline{fVars} method:

\lstinputlisting[linerange=27-37]{../Scala/Parser/src/PaperCode/Sec4OA/Code1.scala}% APPLY:linerange=FVAR_CLASSES

Besides introducing boilerplate code, more classes have to be defined along with future extensions. They pollutes the namespace heavily. More importantly, the \lstinline{VarExprParser} we defined is no longer suitable for parsing the new ASTs! All the methods in the previous parser has type \inlinecode{Parser[Expr]}, which indicates the result has type \inlinecode{Expr}, whereas the type should be \inlinecode{FVarsExpr} now.

Therefore, in order to generate new AST structures, we have to substitute \lstinline{FVarsExpr}, \lstinline{FVarsLit}, \lstinline{FVarsAdd} and \lstinline{FVarsVar} for the old \lstinline{Expr}, \lstinline{Lit}, \lstinline{Add} and \lstinline{Var} correspondingly. Again, to avoid modifying existing code, we have to sacrifice code reuse. In fact, there is a trade-off between separate compilation and type-safe code reuse. For semantic modularity, we need a different approach for building ASTs.

%\bruno{The following text is not very good, needs to be revised. I think
%we need to show the full code and where it breaks. Since I removed text
%about the Expression problem in S3, I think we need to talk more about it %here.}
%It refers to one Scala solution to the Expression Problem.\huang{what's the %solution? does it have the later "critical issue"? if so, why is it a %"solution"} But then programmers have to define new classes for literals, %additions and variables
%by extending the old ones together with \inlinecode{FreeVarsExpr}. The %critical issue is that, the existing parsing code only generates objects of %the old
%types, again it is a modularity issue, since it is a bad idea to modify %existing code again and again with extensions. Other solutions including %modelling constructors as functions introduce unnecessary complexity.

Fortunately, Object Algebras~\cite{Oliveira:2012} enables us to solve this problem. As a nice solution, it separates data variants
and operations, and offers high flexibility in the choice of operations to be performed over ASTs. Before adopting it for parsing, the next section introduces the background of Object Algebras.

\subsection{Object Algebras}\label{subsec:objectalgebras}
Object Algebras captures a design pattern to address the Expression Problem, achieving two dimensions of extensibility (language constructs and operations) in a modular and type-safe way. The definition of data structures is separated from their behaviours, and future extensions on both dimensions no longer require existing code to be modified, supporting separate compilation.

Using Object Algebras in Scala, ASTs as recursive data structures are defined by traits, where each constructor corresponds to an abstract method inside.
Essentially, Object Algebras generalize the {\sc Abstract Factory} pattern~\cite{gamma1995design}, and promote the use of factory methods, instead of constructors, for instantiating objects. The example from Section~\ref{subsec:packratparsing} is used here again for illustration.
At first the language only supports literals and additions:

\lstinputlisting[linerange=5-8]{../Scala/Parser/src/PaperCode/Sec4OA/Code2.scala}% APPLY:linerange=OVERVIEW_OA_ALG
We have two constructors \inlinecode{lit} and \inlinecode{add}, corresponding to literals and additions. The trait \inlinecode{ExprAlg} is called an \textit{Object Algebra interface}. The trait is parametrized by the type
\inlinecode{E}, which abstracts over the concrete type of the AST.

\paragraph{Adding New Operations}
To realize an operation on expressions, we simply instantiate the type parameter by a concrete type and
provides implementations for all cases. Below is an example of pretty-printing an expression:

\lstinputlisting[linerange=12-15]{../Scala/Parser/src/PaperCode/Sec4OA/Code2.scala}% APPLY:linerange=OVERVIEW_OA_PRINT
Here \inlinecode{Print} is called an \textit{Object Algebra}. It traverses an expression bottom-up, and returns a string as the result.
One can define evaluation operation as a new trait on this expression language:

\lstinputlisting[linerange=19-22]{../Scala/Parser/src/PaperCode/Sec4OA/Code2.scala}% APPLY:linerange=OVERVIEW_OA_EVAL

\paragraph{Adding New AST Constructs}
Furthermore, new language constructs can be added by extending \inlinecode{ExprAlg} and adding new cases only. Now we extend the language
with variables. A new Object Algebra interface \inlinecode{VarAlg} is defined as follows:

\lstinputlisting[linerange=26-28]{../Scala/Parser/src/PaperCode/Sec4OA/Code2.scala}% APPLY:linerange=OVERVIEW_OA_ALGEXT
Now pretty-printing on the new language can be realized without modifying existing code:

\lstinputlisting[linerange=32-34]{../Scala/Parser/src/PaperCode/Sec4OA/Code2.scala}% APPLY:linerange=OVERVIEW_OA_EXTPRINT
An observation is that only the new case is implemented for pretty-printing, and the others have been inherited.
In other words, existing code was reused and was not modified!

To create an expression representing \inlinecode{1 + x}, a generic method is defined as follows:

\lstinputlisting[linerange=41-41]{../Scala/Parser/src/PaperCode/Sec4OA/Code2.scala}% APPLY:linerange=OVERVIEW_OA_MAKEEXP

Note how the construction of the abstract syntax happens through
the use of factory methods, instead of constructors.
To pretty print the expression,
we pass an instance of \inlinecode{VarExprPrint} to \inlinecode{makeExp}:

\begin{lstlisting}
makeExp(new VarExprPrint {})
\end{lstlisting}
which results in \inlinecode{"(1 + x)"} as expected.

\subsection{Parsing with Object Algebras}\label{subsec:parsingwithoa}

Parsing produces ASTs as the result. When Object Algebras are used
to build ASTs, an Object Algebra containing the constructor/factory
methods has to be used by the parsing function. Thus, a first attempt
at defining the parser for the small arithmetic language is:

\begin{comment}
\inlinecode{ExprAlg} is defined for a small language using Object Algebras. Since Object Algebras represent ``objects'' implicitly as functions like \inlinecode{ExprAlg[E] =>} \inlinecode{E} for abstract \inlinecode{E}, we first try to build the corresponding parser by type \inlinecode{ExprAlg[E] =>} \inlinecode{Parser[E]}.
\end{comment}

\lstinputlisting[linerange=8-16]{../Scala/Parser/src/PaperCode/Sec4OA/Code3.scala}% APPLY:linerange=BASE_OA_PARSER_BAD
Such a parser works fine, but it is not extensible! For example, we have demonstrated in Section~\ref{subsec:overriding} that method overriding is essential to update \inlinecode{pExpr} for an extended syntax. However, trying to do a similar method override for \inlinecode{pExpr} would require a type \inlinecode{VarExprAlg[E] =>} \inlinecode{Parser[E]}, where the type of the \emph{extended} Object Algebra interface appears in \emph{contravariant} position. This difference prohibits method/field overriding. Scala supports type-refinement of fields if the overriding
field has a type which is a \emph{subtype} of the original field.
Unfortunately, in this case, although \inlinecode{VarExprAlg} is a subtype
of \inlinecode{ExprAlg}, the function type \inlinecode{VarExprAlg[E] =>} \inlinecode{Parser[E]} is a supertype of \inlinecode{ExprAlg[E] =>} \inlinecode{Parser[E]}. Thus it is not possible to have overriding.

% because Object Algebra interfaces extended from \inlinecode{VarExprAlg} %are subtypes of it, but the types of their parsers are supertypes of %\inlinecode{VarExprAlg[E] =>} \inlinecode{Parser[E]}.

\paragraph{A Solution}
A solution to this problem is instead have a field with the Object Algebra
in \lstinline{ExprOAParser}. This approach supports overriding as well.
The parsing code is as follows:

\lstinputlisting[linerange=23-33]{../Scala/Parser/src/PaperCode/Sec4OA/Code3.scala}% APPLY:linerange=BASE_OA_PARSER
This is precisely the pattern that we advocate for modular parsing.
One important remark is we introduce \inlinecode{pE} for recursive calls.
The reason why we use an extra and seemingly redundant field, is due to a subtle issue caused by Scala language and its parser combinator library. There is a restriction of \inlinecode{super} keyword in Scala that \inlinecode{super} can only use fields defined by keyword \inlinecode{def}, can not use those defined by \inlinecode{val}, while the parser combinator library suggests using \inlinecode{val} to define parsers, especially for left-recursive ones. Our workaround is that we use different synonyms for \inlinecode{pE} in different traits, so that we can directly distinguish them by names without using \inlinecode{super}.

\paragraph{Extensions}
Now let's try this pattern with the variables extension:
\lstinputlisting[linerange=37-44]{../Scala/Parser/src/PaperCode/Sec4OA/Code3.scala}% APPLY:linerange=EXT_OA_PARSER

\noindent The type of the object algebra field \lstinline{alg} is first refined
to \lstinline{VarExpAlg[E]}, to allow calling the additional factory method
for variables. Note that, unlike the previous attempt, \lstinline{VarExpAlg[E]}
is a subtype of \lstinline{ExpAlg[E]}, thus the type-refinement of the
field is allowed. Now, the code for parsing variables (\lstinline{pVar}) can
call \lstinline{alg.varE}. The following code illustrates how to use
the parser from a client perspective:

\lstinputlisting[linerange=48-51]{../Scala/Parser/src/PaperCode/Sec4OA/Code3.scala}% APPLY:linerange=EXT_OA_PARSER_CLIENT

In the client code shown above, we first need to choose a concrete
object algebra to initialize the \lstinline{alg} field in \lstinline{VarExprOAParser}. Here we pick the
pretty-printing algebra \lstinline{VarExprPrint}, but any other object
algebra that implements \lstinline{VarExprAlg} would work.
With an instance of \lstinline{VarExprOAParser} in hand, we can call
\lstinline{pE} to obtain the parser to feed to the \lstinline{parse} method.
The result shows that such a pattern provides modular parsing.

Note that, similarly to the approach in Section~\ref{sec:inheritance} independent extensibility is also supported via multiple trait inheritance.
Since independent extensibility is achieved using essentially the same technique as in Section~\ref{sec:inheritance} we omit that code here.

\subsection{Discussion of Object Algebras}\label{subsec:more-oa}

In Section~\ref{subsec:objectalgebras}, \lstinline{makeExp} represents the AST of expression \inlinecode{1 + x} as a method. Parsers with Object Algebras generate such ``implicit objects'', which are actually functions of type \lstinline{Alg[E] => } \lstinline{E} for generic \lstinline{E}. Comparing with traditional objects, one may have concern that if a parser produces function structures, it could be hard to further process them. Especially, in the design of a compiler there will be a lot of desugarings (or AST transformations).

However, we argue that this is not an issue, because functions can be used
in a very clever way. Continuing with the previous example, suppose we want to transform an expression of \inlinecode{VarExprAlg} to an expression of \inlinecode{ExprAlg}, by
replacing some variables with values based on a variable environment. A \textit{transformation} algebra can be implemented as follows:

\lstinputlisting[linerange=45-56]{../Scala/Parser/src/PaperCode/Sec4OA/Code2.scala}% APPLY:linerange=OVERVIEW_OA_REFACTOR
Here the \inlinecode{VarExprAlg} expression \inlinecode{1 + x} is changed to \inlinecode{ExprAlg} expression \inlinecode{1 + 2}, before applying the pretty-printing algebra. The abstract field
\inlinecode{alg} in \inlinecode{Refactor} could be any algebra simply for delegation. With an algebra instance of \inlinecode{alg}, \inlinecode{Refactor} is able to transform a \inlinecode{VarExprAlg} expression before producing the result. Furthermore, transformations can be composed in a linear pipeline-style beforehand~\cite{Zhang2015}.

Another useful technique is the composition of \textit{query} algebras. A query traverses the abstract syntax tree, collects information and returns the result. The aforementioned \lstinline{Print} and \lstinline{Eval} algebras are traditional queries. But as we discussed before, an algebra is first fed to the parser, then \lstinline{parse} is invoked to do the parsing. It is tedious to fed \lstinline{Print} and \lstinline{Eval} separately, as the same input is parsed twice. Instead, we can firstly compose two queries into one~\cite{Oliveira:2012,oliveira2013feature}, before applying it for parsing.
