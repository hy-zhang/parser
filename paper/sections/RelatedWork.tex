\section{Related Work}\label{sec:relatedwork}

%- extensible parsing, language workbenches: rats, noa (this one already uses OA), modular semantic actions, (syntactic modularity, no separate compilation, modular type-checking)
%(read more papers, see if they talk about this issue, some potential solutions)
%(attribute grammars?)
%
%- parser combinators for type-checking, previous work has not shown how to support modularity (ASTs); left-recursion and back-tracking in related techniques
%
%- modularity: object algebras, dtc and mrm (problem with parsing, is there any related work? (PB: a paper on unfolds: build the AST))
%
%(parsing in Javascript: using delegation, does it support modular AST)
%
%noa, shy: shy: only override some interesting cases (transformation is tedious)
%bruijn indices: parsing + transformation

Our work touches upon several topics including extensible parsing,
parser combinators and extensibility techniques. However, as far as we
know there's no work that discusses how to do statically type-safe and 
separately compilable modular parsing.

\begin{comment}
There has been a
great amount of related papers on those topics. Some
inspired us of this paper and encourage us for more exploration. This
section will try to lead a discussion on what difference we have made.
\end{comment}

\paragraph{Syntactically Extensible Parsing} Extensible parsing is
used in many different areas. Extensible parser generators are a
mainstream area specially designed for modular syntax and
parsing. Many parser
generators~\cite{antlr1995,Grimm2006,Gouseti2014,Warth2016} support
modular grammars. They allow users to create new modules where new
non-terminals and production rules can be introduced, some can even
override existing rules in the old grammar modules. For instance,
\textit{Rats!}~\cite{Grimm2006} constructs its own module system for
the collection of grammars.  NOA~\cite{Gouseti2014} uses Java
annotation processing to gather information about grammars in Java
code. After all information is collected NOA merges all grammar
components, produce an ANTLR~\cite{} grammar, and generates
corresponding parsing code from the ANTLR grammar.  Those parser
generators focus on the \textit{syntactic extensibility} of grammars:
they rely on whole compilation to generate a global parser, even if
there is only a slight modification in the grammar. Some of those
parser generators may statically check the correctness and unambiguity
of grammars. In constrast, because our approach is based on parser
combinators, there is no support for ambiguity checking.  However, as
far as we are aware, no extensible parser generators support separate
compilation or modular type-checking.

Macro systems like the C preprocessor, C++ templates and
Racket~\cite{Tobin-Hochstadt2011}, and other meta-programming
techniques are a similar area aiming at syntactic extensibility.
SugarJ~\cite{Erdweg2011} conveniently introduces syntactic sugar for
Java using library imports. Composition of syntactic sugar is easy for
users, but it requires many rounds of parsing and adaption. This
significantly affects efficiency of compilation. Since the
implementation was based on SDF~\cite{Heering1989} and
Stratego~\cite{Visser2001}, the approach does not support separate
compilation. Racket adopts a macro system for library-based language
extensibility~\cite{Tobin-Hochstadt2011}. This is quite powerful and it uses 
attributed ASTs for contextual
information. Moreover extensions can be integrated in a modular
way. However such modularity is not flexible enough for language
unification, as the syntax is only built from extensions.  Moreover,
existing macros cannot be further changed after
definition. \haoyuan{correct?}  Extensible
compilers like JastAdd~\cite{Ekman2007} and
Polyglot~\cite{Nystrom2003} also support extensible parsing, but this
is mostly done using parser generators. They focus on the
extensions to a host language. Our library is designed for type-safe 
modular language parsers (not just language 
extensions), with very flexible parser composition.

%\bruno{Have you read
%Languages as Libraries? That is probably an important reference, which
%I think we should cite.}
%\haoyuan{what about
%  metafront? it is a macro system but does it have type safety?}
%\haoyuan{Extensible syntax with lexical scoping?} \haoyuan{"Extensible
%  syntax" proposes a system for extensible syntax, where users write
%  EDSLs in their language with concrete syntax. Users can write rules
%  for type-based disambiguation. But separate compilation is again not
%  mentioned. Shall we mention that thesis?} \haoyuan{attribute
%  grammars?}

\paragraph{Extensible Parsing Algorithms}
Extensible parsing algorithms are another area that relates to
separate compilation. 
\textit{Parse table composition}~\cite{Bravenboer2009} 
is an approach where grammars are compiled to
modular parse tables. Those parse tables are expressed as DFAs
or NFAs, and later they can be composed by an algorithm, to provide
separate compilation for parsing. The generation of parse tables can
be quite expensive in terms of performance. The approach 
is quite different from ours, since it uses parse
 tables, wheres we use parser combinators. 
Our approach supports both
\emph{separate compilation} as well as \emph{modular
  type-checking}. Moreover, the extensibility of parsing is further
available at language composition and lexical level.

%\haoyuan{we do not have explicit
%  correspondence/relationship between abstract syntax and the parser.}

\paragraph{Parser Combinators} Since Burge~\cite{burge1975} firstly
introduced parser generators, and after Wadler popularized them 
in functional programming~\cite{Wadler1985}, parser combinators have been
more and more popularly developed and used in the research area of
parsing. Many parsing libraries produce recursive descent
parsers by introducing functional monadic
parser combinators~\cite{nott237}. Parsec~\cite{Leijen2001} is 
perhaps the most popular parser combinator library in this line. 
It is widely used in Haskell (with various ``clones'' in other languages)
for context-sensitive grammars with infinite lookahead. Nevertheless,
Parsec users suffer from manual left-recursion elimination, 
high cost for backtracking and longest match composition issues, 
as we discussed in Section~\ref{}. Those limitations make Parsec 
(and similar parsing techniques) inadequate for modular parsing.

Some more recent work on parser
combinators~\cite{Ford2002,Might2011,Frost2008} proposed a series of
novel parsing techniques that address the issue of
left-recursion. We have selected Packrat parsing for our approach, 
as the combinators are convenient to use in Scala, and 
integrate well with multiple trait inheritance and
Object Algebras. 
%Also,
%its support for direct left-recursion is already helpful for designing
%real-world parsers practically, though~\cite{warth2008} further
%demonstrated that general left-recursion is supported theoretically.
Despite our choice of Packrat parsing, as argued in Section~\ref{}, 
any parser combinator approach which supports left-recursion, longest 
match composition, and makes the cost of backtracking reasonable,
should be applicable to modular parsing.
%Yet we have mentioned that the components used in our approach can
%have alternatives to achieve better modular parsing.

\paragraph{Extensibility} Various design patterns~\cite{}, in multiple
languages, have been proposed over the years to address extensibility
problems, such as the Expression Problem~\cite{}. However, to the best
of our knowledge, none of the existing extensibility techniques
address the problem of \emph{modularly 
parsing extensible ASTs}.  There has been
a lot of work on extensible ASTs, in both functional and
object-oriented programming.  For instance, the famous ``Datatypes \`a
la Carte'' (DTC)
paper~\cite{swierstra2008} represents modular ASTs using co-products
of every two functors, which forms a binary tree structure. 
Several variants of DTC have been later proposed~\cite{Bahr2011,Bahr2014,Oliveira2015}
All of that work essentially covers how to traverse and 
consume extensible ASTs. 
Only in Bahr's~\cite{Bahr2011} work a brief mention of \emph{unfolds} 
(which are a class of algorithms that \emph{build} ASTs) are mentioned. 
However Bahr's work does not cover parsing.

In OO languages, there are also many design patterns that achieve 
type-safe extensibility~\cite{torgerson,odersky,Oliveira09,Oliveira12,Wang16}. 
We chose Object Algebras~\cite{Oliveira2012} because the pattern is 
relatively lightweight and makes good use of existing OO features,
such as inheritance, generics and subtyping. As seen throughout the paper, 
the parsing code is concise and expressive using Object Algebras.
In principle our approach is not coupled to the use of Object
Algebras, and variants of our approach could be used with other 
extensibility techniques. Similarly to the work in functional
programming, the extensibility techniques in OOP have focused 
on traversing and processing ASTs. We are unware of any work on OOP
that covers how to do modular parsing for extensible ASTs.

It is worth mentioning that Scala case classes~\cite{} provide a near
solution to the Expression Problem. Case classes are similar to
algebraic datatypes but, unlike algebraic datatypes, new cases can be
modularly added. The reason why case classes are not a complete
solution to the Expression Problem is that they do not enforce
exhaustiveness of pattern matching for extensible operations. In other
words run-time pattern matching errors can happen when writting
extensible code with case classes.  So full static type-safety is not
ensured. Nevertheless case classes are a pragmatic approach, which is
widely used in practice. Therefore modular parsing techniques may be
of value for extensible code using case classes.  The approach we
presented in Section 3~\ref{} can readily be adapted to case classes:
all that the users need to do is to use case classes, instead of
standard OO classes in their code.

\begin{comment}
Moreover, in~\cite{Oliveira2012} the authors have discussed the
composition of algebras. In our parsing approach, a parser consumes an
algebra, which is delegated to return the results, during its process
of parsing. Having a set of algebras, it requires multiple parsing
with several times of invocation, which leads to redundant work.
Instead, algebras are supposed to be composed into one before the
invocation of the parser. Bahr et al. lead a similar discussion
in~\cite{Bahr2011}, where queries (or \textit{catamorphisms}) and
transformations (or \textit{homomorphisms}) are composable. They have
also mentioned the dual process of folds, namely
\textit{anamorphisms}. It is potentially related to our work, as
parsing is a representative kind of unfolds, whereas they only
discussed the composition of a cv-coalgebra and a term homomorphism,
which differs from modular parsing.
\end{comment}

\bruno{There's a problem with the text in 4.2. We need to first 
fix the problem in S4.2 and then come back to the text below and 
adapt it accordingly.}
Another interesting observation, as we have mentioned
in~\ref{subsec:objectalgebras}, is that transformation algebras are
frequently used in parsing, as concrete syntax may not always coincide
with its abstract syntax, for instance, the bruijn indices in lambda
calculus, as well as other syntactic sugars. In the Shy
framework~\cite{Zhang2015}, the pattern of transformations is captured
into a template for Java annotation processing, and most of
boilerplate code is automatically generated, hence users only need to
override a few interesting cases, conveniently. It is potentially
useful in our case. The idea of overriding existing cases was also
proposed in MRM. Our approach allows existing parsing code to be
overridden during inheritance.
