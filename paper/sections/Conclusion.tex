\section{Conclusion}\label{sec:conclusion}

This paper presents a solution for type-safe modular parsing. Our solution 
not only enables parsers to evolve together with the abstract syntax, 
but also allows parsing code to be modularly type-checked and separately compiled.

Based on parser combinators, we identify algorithmic
challenges of building modular parsers, and show that Packrat parsing
is suitable as the underlying parsing technique. Our solution does not
require advanced language features. We show that with standard OO
techniques including inheritance and overriding, it is practical to
build modular parsers for OO ASTs. However, the extensibility issue of
traditional OO ASTs motivates us to adopt Object Algebras for full
extensibility and more useful features.

Abstracting language features as reuseable components can be achieved
based on our solution, which is useful for rapid language prototyping.
The TAPL case study shows that we can obtain considerable code reuse
(69\% shorter) compared with a non-modular implementation, using our
modular parsing approach and language feature abstraction.

There are certainly some aspects that can be improved. We observed that the
glue code of composition appears to be boilerplate. Such an issue
is addressed by family polymorphism~\cite{ernst01FP}, and we could solve it by language
features or meta-programming techniques. Moreover, we can possibly
adopt the Shy framework~\cite{Zhang2015} and algebra composition
patterns~\cite{oliveira2013feature}, to improve the usage of Object
Algebras.

For future work, we want to experiment more with semantically modular
parsing techniques for functional programming. Since functional
programming languages like Haskell and ML do not support OO features 
such as subtyping or inheritance, it will be interesting to see
whether other functional features can be used instead. 
Open recursion~\cite{CookThesis} seems to be an alternative to inheritance, and may 
help in doing extensible parsing in functional languages. 
%It is also interesting to see that parsing, to some
%extent, can be viewed as a special example of \textit{unfolds}. So it
%is worthwhile considering to generalize our approach under certain
%circumstances.
