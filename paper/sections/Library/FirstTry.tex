\subsection{First Try on Parsing}\label{subsec:firsttry}

Our story begins with a single parser that only parses a free variable (i.e., a string). In Scala we use the Packrat Parsing
library, which supports lexing and direct left-recursion. To write such a parser, we need to import the library as
well as defining the datatype for which it produces.

\begin{lstlisting}
object OpenRecursion extends StandardTokenParsers with PackratParsers {
    class Expr
    class Var(x : String) extends Expr
    val pVar : PackratParser[Expr] = ident ^^ { x => new Var(x) }
    ...
}
\end{lstlisting}
Note that we are omitting lexing and demo code here. Furthermore, we would like the parser to support applications.
Not only the implementation of that parser is needed, but also its corresponding datatype: (we omit the fact that the code for below few
examples is written in the same object, namely \lstinline{OpenRecursion})
\begin{lstlisting}
class App(e1 : Expr, e2 : Expr) extends Expr
val pApp : PackratParser[Expr] = pVar ~ pVar ^^ { case e1 ~ e2 => new App(e1, e2) }
\end{lstlisting}
Such a parser \lstinline{pApp} can only parse one-layer applications like \lstinline{"x y"}. Hence we revise it using recursion:
\begin{lstlisting}
val pApp : PackratParser[Expr] =
    (pApp | pVar) ~ (pApp | pVar) ^^ { case e1 ~ e2 => new App(e1, e2) }
\end{lstlisting}
Now \lstinline{pApp} manages to parse an arbitrary number of applications like \lstinline{"x y z ..."} in a right-associative way.
Nevertheless, such an approach does not turn out to be a modular extension. The grammar we want for an expression is actually:
\begin{lstlisting}
e ::= x | e ' ' e | ...
\end{lstlisting}
Note that such an \lstinline{"e"} in the second case should not only include all existing cases, but also take all future extensions into account (the ellipsis we used above, for instance, \lstinline{e '+'} \lstinline{e}). This inspires us to use \textit{open recursion}. To define \lstinline{pApp} in a different way, we add a parameter \lstinline{p} to it, representing the explicit ``self reference'' of our whole parser, which is open to future extensions. Note that \lstinline{p} should be defined as a by-name parameter using \lstinline{"=>"}.
\begin{lstlisting}
val pApp : (=> PackratParser[Expr]) => PackratParser[Expr] =
    p => p ~ p ^^ { case e1 ~ e2 => new App(e1, e2) }
\end{lstlisting}
Similarly we redefine \lstinline{pVar} also as a function using the fixpoint \lstinline{p}. Furthermore, we combine \lstinline{pVar} and \lstinline{pApp} together using the parser combinator \lstinline{"|"} for alternative:
 \begin{lstlisting}
val pVar : (=> PackratParser[Expr]) => PackratParser[Expr] =
    p => ident ^^ { x => new Var(x) }
val pVarApp : (=> PackratParser[Expr]) => PackratParser[Expr] =
    p => pApp(p) | pVar(p)
\end{lstlisting}
But what we really want is something that has type \lstinline{PackratParser[Expr]}. In functional programming, we can use ``fix'' to get the fixpoint of a function, in this case we write \lstinline{fix(pVarApp)}, which parses an arbitrary number of applications. Such a function can easily be defined
in Scala based on its laziness:
\begin{lstlisting}
def fix[A](f : (=> A) => A) : A = { lazy val a : A = f(a); a }
\end{lstlisting}
That is the magic of open recursion: we can define as many small components as we like, and they are implemented as functions, with the help of explicit self-reference. Whenever we would like to close the recursion, simply use \lstinline{fix} for their combination. Such a process is guaranteed to terminate when we restrict the input for parsing to be finite, and we have some base cases like \lstinline{pVar}, which terminates recursion in their branches, in addition, with the help of direct left-recursion support from Packrat parsers.

Another thing which catches our attention is that, small parsers are combined using the alternative combinator. It is a matter of delegation: parsing failing at any position will try its next alternative, and for each recursion all the parsers are tried successively until one of them succeeds. It reveals its essence as a recursive descent parser. But it is quite easy and modular to use; with further extensions, they can be implemented separately and then appended to the combination of old ones.
