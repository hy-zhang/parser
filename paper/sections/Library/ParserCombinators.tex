\subsection{Parser Combinators}\label{subsec:parsercombinators}

Using Packrat parsers we obtain the access to a lot of library functions, including parser combinators. In the last subsection we introduced \lstinline{|} for alternation, which realizes delegation pattern in our parsing. Moreover, we have \lstinline{~} for sequential composition, \lstinline{rep} and \lstinline{repsep} for repetitions, \lstinline{chainl1} for left-associative grammars, and so on.

Specifically, Scala parsers have a new combinator \lstinline{|||} also for alternative but with longest match composition. Consider the example in the last subsection, where we used \lstinline{pApp(p)} \lstinline{|} \lstinline{pVar(p)} for alternation. If \lstinline{pVar} is put in the front instead, the parser cannot even parse \lstinline{"x y"}, since the first case succeeds, which unexpectedly stops parsing and discards the rest. But \lstinline{|||} can successfully parse the whole input string. It implies that using \lstinline{|||} we no longer need to worry about the order of composition, which encourages its extensibility.

On the other hand, practically parsers are defined as functions (as shown above), which means we do not apply \lstinline{|||} to them directly, but use \lstinline{|||} implement a new combinator for those functions. We will introduce it later in this paper. 