%\documentclass[preprint]{sigplanconf}
%\documentclass[preprint]{llncs}
\documentclass[a4paper,UKenglish]{lipics}

\usepackage{microtype}%if unwanted, comment out or use option "draft"
\usepackage[table,xcdraw]{xcolor}
\usepackage{color}
\usepackage{amsmath}
\usepackage{stmaryrd}
\usepackage{graphicx}
\usepackage{amssymb}
\usepackage{fancyvrb}
\usepackage{url}
\usepackage{pstricks,pst-node,pst-tree}
\usepackage{theorem}
%% \usepackage{mathpartir}
\usepackage{bbm}
\usepackage{pgf}
\usepackage{multirow}

\usepackage{listings}
\usepackage{verbatim}
\usepackage{graphicx}
\usepackage{wrapfig}

\usepackage[T1]{fontenc}
\usepackage[scaled=0.85]{beramono}
\usepackage{mathpartir}

% "define" code highlights for Java and Scala
\lstdefinelanguage{JavaScala}{
  morekeywords={public,int,interface,implements,default,
    abstract,case,catch,class,def,static,%
    do,else,extends,false,final,finally,%
    for,if,implicit,import,match,mixin,%
    new,null,object,override,package,%
    private,protected,requires,return,sealed,%
    super,this,throw,trait,true,try,%
    type,var,while,yield,with,val},
  otherkeywords={=>,<-,<\%,<:,>:,\#,@},
  sensitive=true,
  morecomment=[l]{//},
  morecomment=[n]{/*}{*/},
  morestring=[b]",
  morestring=[b]',
  morestring=[b]"""
}

\lstdefinelanguage{PlainCode}{}

\lstset{ %
language=JavaScala,                % choose the language of the code
columns=flexible,
lineskip=-1pt,
basicstyle=\ttfamily\small,       % the size of the fonts that are used for the code
numbers=none,                   % where to put the line-numbers
numberstyle=\ttfamily\tiny,      % the size of the fonts that are used for the line-numbers
stepnumber=1,                   % the step between two line-numbers. If it's 1 each line will be numbered
numbersep=5pt,                  % how far the line-numbers are from the code
backgroundcolor=\color{white},  % choose the background color. You must add \usepackage{color}
showspaces=false,               % show spaces adding particular underscores
showstringspaces=false,         % underline spaces within strings
showtabs=false,                 % show tabs within strings adding particular underscores
morekeywords={var},
%  frame=single,                   % adds a frame around the code
tabsize=2,                  % sets default tabsize to 2 spaces
captionpos=none,                   % sets the caption-position to bottom
breaklines=true,                % sets automatic line breaking
breakatwhitespace=false,        % sets if automatic breaks should only happen at whitespace
title=\lstname,                 % show the filename of files included with \lstinputlisting; also try caption instead of title
escapeinside={(**}{**)},          % if you want to add a comment within your code
keywordstyle=\ttfamily\bfseries,
% commentstyle=\color{Gray},
% stringstyle=\color{Green}
}


%%\usepackage{natbib}
%%\bibpunct();A{},
%%\let\cite=\citep

%include lhs2TeX.fmt
%include lhs2TeX.sty
%include forall.fmt

%\pagestyle{plain}

%{\theorembodyfont{\sffamily} \newtheorem{theorem}{Theorem}}
%{\theorembodyfont{\sffamily} \newtheorem{lemma}{Lemma}}
%\newtheorem{theorem}{Theorem}
%\newtheorem{lemma}{Lemma}
%\newenvironment{proof}{\textbf{Proof:\hspace{4mm}}}{$\Box$}
\newcommand{\authornote}[3]{{\color{#2} {\sc #1}: #3}}
\newcommand\bruno[1]{\authornote{bruno}{red}{#1}}
\newcommand\yanlin[1]{\authornote{yanlin}{purple}{#1}}
%\newcommand{\hl}[1]{\textcolor{red}{#1}}
\newcommand\huang[1]{\authornote{huang}{blue}{#1}}
\newcommand\haoyuan[1]{\authornote{haoyuan}{cyan}{#1}}

\newcommand\sem[1]{\llbracket #1 \rrbracket_r}
\newcommand\sems[1]{\llbracket #1 \rrbracket_s}
\newcommand\tsem[1]{\llbracket #1 \rrbracket}
\newcommand{\rbm}[1]{\raisebox{-2.0ex}[0.5ex]{#1}}
\newcommand\nat[0]{\mathbb{N}}
\newcommand\unit[0]{\mathbbm{1}}

\newcommand\mixin{\textbf{@Mixin} }
\usepackage{xspace}

% Author macros::begin %%%%%%%%%%%%%%%%%%%%%%%%%%%%%%%%%%%%%%%%%%%%%%%%
\title{Type-Safe Modular Parsing}
\titlerunning{Type-Safe Modular Parsing}

\author[1]{John Q. Open}
\author[2]{Joan R. Access}
\affil[1]{Dummy University Computing Laboratory\\
  Address, Country\\
  \texttt{open@dummyuni.org}}
\affil[2]{Department of Informatics, Dummy College\\
  Address, Country\\
  \texttt{access@dummycollege.org}}
\authorrunning{J.\,Q. Open and J.\,R. Access}
% mandatory. First: Use abbreviated first/middle names. Second (only in severe
% cases): Use first author plus 'et. al.'

\Copyright{John Q. Open and Joan R. Access}
% mandatory, please use full first names. LIPIcs license is "CC-BY";
% http://creativecommons.org/licenses/by/3.0/

\subjclass{Dummy classification -- please refer to
  \url{http://www.acm.org/about/class/ccs98-html}}
% mandatory: Please choose ACM 1998 classifications from
% http://www.acm.org/about/class/ccs98-html . E.g., cite as "F.1.1 Models of
% Computation".
\keywords{Dummy keyword -- please provide 1--5 keywords}% mandatory: Please provide 1-5 keywords
% Author macros::end %%%%%%%%%%%%%%%%%%%%%%%%%%%%%%%%%%%%%%%%%%%%%%%%%

%Editor-only macros:: begin (do not touch as author)%%%%%%%%%%%%%%%%%%%%%%%%%%%%%%%%%%
\serieslogo{}%please provide filename (without suffix)
\volumeinfo%(easychair interface)
  {Billy Editor and Bill Editors}% editors
  {2}% number of editors: 1, 2, ....
  {Conference title on which this volume is based on}% event
  {1}% volume
  {1}% issue
  {1}% starting page number
\EventShortName{}
\DOI{10.4230/LIPIcs.xxx.yyy.p}% to be completed by the volume editor
% Editor-only macros::end %%%%%%%%%%%%%%%%%%%%%%%%%%%%%%%%%%%%%%%%%%%%%%%

%%%%%%%%%%%%%%%%%%%%%%%%%%%%%%%%%%%%%%%%%%%%%%%%%%%%%%%%%%%%%%%%%%%%%%%%%%%%%%%%
\begin{document}
\maketitle

\begin{abstract}

Text.

\end{abstract}

%if False

%\category{D.3.2}{Programming Languages}
%                {Language Classifications}
%                [Functional Languages]
%\category{F.3.3}{Logics and Meanings of Programs}
%                {Studies of Program Constructs}
%                []

%\terms
%Languages
%
%\keywords
%Mixins, explicit effects, monads, aspect-oriented programming, parametricity,
%interference

%endif

%===============================================================================

\section{Introduction}\label{sec:introduction}
 
The quest for improved modularity, variability and extensibility of
programs has been going on since the early days of Software
Engineering~\cite{}. Modern Programming Languages enable a certain
degree of modularity, but they have limitations as illustrated by
well-known problems such as the Expression Problem~\cite{}. The
Expression Problem refers to the difficulty of writing data
abstractions that can be easily extended with both new operations and
new data variants. Traditionally the kinds of data abstraction found
in functional languages can be extended with new operations, but
adding new data variants is difficult. The traditional object-oriented
approach to data abstraction facilitates adding new data variants
(classes), while adding new operations is more difficult.

A reason why a solution to the Expression Problem is important in
practice is that it is necessary for the development of
\emph{Software-Product Lines} (SPLs)~\cite{}. A software-product line
is a reusable set of components, which can be combined in multiple ways
to obtain different programs. Programming languages offer a concrete
example for SPLs. A SPL for programming languages would allow us to
model various typical operations of programming languages (such as
evaluation, compilation, or parsing) for various different language
constructs (such as binding, arithmetic, conditional or loops)
independently and separately. For example, evaluation components could be defined
independently for binding and arithmetic constructs. If the language
to be implemented is the pure lambda calculus, only evaluation of
binding constructs is necessary. However, more realistic programming
languages will include arithmetic constructs, and will require 
evaluation for such constructs as well. In this case 
both the component for evaluation of binders and arithmetic 
expressions can be combined to implement the desired functionality.

\begin{comment}
Most programming languages share alot of features in
common. 

For example, most languages have language constructs for:
binding (such as variables, functions, and function applications);
basic arithmetic operations; basic logic and conditional operations;
loops; as well as various other features. For each language construct,
various operations (such as evaluation, compilation, or parsing) need
to be implemented. It is reasonable to wonder whether we can simply
implement those features independently of a particular implementation
of a programming language. Evaluation could be defined independently 
for binding and arithmetic constructs. If the language to be
implemented is the pure lambda calculus, only evaluation of binding 
constructs is necessary. Thus only the component that implements 
evaluation for binding needs to be used in such an implementation.
However, more realistic programming languages 
will include arithmetic constructs, and will require an evaluation
function for those. 


Then it would be possible to \emph{reuse}
some of those features in \emph{multiple} different implementations of
programming languages. Essentially, this would enable a SPL for
programming languages, where all

A solution to the Expression Problem could ena



A concrete 
example that illustrates this issue is 
\end{comment}

To address the modularity limitations of Programming Languages several
different approaches have been proposed in the past. Existing
approaches can be broadly divided into two categories:
\emph{syntactic} or \emph{semantic} modularization
techniques. Syntactic modularization techniques are quite popular in
practice. Examples include many tools for developing Software-Product
Lines~\cite{}, some Language Workbenches~\cite{}, or extensible parser
generators~\cite{}.  Most syntactic approaches employ textual
composition techniques such as \emph{superimposition}~\cite{} to
enable the development modular program features. Such textual
composition techniques collect the code for multiple features and
merge it together when a concrete combination of features is needed
for a particular program. As Kastner~\cite{} notes, the typical
drawback of such techniques is that
``\emph{most feature-oriented implementation mechanisms lack proper
  interfaces and support neither modular type checking nor separate
  compilation}''\bruno{reference to ``The road to
  Feature Modularity''}. Syntactic modularization techniques have also
been applied to the problem of \emph{extensible parsing}. There are
several approaches~\cite{} that enable the development of
\emph{syntactically} modular parsers or grammars. However these
approaches do not support separate compilation or modular
type-checking either.\bruno{mention that a reason why syntactic 
modularization techniques are popular is simplicity (in implementation
and also use).}

Semantic modularization techniques move away
from syntactic composition techniques that rely on the textual source
code. This allows us to go one step further in terms of modularity,
and also enable components or features to be modularly type-checked
and separately compiled. Modular type-checking and separate
compilation are desirable properties to have from a Software
Engineering point-of-view, and enable the composition of compiled
binaries as well as ensuring the type-safety of the code composed of
multiple components. Examples of semantic modularization techniques 
include various approaches to \emph{family polymorphism}~\cite{}, as 
well as several techniques for solving the Expression Problem. 
\bruno{strenghten? mention dynamic languages (separate compilation 
but no modular type-checking), and techniques that weaken type-safety 
requirements.} Most semantic
modularization techniques have focused on operations that traverse or 
or process extensible datastructures, such as ASTs.
\bruno{Object Algebras here?} However, as far as
we know there is little work on operations that build/produce ASTs. 
In particular the problem of modular parsing has not been studied in
semantic modularization approaches. This is a shame because, to
realise the vision of software product-lines for programming
languages, modular parsing is a necessity. 

\paragraph{Modular and Extensible Parsing}
  This paper investigates presents \name: a parsing
  combinator library that enables modular parsing.
  \name provides a solution for the problem of \emph{semantic modular
    parsing}. That is, the solutions should not only
  allow complete parsers to be built out of modular parsing
  components, but also enable the parsing components to be \emph{modularly
  type-checked} and \emph{separately compiled}. \name is
  built on top of a Packrat parsing library, but adds new parsing
  combinators to enable modular parsing. The new parsing combinators 
  employ \emph{delegation-based} techniques and \emph{Object Algebras} 
  to support extensibility. The choice of Packrat parsing over other
  parsing techniques turns out to be important for achieving
  performance in a modular setting. \bruno{left recursion, and
    backtracking removal.} 

  By analising the \emph{full} grammar it is possible to remove
  backtracking, which would otherwise increase parsing times. Many
  parsing combinator libraries routinely use backtracting elimination
  to achieve performance. However, in a modular setting this technique
  cannot be used, because the full grammar is not known. Thus we have
  to be very conservative at eliminating backtracting. Unfortunatelly,
  this has a severe impact on performance.

  To evaluate \name we conduct a case study based on the ``Types and
  Programming Languages'' (TAPL) interpreters.  The case study shows
  that \name is effective at reusing parsing code from existing
  interpreters, and the total parsing code is 60\% shorter than the
  non-modular parsing code.\bruno{comment on the efficiency}

In summary our contributions are:

\begin{itemize}

\item {\name:} A parser combinator library that allows the development 
of modular parser. The library uses \emph{delegation} and
\emph{Object-Algebras} to achieve modularity and extensibility.

\item {{\bf A Parsing Technique for OO ASTs:}} A simplified version of
  our technique also enables parsing OO-style ASTs, where new language
  constructs can be easily added.

\item {{\bf Limitations of existing Parser Combinator Techniques:}}
\bruno{Improve and write something here.}

\item {{\bf TAPL case study:}} We conduct a case study with 18 interpreters
  from the TAPL book. The case study show the effectiveness of modular 
  parsing in terms of reuse.

\end{itemize}


\haoyuan{I suggest this part to be moved to Introduction, and only discuss why Packrat is selected among
different parser combinators in Overview.}

\begin{comment}
Although there are many parsing techniques, not all of them are
suitable for type-safe modular parsing. In particular there are many
techniques which fail to provide modular type-checking and separate
compilation. Moreover, even if modular type-checking and separate
compilation are supported, efficiency is another
concern. A parsing technique should have low overhead when applied
in a modular setting. In the remaider of this section, we eliminate
various techniques that fail to satisfy our requirements, and argue
that that Packrat parsing~\cite{Ford2002} is a suitable candidate for
type-safe modular parsing.

\paragraph{Parser Generators} The most widely use tools for parsing
are parser generators. Parser generators help users generate parsers automatically or
semi-automatically from a given grammar. There is no restriction on
the algorithm, while most of them adopts table-based LL~\cite{lewis1968syntax} and LR~\cite{knuth1965translation} parsing
algorithms.
Although efficient, the main drawback of parser generators is that they do not support
modular type-checking and separate compilation.

Modular parsing based on parser generators is supported by many libraries~\cite{antlr1995,Grimm2006,Gouseti2014,Warth2016}. Users can separate the syntax definition and related parsing code into reusable components. Then the corresponding parsers are built by their library utilities. For example, NOA~\cite{Gouseti2014} uses Java annotation processing to collect grammar information, and then generates ANTLR4 parsers. However, such generation procedure requires a whole compilation after the collection of all grammar pieces. Once the grammar changed, even slightly, grammar information must be re-collected and the parser must be re-generated. Hence those libraries only have syntactic modularity.

Generating parsers often requires full information of grammars, thus semantic modularity is difficult to achieve in this way.

\paragraph{Parser Combinators}
Comparing with the parser generators, a \textit{parser combinator}~\cite{burge1975,Wadler1985}
takes several parsers and produce a new parser as its output. Parser combinators are
popular in functional programming, where the parsers are represented
by functions and parser combinators are higher-order functions accepts
them.

At a first look, parser combinators are very suitable for our purpose, because of two
reasons. Firstly, they are naturally modular. The manner of using them
is to write small parsers and use combinators to composed them
together. The construction procedure is explicit and fully controlled
by the programmer. Secondly, each parser combinator is represented by
a piece of code, and also are the parsers it takes. Thus in a
statically typed programming language they can be statically
type-checked.
\end{comment}

\section{Overview}\label{sec:overview}

Our modular parsing library consists of four parts, and we will go through them in this section. The first is underlying parsing technique we used. Parsing has been heavily studied over the years, especially for context-free grammars and its subsets. However, the choice of parsing techniques is not a minor issue here, because of modularity. To achieve modularity, the parsers must be written in a way that can be composed, and different parsing techniques have different suitability for that.

Although every parsing algorithm is able to implement by hand in principle, some of them are not friendly for that. A parser generator is a tool to generate parsers automatically or semi-automatically, from a grammar given by the user. There is no restriction on the algorithm, however most of them adopts LL and LR parsing algorithms, which are table-based.

\huang{TALK ABOUT PREVIOUS WORKS USING PARSER GENERATORS, AND ALSO THE DISADVTANGES OF THEM}

Comparing with the parser generators, a parser combinator takes several parsers and produce a new parser as its output. It is popular in functional programming, where the parsers are represented by functions and parser combinators are higher-order functions accepts them. Parser combinators are very suitable for our purpose, because of two reasons. Firstly, they are naturally modular. The manner of using them is to write small parsers and use combinators to composed them together. The construction procedure is explicit and fully controlled by the programmer. Secondly, each parser combinator is represented by a piece of code, and also are the parsers it takes. Thus in a statically typed programming language they can be type-checked during compile time.

There are variants in the category of parser combinators. Almost all of them adopt top-down, recursive descent parsing, which uses backtracking to search through the possible branches. Simple backtracking parser combinators, such as the famous library Parsec in Haskell, have some drawbacks. One is that they cannot support left-recursive grammars. The common solution is to rewrite a left-recursive grammar into an equivalent one, so called left-recursion elimination. This solution requires information of the whole grammar, so that it cannot be applied in a modular setting because we only have parts of the grammar. Another drawback is performance. Take Parsec as an example, its choice combinator only tries the second alternative if the first fails without any token consumption. If two alternatives have a same non-empty prefix, an auxiliary function \lstinline{try} must be added to backtrack. However, if we want to extend our parsers later, we should always consider the worst case in which all alternatives share common prefixes. Then we need to add \lstinline{try} for all the branches, which results worst case exponential time complexity.

Beyond the simple backtracking one, several advanced parser combinators were studied to resolve those issues. One of them is Packrat parsers, which is our choice. The details of Packrat parser are beyond the scope of this paper. In brief, Packrat parsers use memorization to record the result of applying each parser at each position of the input, so that repeated computation is eliminated, and it supports left-recursive grammars. These properties are very suitable for modularity. Thus we use Packrat parsers as the underlying parsing technique in our library.

It is worth mentioning that the choice of parser combinators will not affect the other parts of our library. One can choose other parser combinators like Parsec, in cases that the performance and supporting of left-recursion are not major concerns.

The second part of our library is open recursion. It is introduced to deal with the problem of recursive calls in modular parsing. We use a simple example to demonstrate the problem, which is parsing a language of natural numbers. The language is defined below.

\begin{lstlisting}[language=PlainCode]
Expr := zero | succ Expr
\end{lstlisting}

Firstly, we need to define the abstract syntax. Here we use Scala's case classes to save spaces, this can also be done using class inheritance.

\begin{lstlisting}
abstract class Expr
case class Zero extends Expr
case class Succ(v: Expr) extends Expr
\end{lstlisting}

And then we write parsing functions for every cases. Some irrelevant details in the code are omitted to outline the main structure. Suppose we have \lstinline{choice} combinator for alternatives and the input tokens are global values which can be accessed by all these functions.

\begin{lstlisting}
def parseExpr(): Expr = choice(parserZero, parseSucc)

def parserZero(): Expr = {...}

def parseSucc(): Expr = {
  parseToken('succ')
  val v = parseExpr()
  Succ(v)
}
\end{lstlisting}

It works well, but later we want to extend the language to support addition.

\begin{lstlisting}[language=PlainCode]
Expr := zero | succ Expr | add Expr Expr
\end{lstlisting}

Extending the abstract syntax is easy.

\begin{lstlisting}
case class Add(a: Expr, b: Expr) extends Expr
\end{lstlisting}



\begin{itemize}
\item Choosing the Parsing Technology
    \begin{itemize}
    \item Parser Generators : why not?
        \begin{itemize}
            \item Not type-safe
            \item No modular type-checking
            \item Not modular or no separate compilation (but we need to mention lots of work on extensible parsing here: example Language Workbenches; Rats)
        \end{itemize}
    \item Parser Combinators
        \begin{itemize}
        \item Backtracking parsers (Parsec)
            \begin{itemize}
            \item Need to remove left recursion (Problem: Transformation is not modular; if we do not know the full grammar then cannot be done)
            \item Need try/backtracking (Problem: Since we do not know the full set of rules in a modular setting, we have to assume worst case scenario and add redundant backtracking. )
            \item Mention some possible workarounds (there may be some but they still have issues)
            \end{itemize}
        \item (Non)-Backtracking/Packrat Parsers (Works well in a modular setting) (find a good name for this type of parsers?)
        \end{itemize}
    \end{itemize}
\item Adapting Parser Combinators for modularity
    \begin{itemize}
    \item Using Object Algebras
    \item Using Open Recursion
    \item Using new modular parser combinators (Library: new alternative combinator, for example)
    \end{itemize}
\item Small example
    \begin{itemize}
    \item small example of a modular parser using our technique
    \item show also the same example without modularity and write a detailed comparison
    \item Lambda Calculus is a good candidate for the example
    \item Show a small extension (adding plus and numeric literals)
    \end{itemize}
\end{itemize}


\section{Background: Scala Packrat Parsing Library}\label{sec:packratparsers}

\bruno{We need to think whether this section should be incoorporated
  in S2}

This section introduces the Scala Packrat Parsing library~\cite{}, which is
used by \name.

\paragraph{An Example}
A Packrat parser has type \lstinline{PackratParser[E]} for some
\lstinline{E}, which indicates the representation\bruno{the
  representation of what?}. The code below illustrates
Scala's Packrat parsing library. 

\begin{lstlisting}
lexical.reserved += ("str")
lexical.delimiters += ("(", ")")

val p : PackratParser[String] =
    "str" ~> ("(" ~> numericLit <~ ")") ^^ { x => x.toString }
\end{lstlisting}

\noindent
 
Note that the code should be defined in an enclosing type that extends
\lstinline{scala.util.parsing.combinator.syntactical.StandardTokenParsers}
for lexing and
\lstinline{scala.util.parsing.combinator.PackratParsers} for Packrat
parsers.
The first two lines are used for lexing. The third line defines a
parser which can, for instance, parse \lstinline{"str(5)"} and produce
the string \lstinline{"5"}. In \lstinline{StandardTokenParsers} there
are a number of basic parsers, such as \lstinline{numericLit} that
parses integers, and \lstinline{stringLit} that parsers string
literals. Moreover, in Scala the syntax for parsing
is quite concise and clear, as shown above, with the help of
parser combinators. The combinators \lstinline{~>} and \lstinline{<~} only collect
results from the parts that arrows point to, and the combinator \lstinline{^^} is
followed by a function which is applied to a parser to return new
values. Finally, a generic \lstinline{runParser}
function is used for testing:

\begin{lstlisting}
def runParser(p: Parser[_]): String => Unit = in => {
    val t = phrase(p)(new lexical.Scanner(in))
    if (t.successful) println(t.get) else scala.sys.error(t.toString)
}
\end{lstlisting}
Now \lstinline{runParser(p)("str(5)")} prints out \lstinline{"5"} as expected.

\begin{table}[t]
\centering
\begin{tabular}{l}
\hline
\begin{lstlisting}
def ~[U](q: => Parser[U]): Parser[~[T, U]]
\end{lstlisting} \\
\hspace{.2in}- A parser combinator for sequential composition. \\
\hline
\begin{lstlisting}
def ^^[U](f: (T) => U): Parser[U]
\end{lstlisting} \\
\hspace{.2in}- A parser combinator for function application. \\
\hline
\begin{lstlisting}
def ^^^[U](v: => U): Parser[U]
\end{lstlisting} \\
\hspace{.2in}- A parser combinator that changes a successful result into the specified value. \\
\hline
\begin{lstlisting}
def <~[U](q: => Parser[U]): Parser[T]
\end{lstlisting} \\
\hspace{.2in}- A parser combinator for sequential composition which keeps only the left result. \\
\hline
\begin{lstlisting}
def ~>[U](q: => Parser[U]): Parser[U]
\end{lstlisting} \\
\hspace{.2in}- A parser combinator for sequential composition which keeps only the right result. \\
\hline
\begin{lstlisting}
def repsep[T](p: => Parser[T], q: => Parser[Any]): Parser[List[T]]
\end{lstlisting} \\
\hspace{.2in}- A parser generator for interleaved repetitions. \\
\hline
\begin{lstlisting}
def ident: Parser[String]
\end{lstlisting} \\
\hspace{.2in}- A parser which matches an identifier. \\
\hline
\begin{lstlisting}
def numericLit: Parser[String]
\end{lstlisting} \\
\hspace{.2in}- A parser which matches a numeric literal. \\
\hline
\begin{lstlisting}
def |[U >: T](q: => Parser[U]): Parser[U]
\end{lstlisting} \\
\hspace{.2in}- A parser combinator for alternative composition. \\
\hline
\begin{lstlisting}
def |||[U >: T](q0: => Parser[U]): Parser[U]
\end{lstlisting} \\
\hspace{.2in}- A parser combinator for alternative with longest match composition. \\
\hline \\
\end{tabular}
\caption{Some frequently used functions in Scala parser API.}\label{tab1}
\end{table}\bruno{We should double-check that this table lists the
  combinators used in the paper.}

\paragraph{Packrat Parsing API} Table~\ref{tab1} shows some common parser combinators
that are used in our implementation.  There are several standard
parser combinators. For example \lstinline{|} is the combinator for
alternative parsers. Other combinators have descriptions of their
semantics on the table.

There are two features of the Packrat Parsing library that play a key role
for modularity, which are discussed next.

\paragraph{Longest Match Composition} The Packrat parsing library 
contains an alternation combinator, which does \emph{longest match
  composition}. This combinator has a different semantics from
\lstinline{|}.
Suppose that we are composing a list of parsers using alternative
\lstinline{|}, in an order. A parser in the front might be successful
in parsing a substring of the input, but then it unexpectedly stops
the parsing and discards the rest, while another parser might be able
to parse the whole input. In that case, using \lstinline{|||} we no
longer need to worry about the order of composition, which potentially
encourages its extensibility. A concrete example will be illustrated
for this later. \bruno{Please show example here. }

\paragraph{Left Recursion} The Scala Packrat Parsing library 
supports direct left-recursion. Left recursion has been considered
as a big problem especially in recursive descent parsers~\cite{}, and many
existing parsing libraries have limited or no support for it. From theoretical point of
view, the algorithm behind Packrat parsers supports both direct and
indirect left-recursion. In practice, the current version of the library is still
buggy in some cases with indirect left-recursion, but we believe that
they will fix it in the future, and direct left-recursion is
already practical to use for a large number of applications.


\section{Modular Parsing Library}\label{sec:library}

%\begin{itemize}
%\item Fixpoints library + explaining delegation with some examples
%\item Alternative combinator + others
%\item Trait Composition to do Language Composition
%\end{itemize}

In this section, we will further look into the mechanism behind our modular parsing library.
We will first introduce some concepts independently from the process of parsing, including open recursion and Object Algebras,
then present how they are integrated in our library.
Starting from simple examples, a series of extensions and refactorings will be applied to illustrate
how we achieve modularity in a type-safe way.

\subsection{Open Recursion and Delegation}\label{subsec:openrecursion}

Open recursion is known as a useful feature that one method body can ``invoke another method of the same object via a special variable called \lstinline{self} or, in some languages, \lstinline[keywords={}]{this}'' (by Ralf Hinze). The interesting thing is that such a variable \lstinline{self} is late-bound, or open to the recursion, which means it can integrate some additional features defined later, whereas the method that takes \lstinline{self} as a parameter can simply delegate the known cases to it.

For sure we cannot always leave it open; upon using it we have to close the recursion, at which time we need to take a fix-point. Here one might suddenly be reminded of the \lstinline{fix} function in some functional languages like Haskell. Below is the famous Fibonacci example:
\begin{lstlisting}[language=Haskell,keywords={}]
fib :: Int -> Int
fib 0 = 0
fib 1 = 1
fib n = fib (n - 1) + fib (n - 2)
\end{lstlisting}
An alternative approach using open recursion is as follows:
\begin{lstlisting}[language=Haskell,keywords={type}]
type Fix t = t -> t

fib' :: Fix (Int -> Int)
fib' self 0 = 0
fib' self 1 = 1
fib' self n = self (n - 1) + self (n - 2)
\end{lstlisting}
where \lstinline{self} is the explicit self-reference, namely the delegator. Functions like \lstinline{fib'} which takes
an explicit self-reference as its parameter are called \textit{generators} (by Ralf Hinze). The definition of \lstinline{fix} is:
\begin{lstlisting}[language=Haskell,keywords={}]
fix f = let x = f x in x
\end{lstlisting}
If we close the recursion on \lstinline{fib'} only, the evaluation of \lstinline{fix fib'} \lstinline{2} is as follows:
\begin{lstlisting}[language=Haskell,keywords={}]
   fix fib' 2
= let x = fib' x in x 2
= let x = fib' x in fib' x 2
= let x = fib' x in (x 1) + (x 0)
= let x = fib' x in (fib' x 1) + (fib' x 0)
= let x = fib' x in 1 + 0
= 1
\end{lstlisting}
which behaves the same as \lstinline{fib 2}. The process of evaluation will terminate if there are some base cases (like the first two in
\lstinline{fib'}) which stop the recursion in their branches, and the evaluation is finally reduced to those cases.

Suppose now we want to make use of open recursion for additional operations. We first define a combinator which combines the results of two generators in a pair:
\begin{lstlisting}[language=Haskell,keywords={}]
(*) :: Fix (a -> b) -> Fix (a -> c) -> Fix (a -> (b, c))
f * g = \self x -> (f (fst . self) x, g (snd . self) x)
\end{lstlisting}
Then we define a generator which prints out a number as a string:
\begin{lstlisting}[language=Haskell,keywords={}]
show' :: Fix (Int -> String)
show' self x = show x
\end{lstlisting}
Now \lstinline{fix (fib'}\lstinline{ * show')} \lstinline{2} results in \lstinline{(1, "2")}. We have seen that both \lstinline{fib'} and \lstinline{show'} are integrated in the evaluation. In Scala, however, there is no \lstinline{fix} function, but it can be easily defined since it is based on lazy evaluation. See below:
\begin{lstlisting}
def fix[A](f : (=> A) => A) : A = {
    lazy val a : A = f(a)
    a
}
\end{lstlisting}
Next we define the general combinator \lstinline{merge} for open recursion. It is similar to the \lstinline{(*)} above, but instead of putting two values in a pair by default, \lstinline{merge} takes a parameter called \lstinline{op}, which tells the associative operation of two results. The definition is straightforward:
\begin{lstlisting}
def merge[E, F, G, H[_,_]](op : F => G => H[F, G], x : (=> E) => F, y : (=> E) => G) : (=> E) => H[F, G] = e => op(x(e))(y(e))
\end{lstlisting}


\section{Object Algebras and Parsing}\label{sec:algebrasandparsing}

Object Algebras, first introduced by ... et al as a solution to the famous \textit{Expression Problem}, provide extensibility on
both data variants and operations for structures like abstract syntax trees. We integrate Object Algebras directly to enhance the
extensibility of open parsers.

\subsection{Object Algebras}\label{subsec:objectalgebras}
Object Algebras captures a design pattern to address the Expression Problem nicely,
achieving two dimensions of extensibility (data variants and operations) in a modular and type-safe way.
Because of this, the definition of data structures is separated from behaviors on them, and future extensions
to both sides no longer require existing code to be modified, supporting separate compilation.

In Object Algebras, ASTs as recursive data structures are defined using traits, where each constructor corresponds
to an abstract method inside. The example from Section~\ref{subsec:parsingwithopen} is again used here for illustration.
At first the language only supports variables and applications:
\begin{lstlisting}
trait ExprAlg[E] {
    def varE(x : String) : E
    def appE(e1 : E, e2 : E) : E
}
\end{lstlisting}
The language has two constructors: literals and additions. The trait \lstinline{ExprAlg} is called an \textit{Object Algebra interface},
whereas as a factory, it cannot produces objects for the expressions directly like traditional approaches, but abstract the results in its
type parameter \lstinline{E} instead. To realize an operation on expressions, we simply instantiate the type parameter by a concrete type and
provides implementations for all cases. Below is an example of collecting all free variables in an expression:
\begin{lstlisting}
trait FreeVars extends ExprAlg[List[String]] {
    def varE(x : String) = List(x)
    def appE(e1 : List[String], e2 : List[String]) = e1 ++ e2
}
\end{lstlisting}
Here \lstinline{FreeVars} is called an \textit{Object Algebra}. It traverses an expression bottom-up, and returns a list of strings as the result.
Furthermore, one can define a new trait to implement pretty-printing on expressions:
\begin{lstlisting}
trait ExprPrint extends ExprAlg[String] {
    def varE(x : String) = x
    def appE(e1 : String, e2 : String) = "(" + e1 + " " + e2 + ")"
}
\end{lstlisting}
On the other hand, the data variants can be extended by inheriting \lstinline{ExprAlg} and adding new cases only. Suppose we want to
have literals in expressions, a new Object Algebra interface \lstinline{LitAlg} is defined as follows:
\begin{lstlisting}
trait LitAlg[E] extends ExprAlg[E] {
    def litE(n : Int) : E
}
\end{lstlisting}
Now pretty-printing on the new language can be realized by code reuse, without modifying existing code:
\begin{lstlisting}
trait LitPrint extends ExprPrint {
    def litE(n : Int) = n.toString
}
\end{lstlisting}
On this, to create an expression of \lstinline{LitAlg}, a generic method is defined as follows:
\begin{lstlisting}
def build[E](alg : LitAlg[E]) : E =
    alg.appE(alg.varE("x"), alg.litE(3))
\end{lstlisting}
Such a method implicitly represents the expression \lstinline{"x 3"}. The code
\begin{lstlisting}
build(new LitPrint(){})
\end{lstlisting}
results in \lstinline{"(x 3)"}, as the result of pretty-printing. Now we observe that an ``object'' of expression actually
has the function type \lstinline{LitAlg[E] =>} \lstinline{E} for generic \lstinline{E}. At this point, one may argue
that it is illusory compared to traditional objects, and since we are struggling against the parsing problem, if a parser
produces structures as functions, it could be hardly be further processed like objects, especially, in the design of a compiler
there will be a lot of desugarings (or transformations). However, we argue that this is not an issue, because functions can be used
in a very clever way. For example, if one wants to transform an \lstinline{ExprAlg} expression to a \lstinline{LitAlg} expression by
replacing some variables with values based on a variable environment, a transformation algebra can be implemented as follows:
\begin{lstlisting}
trait Refactor[E] extends ExprAlg[E] {
    def alg : SubAlg[E]
    def env : Map[String, Int]
    def varE(x : String) = if (env.contains(x)) alg.litE(env(x)) else alg.varE(x)
}
\end{lstlisting}\haoyuan{Is this example OK?}
When \lstinline{env = Map("y"} \lstinline{-> 3)}, the \lstinline{ExprAlg} expression \lstinline{"x y"} is changed to \lstinline{LitAlg} expression \lstinline{"x 3"}. The abstract method
\lstinline{alg} in \lstinline{Refactor} can be any algebra of \lstinline{SubAlg}, simply for delegation. By passing a concrete implementation
to \lstinline{alg}, \lstinline{Refactor} will be able to apply transformation to an \lstinline{ExprAlg} expression before \lstinline{alg} returns the result.

\subsection{Parsing with Object Algebras}\label{subsec:parsingwithoa}

Just as shown above, \lstinline{ExprAlg} is defined for a small language in an Object-Algebra style. Furthermore,
A parser can be defined for this whole language at once. Since Object Algebras represent ``objects'' implicitly as functions like \lstinline{ExprAlg[E] =>} \lstinline{E} for generic \lstinline{E}, such a parser should consume an algebra from its parameter, then return a value of \lstinline{Fix[Parser[E]]} as in open recursion, where \lstinline{E} is abstract.
\begin{lstlisting}
trait ExprParser[E] {
    val pE : ExprAlg[E] => Fix[Parser[E]] = alg => p =>
        ident ^^ alg.varE |||
        p ~ p ^^ { case e1 ~ e2 => alg.appE(e1, e2) }
}
\end{lstlisting}
It is observed that \lstinline{pE} is in fact a parser generator, where \lstinline{alg} appears as a parameter, which can be any algebra, and after \lstinline{alg} is fed, it returns a \lstinline{Fix[Parser[E]]}. Again \lstinline{p} is the explicit self-reference of the open parser. Inside the body, the algebra is invoked correspondingly for all cases right after parsing. Note that the two cases are combined using the original \lstinline{|||} operator. To make \lstinline{alg} as a generic algebra, the enclosing trait of the parser, namely \lstinline{ExprParser}, is abstracted over \lstinline{E}. Behaviors, or algebras that can be fed to such a parser, are expected to be defined independently from the Object Algebra interface, just like the algebra \lstinline{ExprPrint} in Section~\ref{subsec:objectalgebras}.

To explain the modularity, we again add literals to expressions, but instead of using inheritance as Section~\ref{subsec:objectalgebras}, \lstinline{LitAlg} is defined independently together with its own parser and algebra:
\begin{lstlisting}
trait LitAlg[E] {
    def litE(n : Int) : E
}

trait LitPrint {
    def litE(n : Int) = n.toString
}

trait LitParser[E] {
    val pE : LitAlg[E] => Fix[Parser[E]] = alg => p =>
        numericLit ^^ alg.litE
}
\end{lstlisting}
In this case, \lstinline{LitAlg} is yet another small language, which is self-contained. To merge the two languages together, we require their grammars
to be merged, and we obtain a combined language, whose parser integrates the two small parsers using alternative. In Scala we use compound types, namely use \lstinline{"with"} for two Object Algebra interface types, to avoid polluting the namespace.

\begin{lstlisting}
trait ExprLitParser[E] {
    val pE : (ExprAlg[E] with LitAlg[E]) => Fix[Parser[E]] = alg => p => {
        val pExprE = new ExprParser[E](){}.pE(alg)(p)
        val pLitE = new LitParser[E](){}.pE(alg)(p)
        pExprE ||| pLitE
    }
}
\end{lstlisting}

%But we can only abstract over the whole interface type, instead of working on its type parameters, as the number of type parameters is not fixed, where we will discuss more on multiple syntax later. In summary, \lstinline{"|||"} produces a combined parser, which takes a compound algebra as its parameter.

In client code, a user can either define a new function with type \lstinline{(ExprAlg[E] with LitAlg[E]) =>} \lstinline{Parser[E]} based on \lstinline{pE}, which implies an algebra is applied after \lstinline{fix}, or simply feed an algebra to \lstinline{pE} at first, then obtain the result from \lstinline{fix}:
\begin{lstlisting}
trait ExprLitPrint extends ExprPrint with LitPrint
val parsePrint = fix(new ExprLitParser[String](){}.pE(new ExprLitPrint(){}))
val result = runParser(parsePrint)("x 3") // prints "(x 3)"
\end{lstlisting}
\haoyuan{Is there a way to avoid defining ExprLitPrint?}

\section{Implementation}\label{sec:implementation}

\subsection{Fusion of Concepts}\label{subsec:fusion}

In the previous sections we have talked about Packrat Parsing, open recursion and Object Algebras. These components fuse together in our library implementation, for the initial goal, which is to build extensible parsers. We want to argue here that those aspects are totally orthogonal, and hence can have their alternatives. For example, Object Algebras are helpful to build extensible data structures, whereas in functional programming, there has been a large amount of related work on extensible datatypes, including DTC and MRM for the Haskell language. And a better
parsing library would potentially introduce a lot of fancy features and parser combinators, run with high efficiency, or even support indirect left-recursion.
\haoyuan{Open recursion could also have alternatives?} For now, our implementation integrates the existing ones, and has turned out to be practical in various applications.

The library is made up of around twenty lines of code only, as shown in Figure~\ref{fig:sourcelibrary}. It consists of some types wrapped up by synonyms, basic functions \lstinline{fix} and \lstinline{runParser}, and the combinator \lstinline{|||} from Section~\ref{subsec:differentsyntax}.

\begin{figure}[htbp]
\centering
\begin{lstlisting}
import scala.util.parsing.combinator._
import scala.util.parsing.combinator.syntactical._

object Library extends StandardTokenParsers with PackratParsers {
    type Fix[T] = (=> T) => T
    type Parser[T] = PackratParser[T]
    type OpenParser[Alg, List, T] = Alg => (=> List) => Parser[T]

    def fix[A](f: Fix[A]): A = { lazy val a: A = f(a); a }
    def runParser(p : Parser[_]) : String => Unit = in => {
        phrase(p)(new lexical.Scanner(in)) match {
            case t if t.successful => println(t.get)
            case t                 => scala.sys.error(t.toString)
        }
    }

    implicit class Combinator[Alg1, Alg2, R, E](x : Alg1 => (=> R) => Parser[E]) {
        def |||(y : Alg2 => (=> R) => Parser[E])
            : Alg1 with Alg2 => (=> R) => Parser[E]
                = alg => l => x(alg)(l) ||| y(alg)(l)
    }
}
\end{lstlisting}
\caption{Source code of our library.}\label{fig:sourcelibrary}
\end{figure}

\subsection{Framework of a Language Component}\label{subsec:framework}

Having pointed out that our goal is to realize modular and extensible parsers, we recommend users to write code in a modular and
organized way. The framework we use for a language, or rather a language component in our case study, consists of the following three
parts:
\begin{itemize}
\item \textbf{Object Algebra interface:} defined as a trait with several cases, corresponding to the grammar rules.
\item \textbf{Parser:} implementation of the parser, using open recursion and Object Algebras, with its scoping trait abstracted over the type parameters.
\item \textbf{Algebra:} some behaviors (operations) on the data structure, such as pretty-printing and so on.
\end{itemize}

Figure~\ref{fig:objectexpr} shows how to encapsulate \lstinline{ExprAlg} and its related stuff in object \lstinline{Expr} using our library. Note that we use type bounds to address the issue remained in Section~\ref{subsec:differentsyntax}. \lstinline{List[E]} is a type synonym for a record type, while the type bound \lstinline{L} \lstinline{<:} \lstinline{List[E]} in \lstinline{AlgParser} states that the fix-point of our parser is a record, which contains at least \lstinline{pE} for parsing expressions. Hence it saves us from updating the type of fix-points all the time.

\begin{figure}[htbp]
\centering
\begin{lstlisting}
object Expr {
    trait Alg[E] {
        def varE(x : String) : E
        def appE(e1 : E, e2 : E) : E
    }

    type List[E] = { val pE : Parser[E] }

    trait AlgParser[E, L <: List[E]] {
        val pExprE : OpenParser[Alg[E], L, E] = alg => p => {
            lazy val pE = p.pE
            ident ^^ alg.varE |||
            pE ~ pE ^^ { case e1 ~ e2 => alg.appE(e1, e2) }
        }
    }

    trait Print extends Alg[String] {
        def varE(x : String) = x
        def appE(e1 : String, e2 : String) = "(" + e1 + " " + e2 + ")"
    }
}
\end{lstlisting}
\caption{Object \lstinline{Expr} for expressions.}\label{fig:objectexpr}
\end{figure}

Now \lstinline{Expr} turns out to be a self-contained small language with parsing and pretty-printing. To construct a different language component, we implement \lstinline{LamAlg} in Figure~\ref{fig:objectlam}, where we declare a different \lstinline{List} for multiple syntax.

\begin{figure}[htbp]
\centering
\begin{lstlisting}
object Lam {
    trait Alg[E, T] {
        def intT() : T
        def lamE(x : String, t : T, e : E) : E
    }

    type List[E, T] = { val pE : Parser[E]; val pT : Parser[T] }

    trait AlgParser[E, T, L <: List[E, T]] {
        lexical.reserved += ("Int")
        lexical.delimiters += ("\\", ":", ".")
        val pLamE : OpenParser[Alg[E, T], L, E] = alg => p =>
            ("\\" ~> ident) ~ (":" ~> p.pT) ~ ("." ~> p.pE) ^^
                { case x ~ t ~ e => alg.lamE(x, t, e)}
        val pLamT : OpenParser[Alg[E, T], L, T] = alg => p =>
            "Int" ^^^ alg.intT
    }

    trait Print extends Alg[String, String] {
        def intT() = "Int"
        def lamE(x : String, t : String, e : String) = "\\" + x + ":" + t + "." + e
    }
}
\end{lstlisting}
\caption{Object \lstinline{Lam} for lambdas.}\label{fig:objectlam}
\end{figure}

\subsection{More modularity}

We have optimized the framework for not only modular parsers, but also languages, and that is the reason why we call them language components. To achieve
higher-order modularity, it is necessary to follow the structure when composing two small components. Figure~\ref{fig:objectlamexpr} presents the combination of \lstinline{Expr} and \lstinline{Lam}, as a new language, which is again modular for future compositions. Note that the \lstinline{AlgParser} extends both parent parsers, so as to reuse their parser functions, as well as composing lexers automatically.

\begin{figure}[htbp]
\centering
\begin{lstlisting}
object LamExpr {
    type List[E, T] = Lam.List[E, T]
    type Alg[E, T] = Expr.Alg[E] with Lam.Alg[E, T]
    trait Print extends Expr.Print with Lam.Print

    trait AlgParser[E, T, L <: List[E, T]] extends Expr.AlgParser[E, L]
            with Lam.AlgParser[E, T, L] {
        val pLamExprE = pExprE ||| pLamE
        val pLamExprT = pLamT
    }
}
\end{lstlisting}
\caption{Object \lstinline{LamExpr}, composing \lstinline{Lam} and \lstinline{Expr} together.}\label{fig:objectlamexpr}
\end{figure}

Below we give an example of client code for testing \lstinline{LamExpr}. \lstinline{parse} is a generic parser generator, which takes an algebra.
In \lstinline{parseAndPrint} we pass the printing algebra to \lstinline{parse}, so that it parses the input and apply pretty-printing, returning the result as expected.

\begin{lstlisting}
class List[E, T](pe : Parser[E], pt : Parser[T]) {
    val pE = pe; val pT = pt
}
def parse[E, T](inp: String)(alg: LamExpr.Alg[E, T]) = {
    def parser : Fix[List[E, T]] = l => {
        val lang = new LamExpr.AlgParser[E, T, List[E, T]] {}
        new List[E, T](lang.pLamExprE(alg)(l), lang.pLamExprT(alg)(l))
    }
    runParser(fix(parser).pE)(inp)
}
def parseAndPrint(inp: String) = parse(inp)(new LamExpr.Print {})

val result = parseAndPrint("\\x:Int.x y") // prints "\\x:Int.(x y)"
\end{lstlisting}


% Object Algebras
% type synonyms, Parser[Expr], Fix[..]
% more: two pretty-printers OldPretty and NewPretty, they are separated to show modularity.
% subsection: modify the code using List[E] and type bounds.
% more subsections: lexing, traits, operations, language composition (more extensibility)
% use paragraphs
%...

\section{Case Studies}\label{sec:casestudy}


To demonstrate the utility of our modular parsing approach, we
implemented parsers of the first 18 calculi from book \textit{Types and Programming Languages} (TAPL) \cite{pierce2002types}. We compared our implementation with a non-modular implementation
we found online, which is also written in Scala and uses the same Packrat parsing library.
We counted source lines of code (SLOC) and measured execution time for both implementations.
The result suggests that our implementation saves 69\% code comparing with that non-modular one.

\subsection{Implementation}\label{subsec:implementation}

TAPL introduces several calculi from simple to complex, by gradually adding new features to syntax. These calculi are suitable for our case study for mainly two reasons. Firstly, they capture many of the language features
required in realistic programming languages, such as lambdas, records and variants. Secondly, the evolution of
calculi in the book reveals the advantages of modular representation
of abstract syntax and modular parsing, which is the key functionality
of our approach. By extracting common components from those calculi
and reusing them, we obtain considerably code reuse as shown later.

\paragraph{Extracting Language Components}
Using the pattern demonstrated in Section~\ref{subsec:language-component}, we extract
reusable components from all the calculi. Each
component, which may contain several syntactical structures,
represents a certain feature of the language. For
example, the \lstinline{VarApp} component below represents variables and
function applications.

\lstinputlisting[linerange=13-29]{code/src/papercode/Sec6CaseStudy/Code1.scala}% APPLY:linerange=CASESTUDY_VARAPP

Similar as before, each component is represented by a Scala object which includes \lstinline{Alg} for the abstract syntax, \lstinline{Print} for pretty-printing, and \lstinline{Parse} for parsing.

We have some naming conventions in our code, \lstinline{E} represents
expressions, \lstinline{T} represents types and \lstinline{K}
represents kinds. They are the three sorts of syntax in our case study.
In the component \lstinline{VarApp} we only have expressions. We use some helper traits for eliminating duplicate definitions, such as \inlinecode{EParser} containing \inlinecode{pE} for parsing expressions.

\lstinputlisting[linerange=9-9]{code/src/papercode/Sec6CaseStudy/Code1.scala}% APPLY:linerange=CASESTUDY_EPARSER

\paragraph{Composing Language Components}
Each calculus could be composed directly from components and other
calculi as needed. For example, the calculus \lstinline{Untyped} in our case study,
representing the famous untyped lambda calculus, is constructed from components \lstinline{VarApp}
and untyped lambda abstraction \lstinline{UntypedAbs}.

The code of building \lstinline{Untyped} is presented in Figure~\ref{fig:casestudy-untyped}. Note that in the parser \inlinecode{Parse}, we need to override the object algebra interface and
parsing functions accordingly.

\begin{figure}[t]
\lstinputlisting[linerange=33-60]{code/src/papercode/Sec6CaseStudy/Code1.scala}% APPLY:linerange=CASESTUDY_UNTYPED
\caption{Build the \inlinecode{Untyped} calculus by composing to language components.}\label{fig:casestudy-untyped}
\end{figure}


\paragraph{Dependency Overview}
Figure \ref{fig:dependency} shows the
dependency of all the components and calculi in our case study. Grey
boxes are calculi and white boxes are components. An arrow starting
from box A to box B denotes that B includes and thus reuses A.

\begin{figure*}
    \centering
    \includegraphics[width=\textwidth]{resources/depGraph.pdf}
    \caption{Dependency graph of all calculi and components in case study.}
    \label{fig:dependency}
\end{figure*}

As shown in the graph, some components such as \lstinline{VarApp} are
created from scratch, while others such as \lstinline{Typed} are
extended from existing components. Since calculi and components have similar signatures, each calculus
can also be extended and reused directly. For example, calculus \lstinline{FullRef} extends from
calculus \lstinline{FullSimple}.

From the dependency graph, we know
that common components such as \lstinline{VarApp} are reused in
lots of calculi. Such reuse could shorten the code considerably. We
will show this advantage and inspect the possible performance penalty in
a quantitative way in the next subsection.

\subsection{Comparison}\label{subsec:cs-comparison}

\newcommand\ourimpl{$\texttt{Mod}_{\texttt{OA}}$}
\newcommand\ilyaimpl{\texttt{NonMod}}
\newcommand\ourclass{$\texttt{Mod}_{\texttt{CLASS}}$}
\newcommand\ilyalongest{$\texttt{NonMod}_{\texttt{|||}}$}

We compared our implementation (named \ourimpl{}) with an implementation
by Ilya Klyuchnikov (named \ilyaimpl{}), available online\footnote{https://github.com/ilya-klyuchnikov/tapl-scala/}.
\ilyaimpl{} is suitable for comparison, because it is also
written in Scala using the same parser combinator library.
Furthermore, it includes parsers of all the 18 calculi we have, but
written in a non-modular way. Thus it is not able to reuse existing
code when those calculi share common features.

The comparison is made from two aspects. First, we want to discover
the amount of code reuse using our modular parsing approach.
For this purpose, we measured source lines of code (SLOC) of two implementations.
Second, we are interested to assess the performance penalty caused by modularity.
Thus we compared the execution time of parsing random expressions between two implementations.

\paragraph{Standard of Comparison}
In the SLOC comparison, all blank lines and comment lines are excluded,
and we formatted the code of both implementations to guarantee that
the length of each line does not exceed 120 characters. Furthermore,
because \ilyaimpl{} has extra code such as semantics,
we removed all irrelevant code and only keep abstract
syntax definition, parser and pretty-printer for each calculus, to
ensure a fair comparison.

For the comparison of execution time, we built a generator to randomly
generate valid expressions of each calculus, according to the syntax. These expressions are
written to test files, one file per calculus. Each test file consists of 500
expressions randomly generated, and the size of test files varies from 20KB to 100KB.
Then we run the corresponding parser to parse the file and the pretty-printer to print the result.
The average execution time of 5 runs excluding reading input file was calculated, in milliseconds.

\begin{table*}
    \centering
    \begin{tabular}{|l|r|r|r|r|r|r|}
      \hline
        \multirow{2}{*}{\bfseries Calculus Name} & \multicolumn{3}{ c| }{\bfseries SLOC} & \multicolumn{3}{ c| }{\bfseries Time (ms)} \\ \cline{2-7}
        \multicolumn{1}{|c|}{} & \ilyaimpl{} & \ourimpl{} & \bfseries (+/-)\% & \ilyaimpl{} & \ourimpl{} & \bfseries (+/-)\% \\
      \hline
      Arith & 77 & 79 & +2.6 & 112 & 117 & +4.5 \\
Untyped & 48 & 59 & +22.9 & 85 & 134 & +57.6 \\
FullUntyped & 131 & 95 & -27.5 & 199 & 277 & +39.2 \\
TyArith & 89 & 55 & -38.2 & 89 & 105 & +18.0 \\
SimpleBool & 90 & 59 & -34.4 & 114 & 149 & +30.7 \\
FullSimple & 240 & 144 & -40.0 & 350 & 491 & +40.3 \\
Bot & 87 & 61 & -29.9 & 101 & 229 & +126.7 \\
FullRef & 273 & 79 & -71.1 & 344 & 537 & +56.1 \\
FullError & 111 & 52 & -53.2 & 141 & 191 & +35.5 \\
RcdSubBot & 125 & 32 & -74.4 & 147 & 171 & +16.3 \\
FullSub & 221 & 50 & -77.4 & 297 & 343 & +15.5 \\
FullEquiRec & 246 & 52 & -78.9 & 415 & 514 & +23.9 \\
FullIsoRec & 255 & 61 & -76.1 & 323 & 383 & +18.6 \\
EquiRec & 81 & 31 & -61.7 & 100 & 84 & -16.0 \\
Recon & 138 & 37 & -73.2 & 163 & 172 & +5.5 \\
FullRecon & 142 & 37 & -73.9 & 175 & 197 & +12.6 \\
FullPoly & 244 & 89 & -63.5 & 358 & 625 & +74.6 \\
FullOmega & 311 & 89 & -71.4 & 342 & 394 & +15.2 \\
\hline
Total & 2909 & 1161 & -60.1 & 3855 & 5113 & +32.6 \\

      \hline
      \multicolumn{7}{c}{}
    \end{tabular}
    \caption{Comparison of SLOC and execution time.}
    \label{tab:comparison}
\end{table*}

\paragraph{Comparison Results}
Table \ref{tab:comparison} shows results of the comparison.
The overall result is that 69.2\% of code is reduced using our
approach, and our implementation is 42.7\% slower.

The good SLOC result is because of that the code of common language features
such as variables, lambda abstractions, etc., are reused lots of times in
the whole case study. We can see that in the first two calculi
\lstinline{Arith} and \lstinline{Untyped} we are not better than \ilyaimpl{}, because in such two cases we do not reuse any existing
components. However in the following 16 calculi, we can reuse
language components, resulting considerably code reduction. In particular,
the calculi \inlinecode{EquiRec}, \inlinecode{Recon} and some others are only 22 lines
in our implementation, because we only compose existing codes in such cases.

To discover the reasons of slower execution time, we made some more experiments
on two possible factors which could affect the performance.
They are object algebra and the longest match alternative combinator.
We use object algebra for ASTs and the longest match alternative combinator \inlinecode{|||} for parsing,
while \ilyaimpl{} uses case class and the ordinary alternative combinator.

We implemented two more versions. One is a modified version of our implementation, named \ourclass{}, with object algebra replaced by case class for the ASTs.
The other is a modified version of \ilyaimpl{}, named \ilyalongest{}, using the longest match alternative combinator instead of the ordinary one.

\begin{table*}
    \centering
    \begin{tabular}{|l|r|r|r|r|r|r|r|}
      \hline
        \multirow{2}{*}{\bfseries Calculus Name} & \ilyaimpl{} & \multicolumn{2}{ c| }{\ourimpl{}} & \multicolumn{2}{ c| }{\ilyalongest{}} & \multicolumn{2}{ c| }{\ourclass{}} \\ \cline{2-8}
        \multicolumn{1}{|c|}{} & \multicolumn{1}{c|}{\bfseries Time} & \bfseries Time & \bfseries (+/-)\% & \bfseries Time & \bfseries (+/-)\% & \bfseries Time & \bfseries (+/-)\% \\
      \hline
        Arith & 741 & 913 & +23.2 & 793 & +7.0 & 932 & +25.8 \\
        Untyped & 770 & 1018 & +32.2 & 821 & +6.6 & 1007 & +30.8 \\
        FullUntyped & 1297 & 1854 & +42.9 & 1343 & +3.5 & 1767 & +36.2 \\
        TyArith & 746 & 888 & +19.0 & 772 & +3.5 & 918 & +23.1 \\
        SimpleBool & 1376 & 1782 & +29.5 & 1494 & +8.6 & 1824 & +32.6 \\
        FullSimple & 1441 & 2270 & +57.5 & 1574 & +9.2 & 2226 & +54.5 \\
        Bot & 1080 & 1287 & +19.2 & 1078 & -0.2 & 1306 & +20.9 \\
      %\hline
        %\multicolumn{1}{|c|}{\dots} & \multicolumn{7}{c|}{\dots} \\
      %\hline
        FullRef & 1438 & 2291 & +59.3 & 1544 & +7.4 & 2142 & +49.0 \\
        FullError & 1410 & 1946 & +38.0 & 1524 & +8.1 & 1981 & +40.5 \\
        RcdSubBot & 1247 & 1524 & +22.2 & 1285 & +3.0 & 1612 & +29.3 \\
        FullSub & 1320 & 1979 & +49.9 & 1393 & +5.5 & 1899 & +43.9 \\
        FullEquiRec & 1407 & 2200 & +56.4 & 1561 & +10.9 & 2156 & +53.2 \\
        FullIsoRec & 1492 & 2253 & +51.0 & 1648 & +10.5 & 2236 & +49.9 \\
        EquiRec & 994 & 1254 & +26.2 & 1048 & +5.4 & 1304 & +31.2 \\
        Recon & 1044 & 1482 & +42.0 & 1128 & +8.0 & 1506 & +44.3 \\
        FullRecon & 1094 & 1645 & +50.4 & 1161 & +6.1 & 1652 & +51.0 \\
        FullPoly & 1398 & 2086 & +49.2 & 1511 & +8.1 & 2019 & +44.4 \\
        FullOmega & 1451 & 2352 & +62.1 & 1582 & +9.0 & 2308 & +59.1 \\
      \hline
        Total & 21746 & 31024 & +42.7 & 23260 & +7.0 & 30795 & +41.6 \\
      \hline
        \multicolumn{8}{c}{}
    \end{tabular}
    \caption{Execution time of four implementations.}
    \label{tab:ext-comparison}
\end{table*}

Table \ref{tab:ext-comparison} shows comparison results of these four versions. It suggests that the difference of running time between
using object algebra and class is little, roughly 1\%.
The usage of longest match combinator slows the performance by 7\%. The main reason of slower
execution time may be the overall structure of the modular parsing approach, because we indeed have
more intermediate function calls and method overriding. However, it is worth mentioning that
because of the memoization technique of Packrat parsers, we are only constant times
slower, the algorithmic complexity is still the same.


\section{Related Work}\label{sec:relatedwork}

- extensible parsing, language workbenches: rats, noa (this one already uses OA), modular semantic actions, (syntactic modularity, no separate compilation, modular type-checking)
(read more papers, see if they talk about this issue, some potential solutions)
(attribute grammars?)

- parser combinators for type-checking, previous work has not shown how to support modularity (ASTs); left-recursion and back-tracking in related techniques

- modularity: object algebras, dtc and mrm (problem with parsing, is there any related work? (PB: a paper on unfolds: build the AST))

(parsing in Javascript: using delegation, does it support modular AST)

noa, shy: shy: only override some interesting cases (transformation is tedious)
bruijn indices: parsing + transformation

Our work integrates several components including extensible parsing, parser combinators and modular datatypes. There has been a great amount of related papers
on those hot topics, of which some inspired us of this paper and encourages us for more exploration. This section will try to lead a discussion on what difference we have made.

\paragraph*{Extensible Parsing} Extensible parsing is achieved in many different areas, of which parser generators are a mainstream area specially designed for modular syntax and parsing. Many parser generators [OMeta, ANTLR, Rats!, noa, Ohm] \haoyuan{correct for Ohm?} support modular grammars, more specifically, they allow users to create new modules where new non-terminals and production rules can be introduced, some can even override existing rules in the old grammar modules. For instance, \textit{Rats!} [Rats!]
constructs its own module system for the collection of grammars, while NOA [NOA] uses Java annotation processing to gather information together. Those parser combinators focus on the \textit{syntactic extensibility} of grammars, and rely on a whole compilation to generate a global parser, even with a slight modification. Some of them may statically check the correctness and unambiguity of grammars, but separate compilation and modular type-checking remain unsolved, together with the issue of performance.

Besides, macro systems like C preprocessor, C++ templates [..] and Racket [..], and other meta-programming techniques [..] are a similar area aiming at syntactic extensibility as well, which sacrifices type safety. SugarJ [] is another well-known tool that conveniently introduces syntactic sugars in Java programming by library imports. Composition of syntactic sugars is easy for users, whereas it requires many rounds of parsing and adaption, which highly affects efficiency of compilation, moreover, the implementation was based on SDF [] and Stratego [], which focused little on separate compilation. Extensible compilers like JastAdd [] and Polyglot [], however, are somehow more ambitious, but they involve extensible parsing mostly by parser generators as well, and they focus more on the extensions to a host language. Our library is designed for modular language parsers in a type-safe way, with flexible language composition. Although overriding existing production rules is tricky with our approach, we could perhaps make use of embedded domain-specific languages on top of parser combinators and transformations for overriding, but it is anyway an orthogonal issue. \haoyuan{???} \haoyuan{BTW what about Racket?} \haoyuan{what about metafront? it is a macro system but does it have type safety?} \haoyuan{Extensible syntax with lexical scoping?} \haoyuan{"Extensible syntax" proposes a system for extensible syntax, where users write EDSLs in their language with concrete syntax. Users can write rules for type-based disambiguation. But separate compilation is again not mentioned. Shall we mention that thesis?} \haoyuan{attribute grammars?}

On the other hand, extensible parsing algorithms are another area that indeed relates to separate compilation. Specifically, [MB] introduces \textit{parse table composition}, in which paper grammars are compiled to the generation of parse tables, which are DFAs or NFAs, later they can be composed by an algorithm, so as to provide separate compilation for parsing. Nevertheless, the generation of parse tables is rather expensive, and furthermore, our approach supports separate compilation as well as modular type-checking, and the idea is not restricted to Scala but applicable to many functional languages, on whose type system the safety of modular parsing can rely. The extensibility of parsing in our approach is further available at language composition and lexical level.
\haoyuan{I mentioned one shortcoming of our approach here, which is we cannot override existing production rules; another one is that we do not have explicit correspondence/relationship between abstract syntax and the parser.}

\paragraph*{Parser Combinators} Since [Recursive Programming Techniques 1975] firstly introduced parser generators, and after Wadler discussed more details on backtracking in [1985], parser combinators have been more and more popularly developed and used in the research area of parsing. Among them many parsing libraries produces recursive descent parsers by introducing functional, or more specifically, monadic parser combinators with a foundation on [monadic parser combinators]. Related work includes Parsec [], which is frequently used in Haskell for context-sensitive grammars with infinite lookahead. Nevertheless, left-recursion is known as a big issue in recursive descent parsers, in which case programmers using libraries like Parsec have to manually eliminate left-recursion in the grammar, which is cumbersome, but also it distorts the shape of its old grammar, restricting the modularity of parsing to extreme extent.
There are certainly some approaches to afford the limitation of parser combinators, for example, we have conducted our previous experiment in Haskell, by using open recursion, MRM [] and Parsec. To support direct left-recursive grammars, we made use of the state monad in Parsec to ensure that a left-recursive production rule will not be applied successively in parsing. It turned out to be successful, yet introduced complexity in programming, hence we believe that the parsing library is obligated for left-recursion.

Some more recent papers [Packrat 2006, parsing with derivatives 2011, parser combinator for ambiguous left-recursive grammars] proposed a series of novel parsing techniques, moreover, they did put an eye on the issue of left-recursion. As we explained before, we have selected Packrat Parsing for our prototype implementation, as the combinators are convenient to use in Scala, which is a platform where functional laziness and Object Algebras can perfectly coexist. Also, its support for direct left-recursion is already helpful for designing real-world parsers practically, though [packrat parsing can support left recursion] further demonstrated that general left-recursion is supported from the theoretical point of view. We have also mentioned that the components used in our library can have alternatives, for instance, a different set of parser combinators may support ambiguous and left-recursive grammars, or be applicable to a different subset of context-free grammars or more, or even lead us to another level of performance.

Another important issue in previous papers, to the best of our knowledge, is that none of them deal with modular datatypes or abstract syntax trees. \haoyuan{Well, I need to go to check JS and Patrick Bahr's paper.} The modularity of data structures is yet necessary for modular parsing that supports separate compilation and modular type-checking. In our library, Object Algebras have been adopted for extensible ASTs, together with the behaviors on them.

\paragraph*{Modular Datatypes} Text.


\section{Conclusion}\label{sec:conclusion}

\haoyuan{Huang: to write a couple of sentences for a summary.}

For a better user experience, there are certainly some aspects in our pattern for modular parsing that needs
improvements. Specifically, in Section~\ref{subsec:implementation} we have observed that the glue code in \lstinline{Untyped}
appears to be boilerplate, as we need to compose the data structures, parsers and operations respectively from the same set of super types. Such an issue refers to \textit{family polymorphism} [], and we may rely it on a language feature or meta-programming techniques. Moreover, to make better use of Object Algebras, we can possibly adopt some patterns in Shy, together with the composition of algebras, to reduce boilerplate by using code generation.

For future work, we will experiment more on how open recursion contributes to extensible parsing in functional languages, by making use of laziness. It is also interesting to see that parsing, to some extent, can be viewed as a special example of unfolds. So it is worthwhile considering to generalize the composition of unfolds under certain circumstances.

%===============================================================================

\bibliographystyle{plain}
\bibliography{paper}

\appendix

\end{document}

%%% Local Variables:
%%% mode: latex
%%% TeX-master: "."
%%% End:
